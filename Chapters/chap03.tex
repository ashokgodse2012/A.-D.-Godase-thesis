%
% File: chap01.tex
% Author: Victor F. Brena-Medina
% Description: Introduction chapter where the biology goes.
%
\let\textcircled=\pgftextcircled
\chapter{On the properties of k-Fibonacci and k-Lucas numbers}
\label{chap:On the properties of k-Fibonacci and k-Lucas numbers}
In this chapter, some properties of $k-$Fibonacci and $k-$Lucas sequences are derived and proved by using matrix methods. We defined matrices $S$, $M$, $M_{k}(n,m)$, $T_{k}(n)$, $S_{k}{(n,m)}$, $A_{n}$, $E$, $Y_{n}$, $W_{n}$, $G_{n}$ and $H_{n}$ for $k$-Fibonacci and $k$-Lucas sequences, using these matrices many interesting identities for $k-$Fibonacci and $k-$Lucas sequences are derived.
\vspace{2mm}
\let\thefootnote\relax\footnote{\textbf{\hspace{-0.78cm}The content of this chapter is published in the following papers.}}\footnotetext{\hspace{-0.78cm}Determinantal Identities for k Lucas Sequence, Journal of New Theory, 19(2018), 01-19.}
\footnotetext{\hspace{-0.78cm}Properties of k- Fibonacci Sequence Using Matrix Method, MAYFEB Journal of Mathematics, 12(2016), 01-07.}
\footnotetext{\hspace{-0.78cm}Summation identities for k-Fibonacci and k-Lucas numbers using matrix methods, Int. J. Adv. Appl. Math. and Mech. 5(2016), 74-80.}

\section{Introduction}
This chapter represents an interesting investigation about some special relations between matrices and $k$-Fibonacci sequences,$k$-Lucas sequences. This investigation is valuable to obtain new $k$-Fibonacci ,$k$-Lucas identities by different methods. This chapter contributes to $k$-Fibonacci,$k$-Lucas numbers literature, and encourage  to investigate the properties of such number sequences.
The most commonly used matrix in relation to the recurrence relation of $k$-Fibonacci sequence is
\begin{align}
 M = {\left[
          \begin{array}{cc}
            k & 1 \\
            1 & 0 \\
          \end{array}
        \right]},
\end{align}
which for $k=1$ reduces to the ordinary $Q$- matrix studied in literature of $k$-Fibonacci sequence. In next section,  we define more general matrices $M_{k}(n,m), T_{k}(n), S_{k}{(n,m)}$ for $Q$-matrix. We use these matrices to develop various summation identities involving terms from the numbers $F_{k,n}$ and $L_{k,n}$. Several identities for $F_{k,n}$ and $L_{k,n}$ are proved by many authors using Binet forms, some of these are listed below
\begin{align*}
&F_{k,n+1}+F_{k,n-1}= L_{k,n},\\ 
&F_{k,n+1}+L_{k,n-1}= \Delta F_{k,n}, \\
&F_{k,2n}-2(-1)^n= \Delta F_{k,n}^2, \\
&F_{k,m+n}-(-1)^{m}L_{k,n-m}= F_{k,m}L_{k,n}, \\ 
&L_{k,m+n}-(-1)^{m}L_{k,n-m}= \Delta F_{k,m}F_{k,n},\\  
&F_{k,m+n}F_{k,n-m}-F_{k,n}^{2}= (-1)^{n-m+1}F_{k,m}^2,\\  
&L_{k,m+n}L_{k,n-m}-L_{k,n}^{2}= (-1)^{n-m}F_{k,m}^2,\\  
&F_{k,m+n}F_{k,r+m}-(-1)^{m}F_{k,n}F_{k,r}= F_{k,m}F_{k,n+r+m},\\ 
&L_{k,mn}L_{k,n}+\Delta F_{k,mn}F_{k,n}= 2L_{k,(m+1)n},\\
&F_{k,mn}L_{k,n}+L_{k,mn}F_{k,n}= 2F_{k,(m+1)n},\\
&L_{k,m+n}^2+(-1)^{m-1}L_{k,n}^2 = \Delta F_{k,2n+m}F_{k,m},\\
&L_{k,m+n}L_{k,n}+(-1)^{m+1}L_{k,n-m}L_{k,n} = \Delta F_{k,2n}F_{k,m},\\
&F_{k,m+2rn}F_{k,2n+m}+(-1)^{m+1}F_{k,2rn}F_{k,2n} = F_{k,2(r+1)n+m}F_{k,m}.
\end{align*}
The matrix $M$ is generalized using principle of mathematical induction as 
\begin{align*}
 M^n = {
          \begin{bmatrix}
            F_{k,n+1} & F_{k,n} \\
                       F_{k,n} & F_{k,n-1}
          \end{bmatrix}
       },\quad\text{where $n$ is an integer.}
\end{align*}
\subsection{Properties of $k$-Fibonacci and $k$-Lucas sequences using matrices}

\begin{lemma} \label{3-l-1}
If $X$ is a square matrix with $X^2=kX+I$, then $ X^{n}=F_{k,n}X+F_{k,n-1}I$, for all $n\in Z$
\end{lemma}
\begin{proof}
If $n=0$ then result is obvious. If $n=1$ then
\begin{align*}
(X)^1&=F_{k,1}X+F_{k,0}I\\
		 &=1X+0I\\
		 &=X
\end{align*}		
Hence result is true for $n=1$. It can be shown by induction that,\\
%\begin{center}
$ X^{n}=F_{k,n}X+F_{k,n-1}I$, for all $n\in Z$.
%\end{center}
Assume that $X^{n}=F_{k,n}X+F_{k,n-1}I$ and prove that $X^{n+1}=F_{k,n+1}X+F_{k,n}I$.\\
Consider,
\begin{align*}
F_{k,n+1}X+F_{k,n}I&=(kF_{k,n}+F_{k,n-1})X+F_{k,n}I\\
									 &=(kX+I)F_{k,n}+XF_{k,n-1}\\
									 &=X^2F_{k,n}+XF_{k,n-1}\\
									 &=X(XF_{k,n}+F_{k,n-1})\\
									 &=X(X^{n})\\
									 &=X^{n+1}
\end{align*}		
Hence, $X^{n+1}=F_{k,n+1}X+F_{k,n}I$.
%\begin{center}
By first principle of Induction,\\ $X^{n}=F_{k,n}X+F_{k,n-1}I,$ for all $n\in Z$.\\
%\end{center}
%\begin{center}
Next, we show that $X^{-(n)}=F_{k,-n}X+F_{k,-n-1}I$, for all $n\in Z^{+}$.
%\end{center}
Let $Y=kI-X,$ then $XY=-I$, i. e. $Y=-X^{-1}$, now consider
\begin{align*}
								Y^2&=(kI-X)^2\\
									 &=k^2I-2kX+X^2\\
									 &=k^2I-2kX+kX+I\\
									 &=k^2I-kX+I\\
									 &=k(kI-X)+X+I\\
									 &=kY+I.
									 \end{align*}	
From first part, we obtain $$Y^n=F_{k,n}Y+F_{k,n-1}I,$$
It gives that
	\begin{align*}	
	(-X^{-1})^n&=F_{k,n}(kI-X)+F_{k,n-1}I\\
			 (-1)^nX^{-n}&=-F_{k,n}X+F_{k,n+1}I.
\end{align*}	
Since, $F_{k,-n}=(-1)^{n+1}F_{k,n}$, $F_{k,-n-1}=(-1)^{n}F_{k,n+1}$, gives $ X^{-(n)}=F_{k,-n}X+F_{k,-n-1}I,$ for all $n\in Z^{+}.$.
\end{proof}
\begin{corollary}
Let, $M={
 \begin{bmatrix}{}
    k & 1 \\
    1& 0 \\
	\end{bmatrix}
	}$,
	then $M^n={
 \begin{bmatrix}{}
    F_{k,n+1} & F_{k,n}\vspace{.2cm} \\
    F_{k,n} & F_{k,n-1} \\
	\end{bmatrix}
	}$.	
\end{corollary}
\begin{proof}
Since $M^2 = k M+I$, therefore using lemma(\ref{3-l-1}), we have
\begin{align*}
							M^n & = F_{k,n}M+F_{k,n-1}I\\
							 & = {
 \begin{bmatrix}{}
    kF_{k,n} & F_{k,n}\vspace{.2cm} \\
    F_{k,n} & 0 \\
	\end{bmatrix}
	}+{
 \begin{bmatrix}{}
    F_{k,n-1} & 0\vspace{.2cm} \\
    0 & F_{k,n-1} \\
	\end{bmatrix}
	}\\
		& = {
 \begin{bmatrix}{}
    F_{k,n+1} & F_{k,n}\vspace{.2cm} \\
    F_{k,n} & F_{k,n-1} \\
	\end{bmatrix}
	},\text{for all}\quad n\in Z.
\end{align*}
\end{proof}
\begin{corollary}
If, $S={
 \begin{bmatrix}{}
    \dfrac{k}{2} & \dfrac{k^2+4}{2} \vspace{.2cm}\\
    \dfrac{1}{2}& \dfrac{k}{2} \\
	\end{bmatrix}
	}, $
	then $S^n={
 \begin{bmatrix}{}
    \dfrac{L_{k,n}}{2} & \dfrac{(k^2+4)F_{k,n}}{2}\vspace{.2cm} \\
    \dfrac{F_{k,n}}{2} & \dfrac{L_{k,n}}{2} \\
	\end{bmatrix}
	}$, for every $n\in Z$.	
\end{corollary}
\begin{proof}
Since $$S^2=\begin{bmatrix}{}
    \dfrac{k^2+2}{2} & \dfrac{k(k^2+4)}{2} \vspace{.2cm}\\
    \dfrac{k}{2}& \dfrac{k^2+2}{2} \\
	\end{bmatrix}=kS+I.$$
	Using lemma (\ref{3-l-1}), we have
	$$S^n=F_{k,n}S+F_{k,n-1}I.$$
	It gives that
	$S^n={
 \begin{bmatrix}{}
    \dfrac{L_{k,n}}{2} & \dfrac{(k^2+4)F_{k,n}}{2}\vspace{.2cm} \\
    \dfrac{F_{k,n}}{2} & \dfrac{L_{k,n}}{2} \\
	\end{bmatrix}
	}$, for every $n\in Z$.
\end{proof}
\begin{lemma}For all $n\in Z$,
$L^2_{k,n}-(k^2+4)F^2_{k,n}=4(-1)^n.$ 
\end{lemma}
\begin{proof}
Since, $$det(S)=-1,$$
$$det(S^n)=(-1)^n.$$
Moreover since,
$$S^n={
 \begin{bmatrix}{}
    \dfrac{L_{k,n}}{2} & \dfrac{(k^2+4)F_{k,n}}{2} \vspace{.2cm}\\
    \dfrac{F_{k,n}}{2} & \dfrac{L_{k,n}}{2} \\
	\end{bmatrix}
	}.$$
	We get
	$$det(S^n)=\frac{L^2_{k,n}}{4}-\frac{(k^2+4)F^2_{k,n}}{4}.$$
Thus it follows that
$L^2_{k,n}-(k^2+4)F^2_{k,n}=4(-1)^n$, for all $n\in Z$.
\end{proof}
\begin{lemma}For $n,m\in Z$ \label{3-l-2}
\begin{align}
2L_{k,n+m}&=L_{k,n}L_{k,m}+(k^2+4)F_{k,n}F_{k,m},\\ 
2F_{k,n+m}&=F_{k,n}L_{k,m}+L_{k,n}F_{k,m}.
\end{align}
\end{lemma}
\begin{proof}
Since, \begin{align*}
S^{n+m}&=S^n\cdot S^m\\
									 &={
 \begin{bmatrix}{}
    \dfrac{L_{k,n}}{2} & \dfrac{(k^2+4)F_{k,n}}{2}\vspace{.2cm} \\
    \dfrac{F_{k,n}}{2} & \dfrac{L_{k,n}}{2} \\
	\end{bmatrix}
	}.{
 \begin{bmatrix}{}
    \dfrac{L_{k,m}}{2} & \dfrac{(k^2+4)F_{k,m}}{2}\vspace{.2cm} \\
    \dfrac{F_{k,m}}{2} & \dfrac{L_{k,m}}{2} \\
	\end{bmatrix}
	}\\
									 &={
 \begin{bmatrix}{}
    \dfrac{L_{k,n}L_{k,m}+(k^2+4)F_{k,n}F_{k,m}}{4} & \dfrac{(k^2+4)[L_{k,n}F_{k,m}+F_{k,n}L_{k,m}}{4}\vspace{.2cm} \\
    \dfrac{L_{k,n}F_{k,m}+F_{k,n}L_{k,m}}{4} & \dfrac{L_{k,n}L_{k,m}+(k^2+4)F_{k,n}F_{k,m}}{4} \\
	\end{bmatrix}
	}.
\end{align*}		
Also, we have
$$S^{n+m}={
 \begin{bmatrix}{}
    \dfrac{L_{k,n+m}}{2} & \dfrac{(k^2+4)F_{k,n+m}}{2} \vspace{.2cm}\\
    \dfrac{F_{k,n+m}}{2} & \dfrac{L_{k,n+m}}{2} \\
	\end{bmatrix}
	}.$$
	It gives that
\begin{align*}
2L_{k,n+m}&=L_{k,n}L_{k,m}+(k^2+4)F_{k,n}F_{k,m},\\ 
2F_{k,n+m}&=F_{k,n}L_{k,m}+L_{k,n}F_{k,m}, \text{for all}\quad n,m\in Z.
\end{align*}
\end{proof}
\begin{lemma}For $n,m\in Z$\label{3-l-3}
\begin{align}
2(-1)^mL_{k,n-m}&=L_{k,n}L_{k,m}-(k^2+4)F_{k,n}F_{k,m},\\
2(-1)^mF_{k,n-m}&=F_{k,n}L_{k,m}-L_{k,n}F_{k,m}.
\end{align}
\end{lemma}
\begin{proof}
Since
\begin{align*}
S^{n-m}&=S^n.S^{-m}\\
			 &=S^n.(S^{m})^{-1}\\
			 &=S^n.(-1)^m
 \begin{bmatrix}{}
    \dfrac{L_{k,m}}{2} & \dfrac{-(k^2+4)F_{k,m}}{2} \vspace{.2cm}\\
    \dfrac{-F_{k,m}}{2} & \dfrac{L_{k,m}}{2} \\
	\end{bmatrix}
	\\
	&=(-1)^m{
 \begin{bmatrix}{}
    \dfrac{L_{k,n}}{2} & \dfrac{(k^2+4)F_{k,n}}{2} \vspace{.2cm}\\
    \dfrac{F_{k,n}}{2} & \dfrac{L_{k,n}}{2} \\
	\end{bmatrix}
	}{
 \begin{bmatrix}{}
    \dfrac{L_{k,m}}{2} & \dfrac{-(k^2+4)F_{k,m}}{2}\vspace{.2cm} \\
    \frac{-F_{k,m}}{2} & \frac{L_{k,m}}{2} \\
	\end{bmatrix}
	}\\								
	&=(-1)^m{
 \begin{bmatrix}{}
    \dfrac{L_{k,n}L_{k,m}-(k^2+4)F_{k,n}F_{k,m}}{4} & \dfrac{(k^2+4)[L_{k,n}F_{k,m}-F_{k,n}L_{k,m}}{4}\vspace{.2cm} \\
    \dfrac{L_{k,n}F_{k,m}-F_{k,n}L_{k,m}}{4} & \dfrac{L_{k,n}L_{k,m}-(k^2+4)F_{k,n}F_{k,m}}{4} \\
	\end{bmatrix}
	}.
\end{align*}		
But, we have
$$S^{n-m}={
 \begin{bmatrix}{}
    \dfrac{L_{k,n-m}}{2} & \dfrac{(k^2+4)F_{k,n-m}}{2}\vspace{.2cm} \\
    \dfrac{F_{k,n-m}}{2} & \dfrac{L_{k,n-m}}{2} \\
	\end{bmatrix}
	}.$$
	It gives that
	\begin{align*}
	2(-1)^mL_{k,n-m}&=L_{k,n}L_{k,m}-(k^2+4)F_{k,n}F_{k,m},\\
2(-1)^mF_{k,n-m}&=F_{k,n}L_{k,m}-L_{k,n}F_{k,m}, \text{for all}\quad n,m\in Z.
	\end{align*}
\end{proof}
\begin{lemma}For all $n,m\in Z$\label{3-l-4}
\begin{align}
(-1)^mL_{k,n-m}+L_{k,n+m}&=L_{k,n}L_{k,m},\\
(-1)^mF_{k,n-m}+F_{k,n+m}&=F_{k,n}L_{k,m}.
\end{align}
\end{lemma}
\begin{proof}
using definition of the matrix $S^n$, it can be seen that 
$$S^{n+m}+(-1)^mS^{n-m}={
 \begin{bmatrix}{}
    \dfrac{L_{k,n+m}+(-1)^mL_{k,n-m}}{2} & \dfrac{(k^2+4)[F_{k,n+m}+(-1)^mF_{k,n-m}}{2}\vspace{.2cm} \\
    \dfrac{F_{k,n+m}+(-1)^mF_{k,n-m}}{2} & \dfrac{L_{k,n+m}+(-1)^mL_{k,n-m}}{2} \\
	\end{bmatrix}
	}.$$
	On the other hand, we have
	\begin{align*}
	S^{n+m}+(-1)^mS^{n-m}&=S^nS^m+(-1)^mS^nS^{-m}\\
											 &=S^n[S^m+(-1)^mS^{-m}]\\
											 &={
 \begin{bmatrix}{}
    \dfrac{L_{k,n}}{2} & \dfrac{(k^2+4)F_{k,n}}{2} \vspace{.2cm}\\
    \dfrac{F_{k,n}}{2} & \dfrac{L_{k,n}}{2} \\
	\end{bmatrix}
	}\langle{
 \begin{bmatrix}{}
    \dfrac{L_{k,m}}{2} & \dfrac{(k^2+4)F_{k,m}}{2}\vspace{.2cm} \\
    \dfrac{F_{k,m}}{2} & \dfrac{L_{k,m}}{2} \\
	\end{bmatrix}
	}\\&+(-1)^m{
 \begin{bmatrix}{}
    \dfrac{L_{k,m}}{2} & \dfrac{-(k^2+4)F_{k,m}}{2} \vspace{.2cm}\\
    \dfrac{-F_{k,m}}{2} & \dfrac{L_{k,m}}{2} \\
	\end{bmatrix}
	}\rangle\\			
					&={
 \begin{bmatrix}{}
    \dfrac{L_{k,n}}{2} & \dfrac{(k^2+4)F_{k,n}}{2}\vspace{.2cm} \\
    \dfrac{F_{k,n}}{2} & \dfrac{L_{k,n}}{2} \\
	\end{bmatrix}
	}.
 \begin{bmatrix}{}
    L_{k,m} & 0 \vspace{.2cm}\\
    0 & L_{k,m} \\
		\end{bmatrix}
	\\
		&={
 \begin{bmatrix}{}
    \dfrac{L_{k,m}L_{k,n}}{2} & \dfrac{(k^2+4)F_{k,n}L_{k,m}}{2}\vspace{.2cm} \\
    \dfrac{F_{k,n}L_{k,m}}{2} & \dfrac{L_{k,m}L_{k,n}}{2}\\
	\end{bmatrix}
	}.
	\end{align*}
	It gives that
\begin{align*}
(-1)^mL_{k,n-m}+L_{k,n+m}&=L_{k,n}L_{k,m},\\
(-1)^mF_{k,n-m}+F_{k,n+m}&=F_{k,n}L_{k,m}, \text{for all}\quad n,m\in Z.
\end{align*}
\end{proof}
\begin{lemma}For all $x,y,z\in Z$
\begin{align}
8F_{k,x+y+z}&=L_{k,x}L_{k,y}F_{k,z}+F_{k,x}L_{k,y}L_{k,z}+L_{k,x}F_{k,y}L_{k,z}+(k^2+4)F_{k,x}F_{k,y}F_{k,z},\\
8L_{k,x+y+z}&=L_{k,x}L_{k,y}L_{k,z}+(k^2+4)[L_{k,x}F_{k,y}F_{k,z}+F_{k,x}L_{k,y}F_{k,z}+F_{k,x}F_{k,y}L_{k,z}.
\end{align}
\end{lemma}
\begin{proof}
Using definition of the matrix $S^n$, it can be seen that 
$$S^{x+y+z}={
 \begin{bmatrix}{}
    \dfrac{L_{k,x+y+z}}{2} & \dfrac{(k^2+4)F_{k,x+y+z}}{2}\vspace{.2cm} \\
    \dfrac{F_{k,x+y+z}}{2}& \dfrac{L_{k,x+y+z}}{2} \\
	\end{bmatrix}
	}.$$
	On the other hand, we have
	\begin{align*}
	S^{x+y+z}&=S^{x+y}S^z\\
 &={
 \begin{bmatrix}{}
    \dfrac{L_{k,x+y}}{2} & \dfrac{(k^2+4)F_{k,x+y}}{2}\vspace{.2cm} \\
    \dfrac{F_{k,x+y}}{2} & \dfrac{L_{k,x+y}}{2} \\
	\end{bmatrix}
	}.{
 \begin{bmatrix}{}
    \frac{L_{k,z}}{2} & \frac{(k^2+4)F_{k,z}}{2}\vspace{.2cm} \\
    \frac{F_{k,z}}{2} & \frac{L_{k,z}}{2} \\
	\end{bmatrix}
	}\\
 &={
 \begin{bmatrix}{}
    \dfrac{L_{k,x+y}L_{k,z}+(k^2+4)F_{k,x+y}F_{k,z}}{4} & \dfrac{(k^2+4)[L_{k,x+y}F_{k,z}+F_{k,x+y}L_{k,z}}{4}\vspace{.2cm} \\
    \dfrac{L_{k,z}F_{k,x+y}+F_{k,z}L_{k,x+y}}{4} & \dfrac{L_{k,x+y}L_{k,z}+(k^2+4)F_{k,x+y}F_{k,z}}{4} \\
	\end{bmatrix}
	}
	\end{align*}
	Using lemma(\ref{3-l-2}),
	$$2L_{k,x+y}=L_{k,x}L_{k,y}+(k^2+4)F_{k,x}F_{k,y}$$
	$$2F_{k,x+y}=L_{k,y}F_{k,x}+(k^2+4)F_{k,y}L_{k,x}.$$
We get
\begin{align*}
&8F_{k,x+y+z}=L_{k,x}L_{k,y}F_{k,z}+F_{k,x}L_{k,y}L_{k,z}+L_{k,x}F_{k,y}L_{k,z}+(k^2+4)F_{k,x}F_{k,y}F_{k,z},\\ 
&8L_{k,x+y+z}=L_{k,x}L_{k,y}L_{k,z}+(k^2+4)[L_{k,x}F_{k,y}F_{k,z}+F_{k,x}L_{k,y}F_{k,z}+F_{k,x}F_{k,y}L_{k,z}, \\&\text{for all} x,y,z\in Z.
\end{align*}
\end{proof}
\begin{theorem}
For all $x,y,z\in Z$
\begin{align*}
&L^2_{k,x+y}-(k^2+4)(-1)^{x+y+1}F_{k,z-x}L_{k,x+y}F_{k,y+z}-(k^2+4)(-1)^{x+z}F^2_{k,y+z}\\&=(-1)^{y+z}L^2_{k,z-x}.
\end{align*}
\end{theorem}
\begin{proof}
Consider the  matrix multiplication
 $${
 \begin{bmatrix}{}
    \dfrac{L_{k,x}}{2} & \dfrac{(k^2+4)F_{k,x}}{2} \vspace{.2cm}\\
    \dfrac{F_{k,z}}{2} & \dfrac{L_{k,z}}{2} \\
	\end{bmatrix}
	}.{
 \begin{bmatrix}{}
    L_{k,y} \vspace{.2cm}\\
    F_{k,y} \\
	\end{bmatrix}
	}={
 \begin{bmatrix}{}
    L_{k,x+y} \vspace{.2cm}\\
    F_{k,y+z} \\
	\end{bmatrix}
	}.$$
	Also
		\begin{align*}
	det.{
 \begin{bmatrix}{}
    \dfrac{L_{k,x}}{2} & \dfrac{(k^2+4)F_{k,x}}{2}\vspace{.2cm} \\
    \dfrac{F_{k,z}}{2} & \dfrac{L_{k,z}}{2} \\
	\end{bmatrix}
	}&=\dfrac{L_{k,x}L_{k,z}-(k^2+4)F_{k,x}F_{k,z}}{4}\\
					&=\frac{(-1)^xL_{k,z-x}}{2}\\
					&=Q\\
					&\neq0.
	\end{align*}
	Hence, we can write
	\begin{align*}
	{
 \begin{bmatrix}{}
    L_{k,y} \vspace{.2cm}\\
    F_{k,y} \\
	\end{bmatrix}
	}&={
 \begin{bmatrix}{}
    \dfrac{L_{k,x}}{2} & \dfrac{(k^2+4)F_{k,x}}{2}\vspace{.2cm} \\
    \dfrac{F_{k,z}}{2} & \dfrac{L_{k,z}}{2} \\
	\end{bmatrix}
	^{-1}}.{
 \begin{bmatrix}{}
    L_{k,x+y} \vspace{.2cm}\\
    F_{k,y+z} \\
	\end{bmatrix}
	}\\
	&=\dfrac{1}{Q}{
 \begin{bmatrix}{}
    \dfrac{L_{k,z}}{2} & \dfrac{-(k^2+4)F_{k,x}}{2}\vspace{.2cm} \\
    \dfrac{-F_{k,z}}{2} & \dfrac{L_{k,x}}{2} \\
	\end{bmatrix}
	}.{
 \begin{bmatrix}{}
    L_{k,x+y}\vspace{.2cm} \\
    F_{k,y+z} \\
	\end{bmatrix}
	}.	
	\end{align*}
	It gives that
	\begin{align*}
	L_{k,y}&=\dfrac{(-1)^x[L_{k,z}L_{k,x+y}-(k^2+4)F_{k,x}F_{k,y+z}]}{L_{k,z-x}},\\
	F_{k,y}&=\dfrac{(-1)^x[L_{k,x}F_{k,z+y}-F_{k,z}L_{k,y+x}]}{L_{k,z-x}}.
	\end{align*}	
Since
$$L^2_{k,y}-(k^2+4)F^2_{k,y}=4(-1)^y.$$
We get
$$[L_{k,z}L_{k,x+y}-(k^2+4)F_{k,x}F_{k,y+z}]^2-(k^2+4)^2[L_{k,x}F_{k,z+y}-F_{k,z}L_{k,y+x}]^2=4(-1)^yL^2_{k,z-x}.$$
Using lemmas(\ref{3-l-3}, \ref{3-l-4}), we get 
\begin{align*}
&(L^2_{k,z}L^2_{k,x+y}-2(k^2+4)L_{k,z}F_{k,x+y}F_{k,y+z}+(k^2+4)^2F^2_{k,x}F^2_{k,y+z})\\&-(k^2+4)(L^2_{k,x}F^2_{k,y+z}-2L_{k,x}F_{k,z}F_{k,y+z}L_{k,x+y}+F^2_{k,z}L^2_{k,x+y})=4(-1)^yL^2_{k,z-x}.
\end{align*}
It gives that
\begin{align*}
&L^2_{k,x+y}-(k^2+4)(-1)^{x+y+1}F_{k,z-x}L_{k,x+y}F_{k,y+z}-(k^2+4)(-1)^{x+z}F^2_{k,y+z}\\&=(-1)^{y+z}L^2_{k,z-x}, \text{for all} x,y,z\in Z.
\end{align*}
\end{proof}
\begin{theorem}For all $x,y,z\in Z $, $x\neq z$
\begin{align*}
L^2_{k,x+y}-(-1)^{x+z}L_{k,z-x}L_{k,x+y}L_{k,y+z}+(-1)^{x+z}L^2_{k,y+z}=(-1)^{y+z+1}(k^2+4)F^2_{k,z-x}.
\end{align*}
\end{theorem}
\begin{proof}
Consider matrix multiplication
$${
 \begin{bmatrix}{}
    \dfrac{L_{k,x}}{2} & \dfrac{(k^2+4)F_{k,x}}{2}\vspace{.2cm} \\
    \dfrac{L_{k,z}}{2} & \dfrac{(k^2+4)F_{k,z}}{2} \\
	\end{bmatrix}
	}.{
 \begin{bmatrix}{}
    L_{k,y} \vspace{.2cm}\\
    F_{k,y} \\
	\end{bmatrix}
	}={
 \begin{bmatrix}{}
    L_{k,x+y}\vspace{.2cm} \\
    L_{k,y+z} \\
	\end{bmatrix}
	}.$$
	Also, we have	
	\begin{align*}
	det{
 \begin{bmatrix}{}
    \dfrac{L_{k,x}}{2} & \dfrac{(k^2+4)F_{k,x}}{2}\vspace{.2cm} \\
    \dfrac{L_{k,z}}{2} & \dfrac{(k^2+4)F_{k,z}}{2} \\
	\end{bmatrix}
	}&=\dfrac{(k^2+4)(-1)^xF_{k,z-x}}{2}\\
					&=P\\
					&\neq0,\text{(if $x\neq z$)}.
	\end{align*}
	Therefore, for $x\neq z,$ we can write
	\begin{align*}
	{
 \begin{bmatrix}{}
    L_{k,y}\vspace{.2cm} \\
    F_{k,y} \\
	\end{bmatrix}
	}&={
 \begin{bmatrix}{}
    \dfrac{L_{k,x}}{2} & \dfrac{(k^2+4)F_{k,x}}{2}\vspace{.2cm} \\
    \dfrac{L_{k,z}}{2} & \dfrac{(k^2+4)F_{k,z}}{2} \\
	\end{bmatrix}
	^{-1}}.{
 \begin{bmatrix}{}
    L_{k,x+y}\vspace{.2cm} \\
    L_{k,y+z} \\
	\end{bmatrix}
	}\\
	&=\dfrac{1}{P}{
 \begin{bmatrix}{}
    \dfrac{(k^2+4)F_{k,z}}{2} & \dfrac{-(k^2+4)F_{k,x}}{2}\vspace{.2cm} \\
    \dfrac{-L_{k,z}}{2} & \dfrac{L_{k,x}}{2} \\
	\end{bmatrix}
	}.{
 \begin{bmatrix}{}
    L_{k,x+y}\vspace{.2cm} \\
    L_{k,y+z} \\
	\end{bmatrix}
	}.
	\end{align*}
	It gives that
	\begin{align*}
	L_{k,y}=\dfrac{(-1)^x[F_{k,z}L_{k,x+y}-F_{k,x}L_{k,y+z}]}{F_{k,z-x}},\\
	F_{k,y}=\dfrac{(-1)^x[L_{k,x}L_{k,z+y}-L_{k,z}L_{k,y+x}]}{(k^2+4)F_{k,z-x}}.
	\end{align*}
Since
$$L^2_{k,y}-(k^2+4)F^2_{k,y}=4(-1)^y.$$
We get
\begin{align*}
&(k^2+4)[F_{k,z}L_{k,x+y}-F_{k,x}L_{k,y+z}]^2-[L_{k,x}L_{k,z+y}-L_{k,z}L_{k,y+x}]^2\\&=4(k^2+4)(-1)^yF^2_{k,z-x},\\
&L^2_{k,x+y}-(-1)^{x+z}L_{k,z-x}L_{k,x+y}L_{k,y+z}+(-1)^{x+z}L^2_{k,y+z}\\&=(-1)^{y+z+1}(k^2+4)F^2_{k,z-x}, \text{for all} x,y,z\in Z ,x\neq z.
\end{align*}
\end{proof}
\begin{theorem}For all $x,y,z\in Z $, $x\neq z$
\begin{align*}
F^2_{k,x+y}-L_{k,x-z}F_{k,x+y}F_{k,y+z}+(-1)^{x+z}F^2_{k,y+z}=(-1)^{y+z}F^2_{k,z-x}. 
\end{align*}
\end{theorem}
\begin{proof}
Consider the matrix multiplication
$${
 \begin{bmatrix}{}
    \dfrac{F_{k,x}}{2} & \dfrac{F_{k,x}}{2}\vspace{.2cm} \\
    \dfrac{F_{k,z}}{2} & \dfrac{L_{k,z}}{2} \\
	\end{bmatrix}
	}{
 \begin{bmatrix}{}
    L_{k,y}\vspace{.2cm} \\
    F_{k,y} \\
	\end{bmatrix}
	}={
 \begin{bmatrix}{}
    F_{k,x+y}\vspace{.2cm} \\
    F_{k,y+z} \\
	\end{bmatrix}}.$$
	Also, we have	
	\begin{align*}
	{
 \begin{vmatrix}{}
    \dfrac{F_{k,x}}{2} & \dfrac{F_{k,x}}{2}\vspace{.2cm} \\
    \dfrac{F_{k,z}}{2} & \dfrac{L_{k,z}}{2} \\
	\end{vmatrix}
	}&=\dfrac{(-1)^zF_{k,x-z}}{2}\\
					&=R\\
					&\neq0, (if x\neq z).
	\end{align*}
	Therefore, for $x\neq z$, we get
	\begin{align*}
	{
 \begin{bmatrix}{}
    L_{k,y}\vspace{.2cm} \\
    F_{k,y} \\
	\end{bmatrix}
	}&={
 \begin{bmatrix}{}
    \dfrac{F_{k,x}}{2} & \dfrac{F_{k,x}}{2}\vspace{.2cm} \\
    \dfrac{F_{k,z}}{2} & \dfrac{L_{k,z}}{2} \\
	\end{bmatrix}
	^{-1}}.{
 \begin{bmatrix}{}
    F_{k,x+y}\vspace{.2cm} \\
    F_{k,y+z} \\
	\end{bmatrix}
	}\\
	&=\dfrac{1}{R}{
 \begin{bmatrix}{}
    \dfrac{L_{k,z}}{2} & \dfrac{-L_{k,x}}{2} \vspace{.2cm}\\
    \dfrac{-F_{k,z}}{2} & \dfrac{F_{k,x}}{2} \\
	\end{bmatrix}
	}.{
 \begin{bmatrix}{}
    F_{k,x+y} \vspace{.2cm}\\
    F_{k,y+z} \\
	\end{bmatrix}
	}.	
	\end{align*}
	It gives that
	\begin{align*}
	L_{k,y}=\frac{(-1)^z[L_{k,z}F_{k,x+y}-L_{k,x}F_{k,y+z}]}{F_{k,x-z}},\\
	F_{k,y}=\frac{(-1)^z[F_{k,x}F_{k,z+y}-F_{k,z}F_{k,y+x}]}{F_{k,x-z}}.
	\end{align*}
	Now consider
	\begin{align*}
	[L_{k,z}F_{k,x+y}-L_{k,x}F_{k,y+z}]^2-(k^2+4)[F_{k,x}F_{k,z+y}-F_{k,z}F_{k,y+x}]^2=4(-1)^yF^2_{k,x-z}.
	\end{align*}
Hence, we get 
\begin{align*}
&F^2_{k,x+y}-L_{k,x-z}F_{k,x+y}F_{k,y+z}+(-1)^{x+z}F^2_{k,y+z}\\&=(-1)^{y+z}F^2_{k,z-x}, \text{for all} x,y,z\in Z, x\neq z.
\end{align*}

\end{proof}
\subsection{Summation Identities for $k$-Fibonacci and $k$-Lucas numbers using matrix methods}
In this section, we define general matrices $M_{k}(n,m)$, $T_{k,n}$ and $S_{k}{(n,m)}$ for k-Fibonacci number. Using these matrices we find some new summation properties for k-Fibonacci and k-Lucas numbers. 
\subsubsection{{The Matrix $M_{k}(n,m)$ }}
First we give a generalization of the matrix $M$ and use it to produce summation identities involving terms of the sequences $F_{k,n}$ and $L_{k,n}$. 
\begin{definition}\label{a3}
\begin{align}
 M_{k}(n,m) =\begin{bmatrix}
            F_{k,n+m} & (-1)^{m+1}F_{k,n}\bigskip \\
            F_{k,n} & (-1)^{m+1}F_{k,n-m}
          \end{bmatrix}, \quad \text{where $m$ and $n$ are integers.}
        \end{align}
\end{definition}
\begin{theorem} Let $M_{k}(n,m)$ be a matrix as in (\ref{a3}) then 
\begin{align*}
 M_{k}(n,m)^r = F_{k,m}^{r}{
          \begin{bmatrix}
                     F_{k,rn+m} & (-1)^{m+1}F_{k,rn}\bigskip \\
            F_{k,rn} & (-1)^{m+1}F_{k,rn-m} 
          \end{bmatrix}}.
\end{align*}
\end{theorem}
\begin{proof}
We use principle of Mathematical induction (P. M. I.). It is clear that the result is true for $r=1$. Assume that the result is true for $r$. i. e.
\begin{align*}
&M_{k}(n,m)^r = F_{k,m}^{r}{
          \begin{bmatrix}
            F_{k,rn+m} & (-1)^{m+1}F_{k,rn}\bigskip \\
            F_{k,rn} & (-1)^{m+1}F_{k,rn-m} 
          \end{bmatrix}
     }.\\				
&\text{Now consider}\\
&M_{k}(n,m)^{r+1}= M_{k}(n,m)^{r}M_{k}(n,m),\\
&= F_{k,m}^{r}{
          \begin{bmatrix}
            F_{k,rn+m} & (-1)^{m+1}F_{k,rn} \bigskip\\
            F_{k,rn} & (-1)^{m+1}F_{k,rn-m} 
          \end{bmatrix}
        }\cdot {
          \begin{bmatrix}
            F_{k,n+m} & (-1)^{m+1}F_{k,n}\bigskip \\
            F_{k,n} & (-1)^{m+1}F_{k,n-m} 
          \end{bmatrix}
        },\\
& = F_{k,m}^{r}{
          \begin{bmatrix}
F_{k,rn+m}F_{k,n+m}+(-1)^{m+1}F_{k,rn}F_{k,n} & (-1)^{m+1}F_{k,rn+m}F_{k,n}+F_{k,rn}F_{k,n-m} \bigskip\\
F_{k,rn}F_{k,n+m}+(-1)^{m+1}F_{k,rn-m}F_{k,n} & (-1)^{m+1}F_{k,rn}F_{k,n}+F_{k,rn-m}F_{k,n-m} 
\end{bmatrix}
},\\
&= F_{k,m}^{r+1}{
          \begin{bmatrix}
            F_{k,(r+1)n+m} & (-1)^{m+1}F_{k,(r+1)n} \bigskip\\
            F_{k,(r+1)n} & (-1)^{m+1}F_{k,(r+1)n-m} 
          \end{bmatrix}
        }.
\end{align*}				
Hence proof.
\end{proof}	
\begin{align*}	
&\text{We find that characteristic equation of $M_{k}(n,m)$ is}\quad
	\lambda^2-F_{k,m}F_{k,n}\lambda+(-1)^nF_{k,m}^2=0,\\	
	&\text{and by Cauchy-Hamilton theorem}\quad
	M_{k}(n,m)^2-F_{k,m}F_{k,n}M_{k}(n,m)+(-1)^nF_{k,m}^2I=0.\\
	&\text{Multiplying both sides of equation (25) by $M_{k}(n,m)^{t}$ gives}\\
&(F_{k,m}F_{k,n}M_{k}(n,m)-(-1)^nF_{k,m}^2I)^rM_{k}(n,m)^t
	=M_{k}(n,m)^{2r+t},\\
 &\text{and expanding gives} 
	\quad\sum_{i=0}^{i=r}\left(^{r} _{i} \right)(-1)^{(r-1)(n+1)}F_{k,m}^{2r-1}F_{k,n}^{i}M_{k}(n,m)^{i+t}=M_{k}(n,m)^{2r+t}.\\	
&\text{Using (18) to equate upper left entries gives}\\
&\sum_{i=0}^{i=r}\left(^{r} _{i} \right)(-1)^{(r-1)(n+1)}L_{k,n}^{i}F_{k,(i+t)n+m}=F_{k,(2r+t)n+m}.\\	
&\text{In similar way, we can obtain}\\
&\sum_{i=0}^{i=r}\left(^{r} _{i} \right)(-1)^{n(r-i)}F_{k,2in+m}=L_{k,n}^rF_{k,rn+m},\\
	&\sum_{i=0}^{i=2r}\left(^{2r} _{i} \right)(-1)^{2nr-i(n-1)}F_{k,2in+m}=\Delta^rF_{k,n}^{2r}F_{k,2rn+m},\\	
&\sum_{i=0}^{i=2r+1}\left(^{2r+1} _{i} \right)(-1)^{n(2r-i+1)+i+1}F_{k,2in+m}= \Delta^rF_{k,n}^{2r+1}L_{k,(2r+1)n+m},\\
&\sum_{i=0}^{i=2r}\left(^{2r} _{i} \right)(-1)^{i}2^iL_{k,n}^{2r-i}F_{k,in+m}=\Delta^rF_{k,n}^{2r}F_{k,m}.
\end{align*}	
\subsubsection{{The Matrix $T_{k,n}$ }}
We now give another generalization of the matrix $M$ and use it to produce summation identities involving terms of the sequences $F_{k,n}$ and $L_{k,n}$. 
\begin{definition}\label{a4}
\begin{align}
 T_{k,n} = {\begin{bmatrix}
            L_{k,n} & F_{k,n}\bigskip \\
            \Delta F_{k,n} & L_{k,n} 
          \end{bmatrix}
     }, \quad \text{where  $n$ is an integer.}
\end{align}
\end{definition}
\begin{theorem} Let $T_{k,n}$ be a matrix as in (\ref{a4}) then 
\begin{align}
 T_{k,n}^m = 2^{m-1}{
          \begin{bmatrix}
            L_{k,nm} & F_{k,nm} \bigskip\\
            \Delta F_{k,nm} & L_{k,nm} 
          \end{bmatrix}
        }.
\end{align}
\end{theorem}
\begin{proof}:
We use principle of Mathematical induction on $m$. It is clear that the result is true for $m=1$. Assume that the result is true for $m$.
\begin{align*}
&T_{k,n}^m = 2^{m-1}{
          \begin{bmatrix}
            L_{k,nm} & F_{k,nm}\bigskip \\
            \Delta F_{k,nm} & L_{k,nm} 
          \end{bmatrix}
        }.\\
&\text{Now consider}\\
&T_{k,n}^{m+1}=T_{k,n}^mT_{k,n}=2^{m-1}{
          \begin{bmatrix}
            L_{k,nm} & F_{k,nm}\bigskip \\
            \Delta F_{k,nm} & L_{k,nm} 
          \end{bmatrix}
        }\cdot {
          \begin{bmatrix}
            L_{k,n} & F_{k,n} \bigskip\\
            \Delta F_{k,n} & L_{k,n} 
          \end{bmatrix}
        }\\
        &=2^{m-1}{
          \begin{bmatrix}
            L_{k,mn}L_{k,n}+\Delta F_{k,nm}F_{k,n} & L_{k,mn}F_{k,n}+F_{k,mn}L_{k,n}\bigskip \\
						\Delta (L_{k,mn}F_{k,n}+\\F_{k,mn}L_{k,n}) & L_{k,mn}L_{k,n}+\Delta F_{k,nm}F_{k,n} 
					 \end{bmatrix}
						}\\			
					&= 2^m{
          \begin{bmatrix}
            L_{k,n(m+1)} & F_{k,n(m+1)}\bigskip \\
            \Delta F_{k,n(m+1)} & L_{k,n(m+1)} 
          \end{bmatrix}
        }.
        \end{align*}
Hence proof.
\end{proof}	
\noindent The characteristic equation of $T_{k,n}$ is
	\begin{align*}
	\lambda^2-2L_{k,n}\lambda+4(-1)^n=0.
	\end{align*}
	By Cauchy-Hamilton theorem
	\begin{align*}
	T_{k,n}^2-2L_{k,n}T_{k,n}+4(-1)^nI=0.
	\end{align*}
It gives that
\begin{align*}
 T_{k,n}^mT_{k,t} = 2^{m}{\left[
          \begin{array}{cc}
            L_{k,nm+t} & F_{k,nm+t}\bigskip \\
            \Delta F_{k,nm+t} & L_{k,nm+t} 
          \end{array}
        \right]}.
\end{align*}
Consider the case $n=1$,
\begin{align}
	T_{k,1}^m=2^{m-1}(F_{k,m}T_{k,1}+2F_{k,m-1}I, \quad \text{where $m\geq 2$.}
	\end{align}
It produces
\begin{align}
	\sum_{i=0}^{i=r}\left(^{r} _{i} \right)F_{k,n-1}^{r-i}F_{k,n}^iF_{k,i+s+t}	=F_{k,nr+s+t}.	
\end{align}
The methods applied to $M_{k}(n,m)$ in previous section when applied to $T_{k,n}$ produce most of the summation identities that we have obtained so far.
\subsubsection{{The Matrix $S_{k}(n,m)$} }
We now give one more generalization of the matrix $M$ and use it to produce summation identities involving terms of the sequences $F_{k,n}$ and $L_{k,n}$. 
\begin{definition}
\begin{align}\label{sknm}
 S_{k}(n,m) = {\begin{bmatrix}
            L_{k,n+m} & (-1)^{m+1}L_{k,n} \bigskip\\
            L_{k,n} & (-1)^{m+1}L_{k,n-m} 
          \end{bmatrix}
        }, \quad \text{where  $n$, $m$ are integers.}
\end{align}
\end{definition}
\begin{theorem} Let $S_{k}(n,m)$ be a matrix as in (\ref{sknm}) then for all integer $r$ 
\begin{align*}
 &S_{k}(n,m)^{2r}= F_{k,m}^{2r-1}\Delta^r{
          \begin{bmatrix}
            F_{k,2rn+m} & (-1)^{m+1}F_{k,2rn}\bigskip \\
            F_{k,2rn} & (-1)^{m+1}F_{k,2rn-m} 
          \end{bmatrix}
      },\\&S_{k}(n,m)^{2r-1}= F_{k,m}^{2r-2}\Delta^{r-1}{\
          \begin{bmatrix}
            L_{k,2(r-1)n+m} & (-1)^{m+1}L_{k,2(r-1)n} \bigskip\\
            L_{k,2(r-1)n} & (-1)^{m+1}L_{k,2(r-1)n-m} 
          \end{bmatrix}
      }.
    \end{align*}    
\end{theorem}
\begin{proof}
Using Principle of Mathematical induction, for $r=1$. 
\begin{align*}
&S_{k}(n,m)^2= {
          \begin{bmatrix}
            L_{k,n+m} & (-1)^{m+1}L_{k,n} \bigskip\\
            L_{k,n} & (-1)^{m+1}L_{k,n-m} 
          \end{bmatrix}
     }\cdot {
          \begin{bmatrix}
            L_{k,n+m} & (-1)^{m+1}L_{k,n}\bigskip \\
            L_{k,n} & (-1)^{m+1}L_{k,n-m} 
          \end{bmatrix}
        }\\&={
          \begin{bmatrix}
            L_{k,n+m}^2\\+(-1)^{m+1}L_{k,n}^2 & (-1)^{m+1}L_{k,n+m}L_{k,n}+L_{k,n}L_{k,n-m} \bigskip\\
            L_{k,n+m}L_{k,n}\\+(-1)^{m+1}L_{k,n}L_{k,n-m} & (-1)^{m+1}L_{k,n+m}^2+L_{k,n}^2 
          \end{bmatrix}
        }\\&=F_{k,m}\Delta {
          \begin{bmatrix}
            F_{k,2n+m} & (-1)^{m+1}F_{k,2n}\bigskip \\
            F_{k,2n} & (-1)^{m+1}F_{k,2n-m} 
          \end{bmatrix}
        }.\\
        &\text{The result is true for $r=1$. Assume that the result is true for $r$ (IH)}.\\
&S_{k}(n,m)^{2r} = F_{k,m}^{2r-1}\Delta^r{
          \begin{bmatrix}
            F_{k,2rn+m} & (-1)^{m+1}F_{k,2rn}\bigskip \\
            F_{k,2rn} & (-1)^{m+1}F_{k,2rn-m} 
          \end{bmatrix}
       }.\\
&\text{Now consider}\\
&S_{k}(n,m)^{2r+2}=S_{k}(n,m)^{2r}S_{k}(n,m)^2\\&= F_{k,m}^{2r-1}\Delta^r{
          \begin{bmatrix}
            F_{k,2rn+m} & (-1)^{m+1}F_{k,2rn} \bigskip\\
            F_{k,2rn} & (-1)^{m+1}F_{k,2rn-m} 
          \end{bmatrix}
       }\cdot F_{k,m}\Delta {\
          \begin{bmatrix}
            F_{k,2n+m} & (-1)^{m+1}F_{k,2n}\bigskip \\
            F_{k,2n} & (-1)^{m+1}F_{k,2n-m} 
          \end{bmatrix}
        }\\&= F_{k,m}^{2r}\Delta^{r+1}
				\cdot{
          \begin{bmatrix}
            F_{k,2rn+m}F_{k,2n+m}\\+(-1)^{m+1}F_{k,2rn}F_{k,2n} & (-1)^{m+1}F_{k,2rn+m}F_{k,2n}\\&+F_{k,2rn}F_{k,2rn-m}\bigskip \\
						F_{k,2rn}F_{k,2n+m}\\+(-1)^{m+1}F_{k,2rn-m}F_{k,2n} & (-1)^{m+1}F_{k,2rn}F_{k,2n}\\&+F_{k,2rn-m}F_{k,2n-m} 
					 \end{bmatrix}
						}\\
&=F_{k,m}^{2r+1}\Delta^{r+1}{
          \begin{bmatrix}
            F_{k,2(r+1)n+m} & (-1)^{m+1}F_{k,2(r+1)n}\bigskip \\
            F_{k,2(r+1)n} & (-1)^{m+1}F_{k,2(r+1)n-m} 
          \end{bmatrix}
        }.
        \end{align*}
Hence proof.
\end{proof}
\noindent The characteristic equation of $S_{k}(n,m)$ is
	\begin{align*}
	&\lambda^2-\Delta F_{k,n}F_{k,m}\lambda-\Delta (-1)^nF_{k,n}^2=0\quad\text{and by Cauchy-Hamilton theorem}\\
		&S_{k}(n,m)^2-\Delta F_{k,n}F_{k,m}S_{k}(n,m)-\Delta (-1)^nF_{k,n}^2I=0.\\
&\text{Manipulating above equation gives}\\
		&\Delta F_{k,m}(F_{k,n}S_{k}(n,m)+ (-1)^nF_{k,m}I)=S_{k}(n,m)^2,\\
		&	(2S_{k}(n,m)-\Delta F_{k,n}F_{k,m}I)=\Delta F_{k,m}^2L_{k,n}^2.\\
	&\therefore\quad	\Delta^rF_{k,m}^r(F_{k,n}S_{k}(n,m)+(-1)^nF_{k,m}I)^r=S_{k}(n,m)^{2r},\\
	&(2S_{k}(n,m)-\Delta F_{k,n}F_{k,m}I)^{2r}=\Delta^r F_{k,m}^{2r}L_{k,n}^{2r}I,\\	
&(2S_{k}(n,m)-\Delta F_{k,n}F_{k,m}I)^{2r+1}=\Delta^r F_{k,m}^{2r}L_{k,n}^{2r}(2S_{k}(n,m)-\Delta F_{k,n}F_{k,m}I).
\end{align*}
Now expanding previous four equations and equating upper left entries of the relevant matrices gives respectively to
\begin{align*}
&\sum_{i=0,i-even}^{i=r}\left(^{r} _{i}\right)(-1)^{n(r-i)+1}\Delta^{\frac{i-1}{2}}F_{k,n}^{i}L_{k,in+m}+\sum_{i=0,i-odd}^{i=r}\left(^{r}_{i}\right)(-1)^{n(r-i)}\Delta^{\frac{i}{2}}F_{k,n}^{i}F_{k,in+m}\\
&=F_{k,2nr+m}\sum_{i=0,i-even}^{i=2r}\left(^{2r} _{i}\right)2^i\Delta^{\frac{2r-i}{2}}F_{k,n}^{2r-i}F_{k,in+m}\\&-\sum_{i=0,i-odd}^{i=2r-1}\left(^{2r}_{i}\right)2^i\Delta^{\frac{2r-1-i}{2}}F_{k,n}^{2r-i}L_{k,in+m}\\
&=L_{k,n}^{2r}F_{k,m}\sum_{i=0,i-odd}^{i=2r+1}\left(^{2r+1} _{i}\right)2^i\Delta^{\frac{2r+3-i}{2}}F_{k,n}^{2r+1-i}F_{k,in+m}\\&-\sum_{i=0,i-even}^{i=2r-1}\left(^{2r+1}_{i}\right)2^i\Delta^{\frac{2r+2-i}{2}}F_{k,n}^{2r+1-i}L_{k,in+m}\\
&=\Delta L_{k,n}^{2r+1}F_{k,m}.
\end{align*}
\begin{theorem} Let $S_{k}(n,m)$ be a matrix as in (\ref{sknm}) then for all integer $r$ 
\begin{align*}
S_{k}(n,m)^{2r} = F_{k,m}^{2r-1}\Delta^r{\left[
          \begin{array}{cc}
           \bigskip
            F_{k,2rn+m} & (-1)^{m+1}F_{k,2rn} \\
             \bigskip
            F_{k,2rn} & (-1)^{m+1}F_{k,2rn-m} \\
          \end{array}
        \right]}\\
  S_{k}(n,m)^{2r-1} = F_{k,m}^{2r-2}\Delta^{r-1}{\left[
          \begin{array}{cc}
           \bigskip
            L_{k,2(r-1)n+m} & (-1)^{m+1}L_{k,2(r-1)n} \\
             \bigskip
            L_{k,2(r-1)n} & (-1)^{m+1}L_{k,2(r-1)n-m} \\
          \end{array}
        \right]}
\end{align*}
\end{theorem}
\begin{theorem}
For $n, m, r \geq 1 $, we have
\begin{align*}
&\sum_{i=0,i-even}^{i=r}\left(^{r} _{i}\right)(-1)^{n(r-i)+1}\Delta^{\frac{i-1}{2}}F_{k,n}^{i}L_{k,in+m}+\sum_{i=0,i-odd}^{i=r}\left(^{r}_{i}\right)(-1)^{n(r-i)}\Delta^{\frac{i}{2}}F_{k,n}^{i}F_{k,in+m}\\&= F_{k,2nr+m},\\	
&\sum_{i=0,i-even}^{i=2r}\left(^{2r} _{i}\right)2^i\Delta^{\frac{2r-i}{2}}F_{k,n}^{2r-i}F_{k,in+m}-\sum_{i=0,i-odd}^{i=2r-1}\left(^{2r}_{i}\right)2^i\Delta^{\frac{2r-1-i}{2}}F_{k,n}^{2r-i}L_{k,in+m}\\&= L_{k,n}^{2r}F_{k,m},\\
&\sum_{i=0,i-odd}^{i=2r+1}\left(^{2r+1} _{i}\right)2^i\Delta^{\frac{2r+3-i}{2}}F_{k,n}^{2r+1-i}F_{k,in+m}\\&-\sum_{i=0,i-even}^{i=2r-1}\left(^{2r+1}_{i}\right)2^i\Delta^{\frac{2r+2-i}{2}}F_{k,n}^{2r+1-i}L_{k,in+m} = \Delta L_{k,n}^{2r+1}F_{k,m}.	
\end{align*}	
\end{theorem}
\begin{definition}\label{a1}
Define $3 \times 3$ matrix $A$ as 
\begin{align*}
A  = {\left[
          \begin{array}{ccc}
           \bigskip
            k^2+1 & k^2+1& -1 \\
             \bigskip
            1 & 0 & 0 \\
             \bigskip
            0 & 1 &0\\
          \end{array}
        \right]}.
\end{align*}
\end{definition}
\begin{definition}\label{a2}
Define $3 \times 3$ matrix $ A_{n}$ as 
\begin{align*}
 A_{n}  = {\left[
          \begin{array}{ccc}
           \bigskip
            \sum_{i=0}^{i=n+1}F_{k,i}^2 & F_{k,n}F_{k,n+2}& - \sum_{i=0}^{i=n}F_{k,i}^2 \\
            \bigskip
            \sum_{i=0}^{i=n}F_{k,i}^2 & F_{k,n-1} F_{k,n+1} & - \sum_{i=0}^{i=n-1}F_{k,i}^2 \\
             \bigskip
            \sum_{i=0}^{i=n-1}F_{k,i}^2 & F_{k,n-2}F_{k,n} &- \sum_{i=0}^{i=n-2}F_{k,i}^2
              \end{array}
        \right]}.
\end{align*}
\end{definition}
where  $n,k$ are integers.
\begin{theorem} Let the matrices $A$ and $A_{n}$ have the form (\ref{a1}) and (\ref{a2}), respectively. Then for all integer $n > 1$, we have
\begin{align*}
  A^n = A_{n}.
\end{align*}
\end{theorem}
\begin{proof}
Using principal of Mathematical induction on $n$. If $n = 2$, then
\begin{align*}
A^2 &= {\left[
          \begin{array}{ccc}
           \bigskip
            k^4 + 3k^2+1 & k^2(k^2+2)& -k^2 -1 \\
             \bigskip
            k^2 + 1 & k^2 + 1 & -1 \\
             \bigskip
            1 & 0 &0\\
          \end{array}
        \right]}\\
        \bigskip
        &= {\left[
          \begin{array}{ccc}
           \bigskip
            \sum_{i=0}^{i=3}F_{k,i}^2 & F_{k,2}F_{k,4}& - \sum_{i=0}^{i=2}F_{k,i}^2 \\
            \bigskip
            \sum_{i=0}^{i=2}F_{k,i}^2 & F_{k,1} F_{k,3} & - \sum_{i=0}^{i=1}F_{k,i}^2 \\
             \bigskip
            \sum_{i=0}^{i= 1}F_{k,i}^2 & F_{k,0}F_{k,2} &- \sum_{i=0}^{i=0}F_{k,i}^2\\
          \end{array}
        \right]}\\
        \bigskip
        &=A_{2}.                                                                
\end{align*}
Hence result is true for $n = 2$, now suppose that the result is true for $n, (n > 2)$.
\begin{align*}
A^n = A_{n}.
\end{align*}
Now consider
\begin{align*}
A^{n+1} & =  A^n A\\
& =  {\left[
          \begin{array}{ccc}
           \bigskip
            \sum_{i=0}^{i=n+1}F_{k,i}^2 & F_{k,n}F_{k,n+2}& - \sum_{i=0}^{i=n}F_{k,i}^2 \\
            \bigskip
            \sum_{i=0}^{i=n}F_{k,i}^2 & F_{k,n-1} F_{k,n+1} & - \sum_{i=0}^{i=n-1}F_{k,i}^2 \\
             \bigskip
            \sum_{i=0}^{i=n-1}F_{k,i}^2 & F_{k,n-2}F_{k,n} &- \sum_{i=0}^{i=n-2}F_{k,i}^2\\
          \end{array}
        \right]} {\left[
          \begin{array}{ccc}
           \bigskip
            k^2+1 & k^2+1& -1 \\
             \bigskip
            1 & 0 & 0 \\
             \bigskip
            0 & 1 &0\\
          \end{array}
        \right]}\\
         \bigskip
        & = {\left[
          \begin{array}{ccc}
           \bigskip
            (k^2+1)\sum_{i=0}^{i=n+1}F_{k,i}^2+F_{k,n}F_{k,n+2} & (k^+1)\sum_{i=0}^{i=n+1}F_{k,i}^2 - \sum_{i=0}^{i=n}F_{k,i}^2& - \sum_{i=0}^{i=n+1}F_{k,i}^2 \\
            \bigskip
            (k^2+1)\sum_{i=0}^{i=n}F_{k,i}^2+F_{k,n -1}F_{k,n+1} & (k^+1)\sum_{i=0}^{i=n}F_{k,i}^2 - \sum_{i=0}^{i=n-1}F_{k,i}^2& - \sum_{i=0}^{i=n}F_{k,i}^2 \\
             \bigskip
            (k^2+1)\sum_{i=0}^{i=n-1}F_{k,i}^2+F_{k,n-2}F_{k,n} & (k^+1)\sum_{i=0}^{i=n-1}F_{k,i}^2 - \sum_{i=0}^{i=n-2}F_{k,i}^2& - \sum_{i=0}^{i=n-1}F_{k,i}^2 \\
          \end{array}
        \right]}.
\end{align*}
Using 
\begin{align*}
&(k^2 +1) F_{k,n} F_{k, n+1} + k F_{k, n-1} F_{k, n+1}= k \sum_{i=0}^{i=n+1}F_{k,i}^2, \\
  &(k^2 +1) F_{k,n} F_{k, n+1} -  F_{k, n-1} F_{k, n}=k F_{k,n} F_{k, n+2}.
\end{align*}
\begin{align*}
A^{n+1} & =   {\left[
          \begin{array}{ccc}
           \bigskip
            \sum_{i=0}^{i=n+2}F_{k,i}^2 & F_{k,n+1}F_{k,n+3}& - \sum_{i=0}^{i=n+1}F_{k,i}^2 \\
            \bigskip
            \sum_{i=0}^{i=n+1}F_{k,i}^2 & F_{k,n} F_{k,n+2} & - \sum_{i=0}^{i=n}F_{k,i}^2 \\
             \bigskip
            \sum_{i=0}^{i=n}F_{k,i}^2 & F_{k,n-1}F_{k,n+1} &- \sum_{i=0}^{i=n-1}F_{k,i}^2\\
          \end{array}
        \right]}\\
        & =  A_{n+1}.
        \end{align*}
\end{proof}
Let the matrix $A$ have the form (\ref{a1}) then
\begin{align*}
A^n = {\left[
          \begin{array}{ccc}
           \bigskip
            \dfrac{F_{k,n+1} F_{k, n+2}}{k} & F_{k,n}F_{k, n+2}& - \dfrac{F_{k,n} F_{k, n+1}}{k} \\
             \bigskip
           \dfrac{F_{k,n} F_{k, n+1}}{k} & F_{k,n-1}F_{k, n+1}& - \dfrac{F_{k,n-1} F_{k, n}}{k} \\
             \bigskip
           \dfrac{F_{k,n-1} F_{k, n}}{k} & F_{k,n-2}F_{k, n}& - \dfrac{F_{k,n-2} F_{k, n-1}}{k} \\
          \end{array}
        \right]}.
\end{align*}
With $det (A)= -1$. The characteristic equation of the matrix $A$ is  
\begin{align*}
\lambda^3 + (-k^2-1)\lambda^2+(-k^2-1)\lambda +1 = 0.
\end{align*}
Thus the eigenvalues of the matrix $A$ are 
\begin{align*}
\lambda_1 & = & \dfrac{1}{2} k^2 + \dfrac{1}{2} k\sqrt{\Delta}+1,\\
\lambda_2 & = & \dfrac{11}{2} k^2 - \dfrac{1}{2} k\sqrt{\Delta}+1,\\
\lambda_3& = & -1.
\end{align*}
We can easily obtain
\begin{align*}
\lambda_1 & = & r_1^2,\\
\lambda_2 & = & r_2^2.
\end{align*}
\begin{definition}
Define the $3\times 3$ matrix $B$ as follows
\begin{align*}
B = {\left[
          \begin{array}{ccc}
           \bigskip
            \lambda_1^2 & \lambda_2^2& \lambda_3^2 \\
             \bigskip
            \lambda_1 & \lambda_2 & \lambda_3 \\
             \bigskip
            1 & 1 &1\\
          \end{array}
        \right]},
\end{align*}
\end{definition}
\begin{align*}
det B = k(\Delta)^{\frac{3}{2}}.
\end{align*}
Let
\begin{align*}
C = B^T = {\left[
          \begin{array}{ccc}
           \bigskip
            \lambda_1^2 & \lambda_1& 1 \\
             \bigskip
            \lambda_2^2 & \lambda_2 & 1 \\
             \bigskip
            \lambda_3^2 & \lambda_3 &1\\
          \end{array}
        \right]}
\end{align*}
and
\begin{align*}
D_i = {\left[
          \begin{array}{c}
           \bigskip
            \lambda_1^{n-i+3}  \\
             \bigskip
            \lambda_2^{n-i+3} \\
             \bigskip
            \lambda_3^{n-i+3} \\
          \end{array}
        \right]}.
\end{align*}
\begin{theorem}
Let the matrix $A_n = (a_{i,j})$ have the form (\ref{a2}). Then for all $i, j$ such that $1 \leq i, j \leq 3$, we have
\begin{align*}
a_{i,j} = \dfrac{det(C_j^{(i)})}{det (C)}.
\end{align*}
\end{theorem}
\begin{definition}\label{a5}
Define $4 \times 4$ matrix $E$ as follows
\begin{align*}
E = {\left[
          \begin{array}{cccc}
           \bigskip
            1 & 0&0 & 0 \\
             \bigskip
            1 & k^2 + 1 & k^2 + 1 & -1 \\
             \bigskip
            0 & 1 & 0 &0\\
            \bigskip
            0 & 0 & 1 & 0\\
          \end{array}
        \right]}.
\end{align*}
where  $n$, $k$ are integers.
\end{definition}
\begin{theorem}
Let the matrix $E$ as in (\ref{a5}), then for $n > 2$
\begin{align*}
E^n = {\left[
          \begin{array}{cccc}
           \bigskip
            1 & 0&0 & 0 \\
             \bigskip
            \dfrac{1}{k}\sum_{i=0}^{i=n}F_{k,i}F_{k,i+1} & \dfrac{F_{k,n+1}F_{k,n+2}}{k}  & F_{k,n}F_{k,n+2} & -\dfrac{F_{k,n}F_{k,n+1}}{k} \\
             \bigskip
            \dfrac{1}{k}\sum_{i=0}^{i=n-1}F_{k,i}F_{k,i+1} & \dfrac{F_{k,n}F_{k,n+1}}{k} & F_{k,n-1}F_{k,n+1} &-\dfrac{F_{k,n-1}F_{k,n}}{k}\\
            \bigskip
            \dfrac{1}{k}\sum_{i=0}^{i=n-2}F_{k,i}F_{k,i+1} & \dfrac{F_{k,n-1}F_{k,n}}{k}  & F_{k,n-2}F_{k,n} & -\dfrac{F_{k,n-2}F_{k,n-1}}{k}\\
          \end{array}
        \right]}.
\end{align*}
\end{theorem}
\begin{proof}
Theorem can easily proved by using Mathematical induction on $n$.
\end{proof}
\begin{theorem}
For $n > 0$, we have 
\begin{equation}
\sum_{i=0}^{i=n}F_{k,i}F_{k,i+1} = \dfrac{F_{k,n+1}^2+ F_{k,n}F_{k,n+2}-1}{2 k^2}.
\end{equation}
\end{theorem}
\begin{definition}
Define $4 \times 4$ matrix 
\begin{align*}
Y_n = {\left[
          \begin{array}{cccc}
           \bigskip
            F_{k,n}^2 & -F_{k,n}F_{k,n+3} & F_{k,n+3}^2& F_{k,n}F_{k,n+3} \\
             \bigskip
            -F_{k,n}F_{k,n+3}  & F_{k,n+3}^2 & F_{k,n+3}F_{k,n} & F_{k,n}^2 \\
             \bigskip
            F_{k,n+3}^2 & F_{k,n}F_{k,n+3} & F_{k,n}^2& -F_{k,n}F_{k,n+3} \\
            \bigskip
            F_{k,n}F_{k,n+3}  & F_{k,n}^2 & - F_{k,n} F_{k,n+3}& F_{k,n+3}^2 \\
       \end{array}
        \right]}
\end{align*}
Where $n, k$ are integers.
\end{definition}
\begin{theorem}
For any positive integer $n$, we have
\begin{align*}
det (Y_n) = -\left[ F_{k,n}^2 + F_{k,n+3}^2\right]^4.
\end{align*}
\end{theorem}
\begin{proof}
Let
\begin{align*}
x= F_{k, n},\\
y = F_{k, n+3}.
\end{align*}
It gives that 
\begin{align*}
det (Y_n) = \begin{vmatrix}
               \bigskip
           x^2 & -xy & y^2& xy \\
             \bigskip
            -xy  & y^2 & xy & x^2 \\
             \bigskip
            y^2 & xy & x^2& -xy \\
            \bigskip
            xy  & x^2 & - xy& y^2 \\
       \end{vmatrix}.
\end{align*}
By interchanging $C_2$ and $C_4$, $C_3$ and $C_4$, $R_3$ and $R_4$, changing the sign of $C_3$ and $R_4$, we get 
\begin{align*}
det (Y_n)= - \begin{vmatrix}
           \bigskip
           x^2 & xy & xy& y^2 \\
             \bigskip
            -xy  & x^2 & -y^2 & xy \\
             \bigskip
            xy &y^2 & -x^2& -xy \\
            \bigskip
            -y^2  & xy & xy & -x^2 \\
      \end{vmatrix}\\
    = - \left( x^2 + y^2\right)^4 .   
 \end{align*}
Finally theorem follows
\begin{align*}
det (Y_n) = -\left[ F_{k,n}^2 + F_{k,n+3}^2\right]^4.
\end{align*}
\end{proof}
\begin{definition}\label{a6}
Define $5 \times 5$ matrix 
\begin{align*}
\begin{footnotesize}
W_n = \begin{bmatrix}
           \bigskip
            F_{k,n}F_{k,n+2}+k^2 F_{k,n+1}^2 & F_{k,n}^2& k^2 F_{k,n+1}^2& F_{k,n+2}^2 & -(F_{k,n}F_{k,n+2}\\&&&&+k^2 F_{k,n+1}^2) \\
             \bigskip
            F_{k,n}F_{k,n+2}+k^2 F_{k,n+1}^2  & F_{k,n}(k F_{k,n+1}+F_{k,n+2} & -k F_{k,n+1}L_{k,n+1} & (k F_{k,n+1}\\&&&-F_{k,n}) F_{k,n+2}& F_{k,n}F_{k,n+2}\\&&&&+k^2 F_{k,n+1}^2 \\
             \bigskip
            0 & 2k F_{k,n+1}F_{k,n+2}& 2 F_{k,n}F_{k,n+2} & -2k F_{k,n}F_{k,n+1}& 0 \\
            \bigskip
            F_{k,n}F_{k,n+2}+k^2 F_{k,n+1}^2  & - F_{k,n}(k F_{k,n+1}+F_{k,n+2}) &k F_{k,n+1}L_{k,n+1}& - F_{k,n+2} (kF_{k,n+1}\\&&&- F_{k,n})& F_{k,n}F_{k,n+2}\\&&&&+k^2F_{k,n+1}^2 \\
     \bigskip
            -(F_{k,n}F_{k,n+2}+k^2 F_{k,n+1}^2)  &  F_{k,n}^2 & k^2 F_{k,n+1}^2 &  F_{k,n+2}^2& F_{k,n}F_{k,n+2}\\&&&&+k^2F_{k,n+1}^2        
\end{bmatrix}.
\end{footnotesize}
\end{align*}
where  $n, k$ are integers.
\end{definition}
\begin{theorem}
Let $W_n$ be a $5 \times 5$ matrix as in (\ref{a6}), then
\begin{equation}
det(W_n) = -32 \left[ k^2 F_{k, n+1}^2 + F_{k, n}(k F_{k, n+1}+F_{k, n})\right] ^5.
\end{equation}
\end{theorem}
\begin{proof}
Put
\begin{align*}
F_{k,n} = x,\\
k F_{k,n+1} = y,\\
F_{k,n+2} = z.
\end{align*}
Therefore
\begin{align*}
det(W_n) =\begin{vmatrix}
           \bigskip
            xz+y^2 & x^2& y^2& z^2 & -(xz+k^2 y^2) \\
             \bigskip
            xz+y^2  & x(y+z) & -y(x+z) & (y-x) z& xz+y^2 \\
             \bigskip
            0 & 2yz& 2 xz & -2xy & 0 \\
            \bigskip
            xz+y^2  & - x(y+z) &y(x+z)& - z (y- x)& xz+y^2 \\
     \bigskip
            -(xz+y^2)  &  x^2 & y^2 &  z^2& xz+y^2\\        
\end{vmatrix}\\
= (xz+y^2)^2 \begin{vmatrix}
           \bigskip
            1 & x^2& y^2& z^2 & -1 \\
             \bigskip
            1  & x(y+z) & -y(x+z) & (y-x) z& 1 \\
             \bigskip
            0 & 2yz& 2 xz & -2xy & 0 \\
            \bigskip
            1  & - x(y+z) &y(x+z)& - z (y- x)& 1 \\
     \bigskip
            -1  &  x^2 & y^2 &  z^2& 1\\        
\end{vmatrix}.
\end{align*}
Making the row-column transformations $(C_5+C_1 \rightarrow C_1)$, $(R_2+R_4\rightarrow R_4)$ and $(- R_1 +R_5 \rightarrow R_5)$ yields
\begin{align*}
det(W_n) = \left( xz+y^2\right)^2 \begin{vmatrix}
           \bigskip
            0 & x^2& y^2& z^2 & -1 \\
             \bigskip
            2  & x(y+z) & -y(x+z) & (y-x) z& 1 \\
             \bigskip
            0 & 2yz& 2 xz & -2xy & 0 \\
            \bigskip
            4  & 0 &0& 0& -2 \\
     \bigskip
            0  &  0 & 0 &  0& 2\\        
\end{vmatrix}\\
= -16(xz+y^2)^2 \begin{vmatrix}
           \bigskip
             x^2& y^2& z^2  \\
             \bigskip
             x(y+z) & -y(x+z) & (y-x) z\\
             \bigskip
             yz&  xz & -xy  \\
\end{vmatrix}.
\end{align*}
Using $z = y+x $, we have
\begin{align*}
det (W_n)= -16(xz+y^2)^2 \begin{vmatrix}
           \bigskip
             x^2& y^2& (x+y)^2  \\
             \bigskip
             x^2+2xy & -y^2-2xy & (y^2-x^2) z\\
             \bigskip
             y^2+xy&  x^2+xy & -xy \\
\end{vmatrix}.
\end{align*}
Making the row transformation $(R_1 + R_3 \rightarrow R_3)$, we obtain
\begin{align*}
det (W_n) = -16(x^2+y^2+xy)^3 \begin{vmatrix}
           \bigskip
             x^2& y^2& (x+y)^2  \\
             \bigskip
             x^2+2xy & -y^2-2xy & (y^2-x^2) z\\
             \bigskip
             1&  1 & 1 \\
\end{vmatrix}\\
= -32 (x^2+y^2 +xy)^5.
\end{align*}
Finally theorem follows
\begin{align*}
det(W_n) = -32 \left[ k^2 F_{k, n+1}^2 + F_{k, n}(k F_{k, n+1}+F_{k, n})\right] ^5.
\end{align*}
\end{proof}
\begin{definition}
Define $n \times n$  super-diagonal matrix $G_n$ as 
\begin{align*}
G_n = {\left[
          \begin{array}{ccccccc}
           \bigskip
            k^2+1 & k^2+1  & -1& 0 &0&..........&0 \\
             \bigskip
           -1 & k^2+1  & k^2+1& -1 &0&..........&0 \\
             \bigskip
           0 & -1  & k^2+1& k^2+1 &-1&..........&0 \\
           \bigskip
            .......... & .......... & ..........& .......... &..........&..........&.......... \\
           \bigskip
            0 & ..........&..........  & 0& -1 &k^2+1&k^2+1\\
            \bigskip
            0 & ..........&..........  & 0& 0 &-1&k^2+1\\
       \end{array}
        \right]_{n \times n}}.
\end{align*}
where  $n, k$ are integers.
\end{definition}
\begin{theorem}\label{a7}
For $n > 1$, we have
\begin{align*}
det(G_n) = \sum_{i=0}^{i=n+1}F_{k,i}^2.
\end{align*}
where $$ det (G_1) = \sum_{i=0}^{i=2}F_{k,i}^2 = k^2+1$$
\end{theorem}
\begin{proof}
Using the Laplace expansion of a determinant gives
\begin{align*}
det(G_n) &= (k^2+1)det(G_n-1) + (k^2+1)det(G_n-2)- det(G_n-3),\\ 
det(G_1)& = \sum_{i=0}^{i=2}F_{k,i}^2,\\
det(G_2) &= \sum_{i=0}^{i=3}F_{k,i}^2,\\
det(G_3) &= \sum_{i=0}^{i=4}F_{k,i}^2.
\end{align*}
These equations gives
\begin{align*}
&(k^2+1)\sum_{i=0}^{i=n+1}F_{k,i}^2 +(k^2+1)\sum_{i=0}^{i=n}F_{k,i}^2 - \sum_{i=0}^{i=n-1}F_{k,i}^2  = \sum_{i=0}^{i=n+2}F_{k,i}^2,\\
&det(G_n) = \sum_{i=0}^{i=n+1}F_{k,i}^2.
\end{align*}
\end{proof}
\begin{definition}
Define $n \times n$ matrix $H_n$ as 
\begin{align*}
H_n = {\left[
          \begin{array}{ccccccc}
           \bigskip
            k^2+1 & k^2+1  & -1& 0 &0&..........&0 \\
             \bigskip
           -1 & k^2+1  & k^2+1& -1 &0&..........&0 \\
             \bigskip
           0 & -1  & k^2+1& k^2+1 &-1&..........&0 \\
           \bigskip
            .......... & .......... & ..........& .......... &..........&..........&.......... \\
           \bigskip
            0 & ..........&..........  & 0& -1 &k^2+1&k^2+1\\
            \bigskip
            0 & ..........&..........  & 0& 0 &-1&0\\
       \end{array}
        \right]_{n \times n}}.
\end{align*}
where  $n, k$ are integers.
\end{definition}
The matrix $ H_n $ is obtained from matrix $ G_n$ by deleting $(n, n)^{th}$ entry.
\begin{theorem}
For $n > 1 $
\begin{align*}
det (H_n) =  F_{k, n-1} F_{k, n+1}
\end{align*}
Where $ det (H_n) = (k^2+1) $.
\end{theorem}
\begin{proof}
Similar to theorem (\ref{a7}), we obtain
\begin{align*}
det(H_n) = (k^2+1)det(H_n-1) + (k^2+1)det(H_n-2)- det(H_n-3). 
\end{align*}
We have
\begin{align*}
det(H_2) = F_{k, 1} F_{k, 3},\\
det(H_3) = F_{k, 2} F_{k, 4},\\
det(H_4) = F_{k, 3} F_{k, 5}.
\end{align*}
It gives that
\begin{align*}
&(k^2+1)F_{k,n+1} F_{k, n+3}+(k^2+1)F_{k,n} F_{k, n+2}-F_{k,n-1} F_{k, n+1} =  F_{k,n+2} F_{k, n+4},\\
&det (H_n) =  F_{k, n-1} F_{k, n+1}.
\end{align*}
\end{proof}
\subsection{Identities for $k$-Fibonacci and $k$-Lucas Sequences using determinants of some matrices}
In this section, some determinantal techniques are used to obtain many $k$-Lucas identities.
\begin{theorem}
If  $n, i, j, t, m$ are positive integers with $0 < t < i$, $i+1 < m$, $j = 1$, then 
\begin{align*}
&det {\left[
          \begin{array}{ccc}
           \bigskip
            L_{k, n+t}^2+4 L_{k, n-i}^2 & L_{k, n+i+j} & L_{k, n+i+j} \\
             \bigskip
            L_{k, n+t}& 4 L_{k, n+i}^2+L_{k, n+i+j}^2 & L_{k, n+t} \\
     \bigskip
            L_{k, n+i}& 2 L_{k, n+i} & \dfrac{L_{k, n+i+j}^2 + L_{k, n+t}^2}{2L_{k, n+i}} \\       
          \end{array}
        \right]}\\& = 8 L_{k, n+i} L_{k, n+t} L_{k, n+i+j}.
\end{align*}
\end{theorem}
\begin{proof}
Let
\begin{align*}
\aleph_1 = det {\left[
          \begin{array}{ccc}
           \bigskip
            L_{k, n+t}^2+4 L_{k, n-i}^2 & L_{k, n+i+j} & L_{k, n+i+j} \\
             \bigskip
            L_{k, n+t}& 4 L_{k, n+i}^2+L_{k, n+i+j}^2 & L_{k, n+t} \\
     \bigskip
            L_{k, n+i}& 2 L_{k, n+i} & \dfrac{L_{k, n+i+j}^2 + L_{k, n+t}^2}{2L_{k, n+i}} \\       
          \end{array}
        \right]}. 
\end{align*}
Assume that 
\begin{align*}
L_{k, n+t} = \phi,\\
L_{k, n+i} = \varphi.
\end{align*}
Then, we have 
\begin{align*}
L_{k, n+i+j} = k \varphi+\phi.
\end{align*}
Now
\begin{align*}
\aleph_1 = det {\left[
          \begin{array}{ccc}
           \bigskip
            \dfrac{\phi^2+\varphi^2}{k \varphi+\phi} & k \varphi+\phi & k \varphi+\phi \\
             \bigskip
            \phi & \frac{\varphi^2 +(k \varphi+\phi)^2}{\phi} & \phi\\
     \bigskip
            \varphi & \varphi & \dfrac{\phi^2 + (k \varphi+\phi)^2}{\varphi} \\       
          \end{array}
        \right]}. 
\end{align*}
Making the row operations $\dfrac{1}{(k \varphi+\phi)}\left[ (k \varphi+\phi)R_1\right]$, $\dfrac{1}{\phi}\left[ \phi R_2\right] $, $\dfrac{1}{\varphi}\left[ \varphi R_3\right] $, we get
\begin{align*}
\aleph_1 = \dfrac{1}{\phi \varphi (k \varphi+\phi)}  det {\left[
          \begin{array}{ccc}
           \bigskip
            \phi^2+\varphi^2 &(k \varphi+\phi)^2 & (k \varphi+\phi)^2 \\
             \bigskip
            \phi^2 & \varphi^2 +(k \varphi+\phi)^2 & \phi^2\\
     \bigskip
            \varphi^2 & \varphi^2 & \phi^2 + (k \varphi+\phi)^2 \\       
          \end{array}
        \right]} .
\end{align*}
Again making the row operations 
$R_1 + R_2 + R_3 \rightarrow  R_1$, $R_3 - R_1 \rightarrow R_3$ and $R_2  - R_1 \rightarrow R_2$, gives that
\begin{align*}
\aleph_1 = \dfrac{1}{\phi \varphi (k \varphi+\phi)}  det {\left[
          \begin{array}{ccc}
           \bigskip
            \phi^2+\varphi^2 &\varphi^2+(k \varphi+\phi)^2 & \phi^{2} +(k \varphi+\phi)^2 \\
             \bigskip
            - \varphi^2& 0  & -(k \varphi+\phi)^2\\
     \bigskip
            \phi^2 & -(k \varphi+\phi)^2 & 0\\       
          \end{array}
        \right]}. 
\end{align*}
Expanding these, we get
\begin{align*}
\aleph_1 = 8 \phi\varphi(k \varphi+\phi). 
\end{align*}
Putting
\begin{align*}
L_{k, n+t} = \phi,
L_{k, n+i} = \varphi,
L_{k, n+i+j} = k \varphi+\phi.
\end{align*}
It gives that
\begin{align*}
\aleph_1 = 8 L_{k, n+i} L_{k, n+t} L_{k, n+i+j}.
\end{align*}
\end{proof}
\begin{theorem}
If  $n, i, j, t, m$ are positive integers with $0 < t < i$, $i+1 < m$, $j = 1$, then
\begin{align*}
&\begin{vmatrix}
            L_{k, n+t}^2 & 2L_{k, n+i}L_{k, n+i+j} & L_{k, n+t}L_{k, n+i+j}+ L_{k, n+i+j} \\
             \bigskip
            L_{k, n+t}^2+2 L_{k, n+i}L_{k, n+t}& 4 L_{k, n+i}^2 & L_{k, n+t}L_{k, n+i+j} \\
     \bigskip
            2 L_{k, n+i} L_{k, n+t}& 4 L_{k, n+i}^2+2 L_{k, n+i}L_{k, n+i+j} & L_{k, n+i+j}^2\end{vmatrix}
            \\& = \left[ 4 L_{k, n+i}L_{k, n+i+j}\right]^2.
\end{align*}
\end{theorem}
\begin{proof}
Let
\begin{align*}
&\aleph\\& = \begin{vmatrix}
L_{k, n+t}^2 & 2L_{k, n+i}L_{k, n+i+j} & L_{k, n+t}L_{k, n+i+j}+ L_{k, n+i+j} \\
\bigskip
L_{k, n+t}^2+2 L_{k, n+i}L_{k, n+t}& 4 L_{k, n+i}^2 & L_{k, n+t}L_{k, n+i+j} \\
\bigskip
2 L_{k, n+i} L_{k, n+t}& 4 L_{k, n+i}^2+2 L_{k, n+i}L_{k, n+i+j} & L_{k, n+i+j}^2
\end{vmatrix}. 
\end{align*}
Assume that 
\begin{align*}
L_{k, n+t} = \phi,
L_{k, n+i} = \varphi.
\end{align*}
Then we have
\begin{align*}
\aleph_2 =\begin{vmatrix}
            \phi^2 & \varphi(k \varphi+\phi) &\phi(k \varphi+\phi)+(k \varphi+\phi)^2 \\
             \bigskip
            \phi^2+ \phi\varphi & \varphi^2 & \phi(k \varphi+\phi)\\
     \bigskip
            \phi\varphi & \varphi^2+\varphi(k \varphi+\phi) & (k \varphi+\phi)^2
                    \end{vmatrix}. 
\end{align*}
Making the row operations $R_2 \rightarrow  R_2 - (R_1+R_3))$, we get
\begin{align*}
\aleph_1 = \dfrac{1}{\phi \varphi (k \varphi+\phi)}  \begin{vmatrix}
            \phi &(k \varphi+\phi) & \phi+(k \varphi+\phi) \\
             \bigskip
            0 & -2(k \varphi+\phi) & -2(k \varphi+\phi)\\
     \bigskip
            \varphi& \varphi+(k \varphi+\phi) & (k \varphi+\phi) \end{vmatrix}.
\end{align*}
Making the Column operations 
$C_2 \rightarrow  C_2 - C_3$ and expanding it gives
\begin{align*}
\aleph_2 = 4 \left[ 2\phi\varphi(k \varphi+\phi)\right]^2.
\end{align*}
Putting
\begin{align*}
L_{k, n+t} = \phi,
L_{k, n+i} = \varphi,
L_{k, n+i+j} = k \varphi+\phi.
\end{align*}
It gives that
\begin{align*}
\aleph_2 = \left[ 4 L_{k, n+i}L_{k, n+i+j}\right]^2.
\end{align*}
\end{proof}
\begin{corollary}
If  $n, i, j, t, m$ are positive integers with $0 < t < i$, $i+1 < m$, $j = 1$, then
\begin{align*}
&det {\left[
          \begin{array}{ccc}
           \bigskip
            -L_{k, n+t}^2 & 2L_{k, n+i}L_{k, n+t} & L_{k, n+t}L_{k, n+i+j} \\
             \bigskip
            2L_{k, n+i}L_{k, n+t} & -  4 L_{k, n+i}^2& 2 L_{k, n+i}L_{k, n+i+j} \\
     \bigskip
             L_{k, n+t} L_{k, n+i+j}& 2 L_{k, n+i}L_{k, n+i+j} & - L_{k, n+i+j}^2\\       
          \end{array}
        \right]} \\&= \left[ 4 L_{k, n+i}L_{k, n+t}L_{k, n+i+j}\right]^2.
\end{align*}
\end{corollary}
\begin{corollary}
If  $n, i, j, t, m$ are positive integers with $0 < t < i$, $i+1 < m$, $j = 1$, then
\begin{align*}
&det {\left[
          \begin{array}{ccc}
           \bigskip
            4L_{k, n+i}^2+L_{k, n+i+j}^2 & 2L_{k, n+i}L_{k, n+t} & L_{k, n+t}L_{k, n+i+j} \\
             \bigskip
            2L_{k, n+i}L_{k, n+t} &  L_{k, n+t}^2& 2 L_{k, n+i}L_{k, n+i+j} \\
     \bigskip
             L_{k, n+t} L_{k, n+i+j}& 2 L_{k, n+i}L_{k, n+i+j} & 4 L_{k, n+i}^2+L_{k, n+t}^2\\       
          \end{array}
        \right]}\\& = \left[ 4 L_{k, n+i}L_{k, n+t}L_{k, n+i+j}\right]^2.
\end{align*}
\end{corollary}
\begin{corollary}
If  $n, i, j, t, m$ are positive integers with $0 < t < i$, $i+1 < m$, $j = 1$, then
\begin{align*}
&\begin{vmatrix}
            (2L_{k, n+i+j}+2L_{k, n+i}\\+L_{k, n+t})& L_{k, n+t} & 2L_{k, n+i} \\
             \bigskip
            L_{k, n+i+j}&  2L_{k, n+t}+2 L_{k, n+i}+L_{k, n+i+j}& 2 L_{k, n+i}\\
     \bigskip
             L_{k, n+i+j}& 2 L_{k, n+t} & 4 L_{k, n+i}+L_{k, n+t}+L_{k, n+i+j}\\       
  \end{vmatrix}        \\& =  2\left[ 2 L_{k, n+i}+L_{k, n+t}+L_{k, n+i+j}\right]^3.
\end{align*}
\end{corollary}
\begin{corollary}
If  $n, i, j, t, m$ are positive integers with $0 < t < i$, $i+1 < m$, $j = 1$, then
\begin{align*}
&det {\left[
          \begin{array}{ccc}
           \bigskip
            1+L_{k, n+t}& 1 & 1 \\
             \bigskip
           1&  1+2 L_{k, n+i}& 1\\
     \bigskip
             1& 1 & 1+L_{k, n+i+j}\\       
          \end{array}
        \right]}\\
        & = \lbrace 2L_{k, n+i}L_{k, n+t}L_{k, n+i+j}\rbrace \lbrace \dfrac{1}{L_{k, n+t}}+ \dfrac{1}{2L_{k, n+i}}+\dfrac{1}{L_{k, n+i+j}}+1\rbrace \\
        &\lbrace 2L_{k, n+i}L_{k, n+t}L_{k, n+i+j}+2 L_{k, n+i}L_{k, n+i+j}+L_{k, n+t}L_{k, n+t}L_{k, n+i+j}+2L_{k, n+i}L_{k, n+t}\rbrace.
\end{align*}
\end{corollary}
\section{Concluding remark}
In this chapter, some new identities obtained for the $k-$Fibonacci and $k-$Lucas sequences using matrix methods. Also, we obtained determinantal identities for $k$ Lucas sequence.
% ------------------------------------------------------------------------

%=========================================================