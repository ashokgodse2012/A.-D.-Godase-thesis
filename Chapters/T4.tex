
%\def\baselinestretch{1}

\chapter{Abelian Theorem and Representation Theorem  for Generalized Sumudu Transform}

\lhead{\scriptsize\itshape\medskip Introduction}
\section{Introduction}
In this chapter we consider another integral transform called the Sumudu transform over some suitable distributional space. In the literature we can see that many researcher have extended various integral transforms to the spaces of generalized functions and studied their operational calculus. Watugala\cite{R90} has introduced a new integral transform namely the Sumudu transform and applied it to solve the physical phenomenon in science and engineering. The fundamental properties of Sumudu transform can be seen in \cite{R89,R90,R91,R95,R96}. Moreover, the applications of Sumudu transform without resorting to a new frequency domain are presented in \cite{R5,R6,R7,R11,R16,R17,R42,R43,R44}. Watugala\cite{R91} have extended the Sumudu transform to two variables with emphasis on solutions to partial differential equations. In\cite{R10}, we can have the applications to convolution type integral equations along with a Laplace-Sumudu duality. Sadikali\cite{R69}presented the abelian theorem for classical Sumudu transform.\\
The  exhaustive literature survey revels that many authors have contributed their work in the extension of Sumudu transform to a generalized function and Bohemians. In \cite{R42,R81} we can have generalized Sumudu transform and its fundamental properties. The distribution theory provides powerful analytical technique to solve many problems that arises in the applied field. In this chapter we prove the abelian theorems and representation theorem for Sumudu transformable generalized functions.
\lhead{\scriptsize\itshape\medskip Sumudu Transformation}
\subsection{The Sumudu Transformation}
For the function $f(t)$ the Sumudu transform is defined by the equation \cite{R90}
\begin{equation}
\mathbb{S}[f(t)] = G(u) = \int_{0}^{\infty}e^{-t}f(ut)dt   \hspace{0.5 cm} u \in (-\tau_{1}, \tau_{2})
 \end{equation} 
provided the integral on the right hand side exists. The Sumudu transform of functions $ f(t)$ ($t\geq 0 $) are come to exists which are piecewise continuous and of exponential order defined over the set\\
 A=[$f(t)$/$\exists$ M,$\tau_{1}$, $\tau_{2}$ $>$ 0 ,$|$f(t)$|$ $<$ M $e^{\frac{|t|}{\tau_{j}}}$ , if t $\in(-1)^{j}\times[0,\infty)$ ]\\
The above transform can be reduced to following form with suitable change in the variable
 \begin{equation}
 \mathbb{S}[f(t)] = G(u) = \frac{1}{u}\int_{0}^{\infty}e^{\frac{-t}{u}}f(t)dt   
 \end{equation}
 The inverse Sumudu transform of function $ G(u)$ is denoted by symbol $\mathbb{S}^{-1}[G(u)]= f(t)$ and is defined with Bromwich contour integral\cite{R60}
 \begin{equation}
 \mathbb{S}^{-1}[G(u)] = f(t) = \lim _{T\rightarrow\infty} \frac{1}{2\Pi i}\int_{\gamma-iT}^{\gamma+iT}e^{st}G(u)du
 \end{equation}
If $R(s,u)$ is the Natural transform, $F(s)$ is the Laplace transform and $G(u)$ is Sumudu transform of the function $f(t) \in A $, then we can have Sumudu-Laplace and Natural-Sumudu duality as
 \begin{equation}
G(\frac{1}{s})=sF(s) \hspace{0.5cm} F(\frac{1}{u})=uG(u)
 \end{equation}
 and
  \begin{equation}
\mathbb{N}[f(t)] = R(s,u) = \int_{0}^{\infty}e^{-st}f(ut)dt =\frac{1}{s}G(\frac{u}{s})
 \end{equation}
 In comparison with the Laplace transform, the Sumudu transform was shown to have units preserving properties and hence may be used to solve problems without resorting to the frequency domain.
\lhead{\scriptsize\itshape\medskip Generalized Sumudu Transform}
\section{Generalized Sumudu Transform}
        In\cite{R42}, authors has extended the Sumudu transform to a certain spaces of distributions using the theory developed by L.Schwartz\cite{R72,R73}. We extend this study to prove the abelian theorem and representation theorem for generalized Sumudu transform. Here we list some required results.
\subsection{Testing function space $\mathfrak{K}_{a,b}$}
 Let $\mathfrak{K}_{a,b}$ denotes the space of all complex valued smooth functions $\phi(t)$ on $-\infty<t<\infty$ on which the functions $\gamma_{k}(\phi)$ defined by\\
 \begin{equation}
 \gamma_{k}(\phi)\triangleq \gamma_{a,b,k}(\phi)\triangleq \underset{0<t<\infty}{Sup.}\vert K_{a,b}(t)D^{k}(t)\vert <\infty
 \end{equation}
 Where \begin{align*}
  K_{a,b}(t) = \begin{cases}
  e^{at}\hspace{0.5cm}  0\leq t<\infty \\
  e^{bt}\hspace{0.5cm}  -\infty<t<0.
  \end{cases}
 \end{align*}
This $\mathfrak{K}_{a,b}$ is linear space under the pointwise addition of function and their multiplication by complex numbers. Each $\gamma_{k}$ is clearly a seminorm on $\mathfrak{K}_{a,b}$ and $\gamma_{0}$ is a norm. We assign the topology generated by the sequence of seminorm ${(\gamma_{k})}_{k=0}^{\infty}$ there by making it a countably multinormed space. Note that for each fixed $u$ the kernel $\frac{1}{u}e^{\frac{-t}{u}}$ as a function of $t$ is a member of $\mathfrak{K}_{a,b}$ iff $a<Re(\frac{1}{u})<b$. With the usual argument we can show that $\mathfrak{K}_{a,b}$ is complete and hence a Frechet space. $\mathfrak{K}_{a,b}^{'}$ denotes the dual of $\mathfrak{K}_{a,b}$ i.e. $f$ is member of $\mathfrak{K}_{a,b}^{'}$ iff it is continuous linear function on $\mathfrak{K}_{a,b}$. Thus $\mathfrak{K}_{a,b}^{'}$ is a space of generalized functions. Note that the properties of testing function space $\mathfrak{K}_{a,b}$ will follows from \cite{R98}.\\
Now we define the generalized Sumudu Transform. Given a generalized Sumudu transformable generalized function $f(t)$,the strip of definition $\Omega_{f}$ for $\mathbb{S}[f]$ is a set in $\mathbb{C}$ defined by $\Omega_{f}\triangleq\lbrace u:\omega_{1}<Re(\frac{1}{u})<\omega_{2}\rbrace$ since $f$ or each $ u\in\Omega_{f}$ the kernel $\frac{1}{u}e^{\frac{-t}{u}}$ as a function of $t$ is a member of $\mathfrak{K}_{\omega_{1},\omega_{2}}^{'}$.For $f\in \mathfrak{K}_{\omega_{1},\omega_{2}}^{'}$, we can define the generalized Sumudu transform of $f(t)$ as conventional function
 \begin{equation}
  G_{f}(u)\triangleq \mathbb{S}[f(t)]\triangleq<f(t),\frac{1}{u}e^{\frac{-t}{u}}>
 \end{equation}
 We call $\Omega_{f}$ the region (or strip) of definition for $\mathbb{S}[f(t)]$ and
$\omega_{1}$ and $\omega_{2}$ are the abscissas of definition. Note that the properties like linearity and continuity of generalized Sumudu transform will follows from \cite{R98}.
The boundedness property for the generalized Sumudu transform is given by equation\\
\begin{equation*}
<f(t),\frac{1}{u}e^{\frac{-t}{u}}>  \leq \frac{1}{\vert u \vert}M.\underset{0\leq k\leq r}{Max}\underset{t}{Sup.}\vert K_{a,b}(t)D^{k}_{t}e^{\frac{-t}{u}} \vert
\end{equation*}
\lhead{\scriptsize\itshape\medskip Analyticity Theorem}
\begin{theorem}
\textbf{[Analyticity Theorem]}\\
If $ G_{f}(u)\triangleq \mathbb{S}[f(t)]$ for $u \in\Omega_{f}$ then 
$ G_{f}(u)$ is analytic on $\Omega_{f}$ and 
\begin{equation}
DG_{f}(u)=\frac{1}{u}<f(t),-\frac{1}{u}e^{\frac{-t}{u}}>
\end{equation}
\end{theorem}
\begin{proof}
 Let $u$ be arbitrary but fixed point in           $\Omega_{f}=\lbrace  u:\omega_{1}<Re(\frac{1}{u})<\omega_{2}\rbrace$. Choose the real positive number $a,b$ and $r$ such that
$\omega_{1}<a<Re(\frac{1}{u})-r<Re(\frac{1}{u})+r<b<\omega_{2}$.\\
Let $\Delta{u}$ be the complex increment such that $\vert\Delta{u}\vert<r$ and as $\Delta{u}\neq0$ we have
\begin{equation}
\frac{ G_{f}(u+\Delta{u})-G_{f}(u)}{\Delta{u}}-<f(t),\frac{\partial}{\partial{u}}\frac{1}{u}e^{\frac{-t}{u}}>=<f(t),\theta_{\Delta{u}}(t)>
\end{equation}
Where, $\theta_{\Delta{u}}(t)=\frac{1}{\Delta{u}}[e^{\frac{-t}{u+\Delta{u}}}-e^{\frac{-t}{u}}]-\frac{\partial}{\partial{u}}\frac{1}{u}e^{\frac{-t}{u}}$\\
Note that $\theta_{\Delta{u}}\in \mathfrak{K}_{a,b}$ so that equation (4.2.4) is meaningful. We shall now show that, as $\vert\Delta{u}\vert\rightarrow 0,   \theta_{\Delta{u}}(t)$ converges to zero in $\mathfrak{K}_{a,b}$. Since $f\in \mathfrak{K}_{a,b}^{'}$ this will imply that $ <f(t),\theta_{\Delta{u}}(t)>\rightarrow 0$. From equation(4.2.4)and choose $a$ close to $\omega_{1}$ and $b$ close to $\omega_{2}$ which gives the analyticity of $G_{f}(u)$ on $\Omega_{f}$.\\
Let $\mathfrak{C}$ denotes the circle with center at $\frac{1}{u}$ and radius $r_{1}$ where $0<r<r_{1}<min(Re(\frac{1}{u})-a,b-Re(\frac{1}{u}))$. We may interchange differentiation on $u$ with differentiation on $t$ and using Cauchy integral formula 
\begin{align*}
(-D_{t})^{k}\theta_{\Delta{u}}(t)&=\frac{1}{\Delta{u}}[(\frac{1}{u+\Delta{u}})^{k}e^{\frac{-t}{u+\Delta{u}}}-(\frac{1}{u})^{k}e^{\frac{-t}{u}}]-\frac{\partial}{\partial{u}}(\frac{1}{u})^{k+1}e^{\frac{-t}{u}}\\
&= \frac{1}{2\Pi i\Delta{u}}\int_{C}[\frac{1}{\xi-(\frac{1}{u+\Delta{u}})}-\frac{1}{\xi-(\frac{1}{u})}]\xi^{k}e^{\frac{-t}{\xi}}d\xi\\
&-\frac{1}{2\Pi i}\int_{C}\frac{\xi^{k}e^{\frac{-t}{\xi}}}{(\xi-(\frac{1}{u}))^{2}}d\xi\\
&=\frac{\Delta{u}}{2\Pi i}\int_{C}\frac{\xi^{k}e^{\frac{-t}{\xi}}}{(\xi-(\frac{1}{u+\Delta{u}}))(\xi-(\frac{1}{u}))^{2}}d\xi
\end{align*}
Now for all $\xi\in C $ and $-\infty<t<\infty$,
$\vert K_{a,b}(t)\xi^{k}e^{\frac{-t}{\xi}}\vert\leq M$ where $M$ is constant independent of $\xi$ and $t$. Moreover $\vert\xi-(\frac{1}{u+\Delta{u}})\vert>r_{1}-r>0$ and $\vert\xi-(\frac{1}{u})\vert =r_{1}$
\begin{align*}
 \vert K_{a,b}(t)D^{k}(t)\theta_\Delta{u}\vert&\leq\frac{\vert\Delta{u}\vert}{2\Pi}\int_{C}\frac{M}{(r_{1}-r)r_{1}^{2}} \vert d\xi\vert\\
 &\leq\frac{\vert\Delta{u}\vert M}{(r_{1}-r)r_{1}^{2}}
\end{align*}
 The right hand side is independent of $t$ and converges to zero as $\vert\Delta{u}\vert\rightarrow 0$. This shows that $\theta_{\Delta{u}}$ converges to zero in $\mathfrak{K}_{a,b}$ as $\vert\Delta{u}\vert\rightarrow 0$ which completes the proof of theorem.
 \end{proof}
 %%%%%%%%%%%%%%%%%%
 \lhead{\scriptsize\itshape\medskip Characteriztion Theorem}
 \begin{theorem}
  [\textbf{Characteriztion Theroem}] 
 The necessary condition for the function $G_{f}(u)$ to be the Sumudu transform of generalized function $f(t)$ are that $G_{f}(u)$ is analytic on $\Omega_{f}$ and for each closed strip $\lbrace u:a\leq Re(\frac{1}{u})\leq b\rbrace $ of $\Omega_{f}$ there be a polynomial such that $\vert G_{f}(u)\vert\leq P(\vert\frac{1}{u}\vert)$ for $a\leq Re(\frac{1}{u})\leq b$. The polynomial P will depend in general on $a$ and $b$.
\end{theorem}

\begin{proof}
 The analyticity of $G_{f}(u)$ has been already proved in the previous theorem. By the definition of the Sumudu transform, $f$ is a member of $\mathfrak{K}_{a,b}^{'}$ where $\omega_{1}<a<b<\omega_{2}$ so that there exists a constant $M$ and non-negative integer $r$ such that for $a\leq Re(\frac{1}{u})\leq b$
 \begin{align*}
 \vert G_{f}(u)\vert &= \vert <f(t),\frac{1}{u}e^{\frac{-t}{u}}>\vert\\
&\leq\frac{1}{\vert u\vert}M \underset{0\leq k\leq r}{Max}\underset{t}{Sup}\vert K_{a,b}(t)D_{t}^{k}e^{\frac{-t}{u}}\vert\\
&\leq\frac{1}{\vert u\vert}M \underset{0\leq k\leq r}{Max}{\vert\frac{1}{u}\vert}^{k}\underset{t}{Sup}\vert K_{a,b}(t)D_{t}^{k}e^{\frac{-t}{u}}\vert\\
&\leq P(\vert\frac{1}{u}\vert)
 \end{align*}
 This polynomial $P(\vert\frac{1}{u}\vert)$ depends in general on the choices of $a$ and $b$.
 \end{proof}
 \lhead{\scriptsize\itshape\medskip Abelian Theorem}
 \section{Abelian Theorem}
 In this section, we prove the initial value theorem and final value theorem for the Sumudu transform and then we prove these results for Sumudu transformable functions.
 \subsection{Initial Value Theorem}
 \begin{theorem}
  Let $ f(t)\in \mathfrak{L}_{1}[0,\infty) $ and if
\begin{enumerate}
\item[i)]there exists a real number $c$ such that $\int_{0}^{\infty}\vert f(t)e^{\frac{-t}{c}}\vert dt < \infty $
\item[ii)]there exists a complex number $\alpha$ and a real number $n>-1$ such that\\
$\underset{t\rightarrow 0^{+}}\lim \frac{f(t)}{t^{n}}=\alpha$ then
$\underset{u\rightarrow \infty}\lim \frac{G(u)}{u^{n}\Gamma(n+1)}=\alpha$
\end{enumerate}
where $\Gamma$ is the Eulers gamma function
\end{theorem}
\begin{proof}
 We know that for $ n>-1 (n \in \mathbb{R}) $ and $ u>0 $
\begin{equation}
\frac{1}{u}\int_{0}^{\infty}t^{n}e^{\frac{-t}{u}}dt=\Gamma(n+1).u^{n}
\end{equation}
Using this result and assuming that $y>0$ we can write
\begin{align*}
\vert {\frac{G(u)}{u^{n}}-\alpha\Gamma(n+1)}\vert&=\frac{1}{u^{n}}\vert {G(u)-\alpha\Gamma(n+1)u^{n}}\vert\\
&=\frac{1}{u^{n+1}}\vert\int_{0}^{\infty}e^{\frac{-t}{u}}(f(t)-\alpha t^{n})dt\vert
\end{align*}
\begin{equation*}
\leq \frac{1}{u^{n+1}}\vert\int_{0}^{y}e^{\frac{-t}{u}}(f(t)-\alpha t^{n})dt\vert + \frac{1}{u^{n+1}}\vert\int_{y}^{\infty}e^{\frac{-t}{u}}(f(t)-\alpha t^{n})dt\vert\\
\end{equation*}
\begin{equation*}
\leq I_{1} + I_{2}
\end{equation*}
Now
\begin{align*}
I_{1}&\leq \int_{0}^{y}e^{(\frac{-t}{u})}(\frac{t}{u})^{n}d(\frac{t}{u})\underset{0<t<y}{Sup.}\vert(\frac{f(t)}{t^{n}}-\alpha )\vert\\
&\leq \Gamma(n+1)\underset{0<t<y}{Sup.}\vert(\frac{f(t)}{t^{n}}-\alpha )\vert
\end{align*}
Since from (ii) $\underset{t\rightarrow 0^{+}}\lim \frac{f(t)}{t^{n}}=\alpha $, then we can choose y so small,so that $ \vert(\frac{f(t)}{t^{n}}-\alpha )\vert  < \frac{\epsilon}{2\Gamma(n+1)} $\\
\text{thus}\hspace{0.5cm}\begin{equation}
I_{1} <\frac{\epsilon}{2} 
\end{equation}
Now we can choose $\frac{1}{c} > 0$ so that $\int_{0}^{\infty}\vert e^{(\frac{-t}{c})}(\frac{f(t)}{t^{n}}-\alpha )\vert < \infty $. Then for $u<c$,we have\\
\begin{align*}
I_{2}&=\frac{1}{u^{n+1}}\vert\int_{y}^{\infty}e^{-(\frac{1}{u}-\frac{1}{c})t}e^{(\frac{-t}{c})}(f(t)-\alpha t^{n})\vert dt\\
&\leq K(c,\alpha) \frac{1}{u^{n+1}}e^{-(\frac{1}{u}-\frac{1}{c})y}
\end{align*}
where $ K(c,\alpha) = \int_{0}^{\infty} e^{(\frac{-t}{c})}\vert f(t)-\alpha t^{n}\vert dt $ is a constant.\\
Now choose $u$ so large, that 
\begin{equation}
I_{2} < \frac{\epsilon}{2}
\end{equation}
Hence for sufficiently large $u$, and from (4.3.2),(4.3.3) we have that
$ \underset{u\rightarrow \infty}\lim \frac{G(u)}{u^{n}\Gamma(n+1)}=\alpha $
\end{proof}
To prove the initial value theorem for generalized Sumudu transform we require following definition.
\begin{definition}
Let $T$ be a distribution defined in a neighborhood of a point, then we say that $T$ has a value $C$ at $x_{0}$ if $T(x_{0})=C$, if the distributional limit $T(x_{0}+\lambda x)$ exists in a neighborhood of zero as $\lambda$ tends to zero. This concept has been introduced by Lojasiewicz.
\end{definition}
Now we prove the initial value theorem for generalized Sumudu transform.
\begin{theorem}
If $f\in \mathfrak{K}_{a,b}^{'}$ and 
$\underset{t\rightarrow 0^{+}}\lim \frac{f(t)}{t^{n}}=\alpha$ in the sense of Lojasiewicz then,
$\vert<f(t)-\alpha t^{n},\frac{1}{u}e^{\frac{-t}{u}}>\vert<\epsilon$ as $u\rightarrow \infty$\\
\end{theorem}
\begin{proof}
We have that from \cite{R53}
\begin{equation}
 f(t)-\alpha t^{n} = D^{p}F
\end{equation} 
where $F$ is a continuous function in a neighbourhood of zero and $ D^{p}F $ is a finite order derivative of continuous function which is distribution by Zemanian\cite{R75}.\\
Hence left hand side of (4.3.4) equation can be written as $ <f(t),\frac{1}{u}e^{\frac{-t}{u}}> $ i.e. $f(t)$ is a distribution.\\
Now using the boundedness properties of generalized function we have
\begin{align*}
<f(t),\frac{1}{u}e^{\frac{-t}{u}}> & \leq \frac{1}{\vert u \vert}M.\underset{0\leq k\leq r}{Max}\underset{t}{Sup.}\vert K_{a,b}(t)D^{k}_{t}e^{\frac{-t}{u}} \vert\\
& \leq \frac{1}{\vert u \vert}M.\underset{0\leq k\leq r}{Max}\underset{t}{Sup.}\vert K_{a,b}(t)(\frac{-1}{u})^{k}e^{\frac{-t}{u}} \vert\\
&\leq \frac{1}{(\vert u \vert)^{k+1}}M.\underset{0\leq k\leq r}{Max}\underset{t}{Sup.}\vert K_{a,b}(t)e^{\frac{-t}{u}} \vert  \\ 
&\leq \epsilon \hspace{0.5cm} \text{as}\hspace{0.5cm} u \rightarrow \infty    
\end{align*}
where $ M,\epsilon,r $ are constants.
\end{proof}
\subsection{Final Value Theorem}
\begin{theorem}
Let $ f(t)\in \mathfrak{L}_{1}[0,\infty) $ and if
\begin{enumerate}
\item[i)]there exists a real number "$c$" such that $\int_{0}^{\infty}\vert f(t)e^{\frac{-t}{c}} \vert dt < \infty $ 
\item[ii)]there exists a complex number $ \alpha $ and a real number $ n>-1 $ such that\\
$\underset{t\rightarrow \infty}\lim \frac{f(t)}{t^{n}}=\alpha$ then
$\underset{u\rightarrow 0^{+}}\lim \frac{G(u)}{u^{n}\Gamma(n+1)}=\alpha$
\end{enumerate}
where $\Gamma $ is the Eulers gamma function.
\end{theorem}
\begin{proof}
 Assuming that $ u>0 $ and $ y>0 $ we can write
 \begin{align*}
\vert {\frac{G(u)}{u^{n}}-\alpha\Gamma(n+1)}\vert &=\frac{1}{u^{n}}\vert {G(u)-\alpha\Gamma(n+1)u^{n}}\vert\\
&=\frac{1}{u^{n+1}}\vert\int_{0}^{\infty}e^{\frac{-t}{u}}(f(t)-\alpha t^{n})dt\vert
\end{align*}
\begin{equation*}
\leq \frac{1}{u^{n+1}}\vert\int_{0}^{y}e^{\frac{-t}{u}}(f(t)-\alpha t^{n})dt\vert + \frac{1}{u^{n+1}}\vert\int_{y}^{\infty}e^{\frac{-t}{u}}(f(t)-\alpha t^{n})dt\vert\\
\end{equation*}
\begin{equation*}
\leq I_{1} + I_{2}
\end{equation*}
Now we will evaluate the $ I_{1} $ and $ I_{2} $ as follows
\begin{align*}
I_{1}&\leq \int_{0}^{y}e^{(\frac{-t}{u})}(\frac{t}{u})^{n}d(\frac{t}{u})\underset{0<t<\infty}{Sup.}\vert(\frac{f(t)}{t^{n}}-\alpha )\vert\\
&\leq \Gamma(n+1)\underset{0<t<\infty}{Sup.}\vert(\frac{f(t)}{t^{n}}-\alpha )\vert
\end{align*}
Since $\underset{t\rightarrow \infty}\lim \frac{f(t)}{t^{n}}=\alpha$ then we can choose y so large,so that $ \vert(\frac{f(t)}{t^{n}}-\alpha )\vert < \frac{\epsilon}{2\Gamma(n+1)} $ \\
\begin{equation}
\text{thus}\hspace{0.2cm} I_{1} <\frac{\epsilon}{2}
\end{equation}
Now we can choose $ \frac{1}{c} > 0 $ such that $ \int_{0}^{\infty}\vert e^{(\frac{-t}{c})}(\frac{f(t)}{t^{n}}-\alpha)\vert < \infty $.Then for $ u<c $,we have\\
\begin{align*}
I_{2}&=\frac{1}{u^{n+1}}\vert\int_{y}^{\infty}e^{-(\frac{1}{u}-\frac{1}{c})t}e^{(\frac{-t}{c})}(f(t)-\alpha t^{n})dt\vert\\
&\leq K \frac{1}{u^{n+1}}e^{-(\frac{1}{u}-\frac{1}{c})y}
\end{align*}
where $ K(c,\alpha) = \int_{0}^{\infty} e^{(\frac{-t}{c})}\vert f(t)-\alpha t^{n}\vert dt $ is a constant.\\
Now choose $ u $ so small that 
\begin{equation}
I_{2} < \frac{\epsilon}{2} 
\end{equation}
Hence from equation (4.3.5),(4.3.6) we get the required result that
$\underset{u\rightarrow 0^{+}}\lim \frac{G(u)}{u^{n}\Gamma(n+1)}=\alpha$
\end{proof}
Now we will extend this final value theorem for generalized Sumudu transform.
\begin{theorem}
Assume that $f\in \mathfrak{K}_{a,b}^{'}$ and $f(t)$ can be decomposed into $ f=f_{1}+f_{2} $ where $ f_{1} $ is ordinary function and $ f_{2}\in \mathcal{E}^{'}(I) $. Also assume that $ f_{1} $ satisfies the hypothesis of classical final value theorem. If $ G_{f}(u) $ is the generalized Sumudu transform of $f(t)$ then\\
$\underset{u\rightarrow 0^{+}}\lim \frac{G_{f}(u) }{u^{n}} = \Gamma{(n+1)}\underset{t\rightarrow \infty}\lim \frac{f(t)}{t^{n}}$\\
where $\mathcal{E}^{'}(I)$ is the space of distributions having compact support with respect to $I$.
\end{theorem}

\begin{proof}

Since $ f=f_{1}+f_{2} $ then we can have
$G_{f}(u)=G_{f_{1}}(u)+G_{f_{2}}(u)$\\
but $G_{f_{2}}(u) = <f_{2}(t),\frac{1}{u}e^{\frac{-t}{u}}>$\\
As $G_{f_{2}}(u)$ is smooth function and is of slow growth,we can relate the rate of growth of $G_{f_{2}}(u)$ to the order of $f_{2}(t)$. Let $\lambda(t)\in\mathcal{E}(I)$ be identically equal to 1 on a neighbourhood support of $ f_{2}(t) $
\begin{align*}
\vert G_{f_{2}}(u)\vert &= \vert< f_{2}(t),\lambda(t)\frac{1}{u}e^{\frac{-t}{u}}>\vert\\
&=C\underset{0\leq t<\infty}{Sup}\vert D_{t}^{r}[\lambda(t)\frac{1}{u}e^{\frac{-t}{u}}]\vert\\
&=C\underset{0\leq t<\infty}{Sup}\sum_{v=0}^{r}\binom{r}{v}\vert D_{t}^{r-v}\lambda(t)\vert \vert D_{t}^{v}\frac{1}{u}e^{\frac{-t}{u}}\vert\\
&=C_{1} \underset{0\leq t<\infty}{Sup}\vert \frac{1}{(u)^{v+1}}e^{\frac{-t}{u}}\vert\\
 \vert\frac{G_{f_{2}}(u)}{u^{n}} \vert & \leq C_{1}\frac{1}{\vert u^{n+v+1}\vert}\vert e^{\frac{-t}{u}}\vert\\
\underset{u\rightarrow 0^{+}}\lim \vert\frac{G_{f_{2}}(u)}{u^{n}} \vert &\leq \underset{u\rightarrow 0^{+}}\lim C_{1}\frac{1}{\vert u^{n+v+1}\vert}\vert e^{\frac{-t}{u}}\vert
\end{align*}
Substituting $ \frac{1}{u} = z $ and $ n+v+1=k $, we have
\begin{align*}
\underset{u\rightarrow 0^{+}}\lim \vert\frac{G_{f_{2}}(u)}{u^{n}} \vert &\leq C_{1}\underset{z\rightarrow \infty}\lim\vert(z)^{k}\vert \vert e^{-tz}\vert\\
&\leq C_{1}\underset{z\rightarrow \infty}\lim \vert\frac{(z)^{k}}{e^{tz}}\vert
\end{align*}
Using the L'Hospital rule we can show that right hand side of above inequality tends to $0$ as $z$ tends to $\infty$
\begin{align*}
\text{Hence}\hspace{0.5cm} \underset{u\rightarrow 0^{+}}\lim \frac{G_{f}(u) }{u^{n}}& =\underset{u\rightarrow 0^{+}}\lim \frac{G_{f_{1}}(u)}{u^{n}}+ \underset{u\rightarrow 0^{+}}\lim \frac{G_{f_{2}}(u) }{u^{n}}\\
&=\underset{u\rightarrow 0^{+}}\lim \frac{G_{f_{1}}(u)}{u^{n}}
\end{align*}
since $f_{1}$ is an ordinary function which satisfies the hypothesis of theorem, the required result follows.
\end{proof} 
\lhead{\scriptsize\itshape\medskip Representation Theorem}
\section{Representation Theorem}
In this section, we prove that every generalized function $f(t)\in \mathfrak{K}_{a,b}^{'}(I)$ can be represented by a finite sum of derivatives of continuous functions on $ I $. This proof is analogous to the method employed in structure theorems for Schwartz distribution\cite{R72,R73,R88}.
\begin{theorem}
Let $f(t)\in \mathfrak{K}_{a,b}^{'}(I)$ then $f(t)$ can be represented as a finite sum\\
\begin{equation}
f(t)=\sum_{i=0}^{n}K_{i}(\dfrac{d}{dt})^{i}\large[k_{a,b}H_{i}(t)]
\end{equation}
where $ H_{i}(t) $ are continuous functions on $ I $
\end{theorem}

\begin{proof}

 For every $ f(t)\in \mathfrak{K}_{a,b}^{'}(I) $, there exists a positive constant $C$ and a non-negative integer $r$ such that for all $\phi\in \mathfrak{K}_{a,b}(I)$\\
\begin{equation}
\vert<f(t),\phi(t)>\vert \leq C\underset{0\leq i\leq r}{Max}(\gamma_{a,b,i}(\phi))
\end{equation}
where $\mathfrak{K}_{a,b}(I)$ is the space of smooth functions with compact support in $I$, then we have \\
\begin{equation}
\vert<f(t),\phi(t)>\vert \leq C\underset{0\leq i\leq r}{Max}\underset{t}{Sup}\vert k_{a,b}(t)D^{i}\phi(t)\vert
\end{equation}
Now define $\phi_{r}(t)=k_{a,b}(t)\phi(t)$,
clearly $\phi_{r}(t)\in \mathfrak{K}_{a,b}(I)$. In this case we have that $\phi\rightarrow\phi_{r}$ is a one-to-one linear map of $\mathfrak{K}_{a,b}(I)$ onto itself. $ \phi(t)=k_{a,b}(t)^{-1}\phi_{r}(t)$\\
$ \dfrac{d\phi}{dt}= P k_{a,b}(t)^{-1}\phi_{r}(t)+k_{a,b}(t)^{-1}\dfrac{d\phi_{r}}{dt}$\\ 
where \begin{align*}
  P = \begin{cases}
  -a\hspace{0.5cm}  0\leq t<\infty \\
 -b\hspace{0.5cm}  -\infty<t<0.
  \end{cases}
 \end{align*}
Let $P_{1}= Max.[1,P]$ then\\
$\dfrac{d\phi}{dt}= P_{1} k_{a,b}(t)^{-1}[\phi_{r}(t)+\dfrac{d\phi_{r}}{dt}]$\\
continuing in this way,we get\\
\begin{equation}
(\dfrac{d\phi}{dt})^{i}= P_{i}k_{a,b}(t)^{-1}\sum_{q=0}^{i}(\dfrac{d}{dt})^{q}\phi_{r}(t)
\end{equation} 
where $P_{i}= \underset{k}{Max}[\binom{i}{k}P^{k-i}]$\\
Using above equation in (4.4.3),we have
\begin{align*}
\vert<f(t),\phi(t)>\vert &\leq C^{'}\underset{0\leq i\leq r}{Max}\underset{t}Sup\vert\sum_{q=0}^{i}(\dfrac{d}{dt})^{q}\phi_{r}(t)\vert
\end{align*}
\begin{align}
&\leq C^{"}\underset{0\leq i\leq r}{Max}\underset{t}Sup\vert(\dfrac{d}{dt})^{i}\phi_{r}(t)\vert
\end{align}
where $C^{'},C^{"}$ are constants.\\
 Now for every $\psi(t)\in \mathfrak{D}(I)$,we have
 \begin{equation}
 \underset{t}Sup\vert\psi(t)\vert\leq\underset{t}Sup\int_{0}^{t}(\dfrac{d\psi}{dx})dx
 \leq \Vert\dfrac{d\psi}{dx}\Vert_{L_{1}(I)}
 \end{equation}
where $L_{1}(I)$ is the space of equivalence classes of Lebesgue integrable functions on $I$.\\
$\therefore$ the equation(4.4.5) becomes
\begin{equation}
\vert<f(t),\phi(t)>\vert \leq C^{\prime\prime\prime}\underset{1\leq i\leq r+1}{Max}\Vert(\dfrac{d}{dt})^{i}\phi_{r}(t)\Vert_{L_{1}(I)}
\end{equation}
where $C^{\prime\prime\prime}$ is another constant.\\
Now consider the linear mapping of $\mathfrak{K}_{a,b}(I)$ into $ L_{1}(I) $ as
$\psi(t)\rightarrow \vert(\dfrac{d}{dt})^{i}\psi(t)\vert_{1\leq i \leq r+1}$\\
Since $\mathfrak{K}_{a,b}(I)$ is linear manifold of $ L_{1}(I)$, equation (4.4.7) can be read as linear functional. The function $\psi(t)$ is continuous on $\mathfrak{K}_{a,b}(I)$ for the topology induced on it by $ L_{1}(I) $. Therefore by using Hahn-Banach theorem,$f(t)$ can be expanded as continuous linear functional on the whole of $ L_{1}(I) $. But dual of $L_{1}(I)$ is isomorphic with $L_{\infty}(I)$,dual space of all equivalence classes of complex valued function on $I$. Hence for each $f(t)\in L_{\infty}(I)$ there exists an $R$ such that $\vert f(t)\vert\leq R $ almost everywhere.\\
Hence there exists functions $g_{i}\in L_{\infty}(I), 1\leq i \leq r+1$ such that
\begin{align*}
 \vert<f(t),\phi(t)>\vert&=\sum_{i=1}^{r+1} <g_{i},(\dfrac{d}{dt})^{i}\phi_{r}(t)>\\
& =\sum_{i=1}^{r+1} <(-1)^{i}\dfrac{d}{dt})^{i}g_{i}k_{a,b}(t),\phi(t)>
 \end{align*}
 \begin{equation}
 f=\sum_{i=1}^{r+1}(-1)^{i}(\dfrac{d}{dt})^{i}g_{i}k_{a,b}(t)
 \end{equation}
Let for each $i$ we set
\begin{equation*}
h_{i}(t)=(-1)^{i}\int_{0}^{t}g_{i}(x)dx
\end{equation*}
Since $g_{i}\in L_{\infty}(I)$,the function $h_{i}(t)$ are also continuous on I and 
\begin{align*}
\vert h_{i}(t)\vert &\leq \int_{0}^{t}\vert g_{i}(x)dx\vert\\
&\leq \vert t \vert \underset{I}{Max}\vert g_{i}\vert\\
&\leq \vert t \vert \Vert g_{i}\Vert_{L_{\infty}(I)} 
\end{align*}
 Also $g_{i} = (-1)^{i}(\dfrac{d}{dt})h_{i}$\\
Hence
\begin{equation}
f = \sum_{i=2}^{r+2}k_{a,b}(t)(\dfrac{d}{dt})^{i}h_{i} 
\end{equation}
Let $r+2 = k$ and using the differentiation formulae
\begin{equation}
v(t)(\dfrac{d}{dt})^{i}h_{i} = \sum_{j=0}^{i}(-1)^{i}\binom{i}{j}[v^{j}h_{i}]^{i-j} 
\end{equation}
and\begin{equation}
(ab)^{j}= \sum_{j=0}^{i}\binom{j}{q}[a^{j-q}b^{q}]
\end{equation} 
We can write equation (4.4.9) as in (4.4.1) where the $H_{i}$ are the continuous functions of $h_{i}$ and therefore continuous functions on $I$.
\end{proof}


%%%%%%%%%%%%%%%%%%%%%%%%%%%%%%%%%%%%%%%%%%%%%%%%%
