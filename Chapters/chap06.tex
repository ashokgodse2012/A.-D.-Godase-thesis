

\let\textcircled=\pgftextcircled
\chapter{On the Properties of Generalized Fibonacci Polynomials}
\label{chap:On the Properties of Generalized Fibonacci  Polynomials}
In this chapter, we introduce two new generalizations of Fibonacci polynomial. We produce an extended Binet's formula for these generalized polynomials and thereby identities such as Simpson's, Catalan's, d'Ocagene's, etc. using matrix algebra. Moreover, we derived some identities of $M_{n}(x)$, $\widehat{F}_{n}(x)$ and $\widehat{L}_{n}(x)$ using matrix and vector methods. 
\vspace{2mm}
\let\thefootnote\relax\footnote{\textbf{\hspace{-0.78cm}The content of this chapter is published in the following papers.}}\footnote{\hspace{-0.78cm}On the properties of generalized Fibonacci like polynomials, Int. J. Adv. Appl. Math. and Mech. 2-3(2015), 234-251.}
\footnotetext{\hspace{-0.78cm}A matrix representation of a generalized Fibonacci polynomial, Journal of New Theory, 19(2019), 01-19.}
\footnotetext{\hspace{-0.78cm}Vector Approach to a New Generalization of Fibonacci Polynomial, Journal of New Theory, 17(2017), 45–56.}
\section{Introduction}
The Fibonacci polynomial has been generalized in many ways, some by preserving the initial conditions and others by preserving the recurrence relation. The Fibonacci polynomial $\{F_{n}(x)\}_{n\in\mathbb{N}}$  is defined as,
 $F_{n+1}=k(x)F_{n}(x)+F_{n-1}(x) $ with $F_{0}(x)=0 ,F_{1}(x)=1 $ for $n\geq{1}$. The Lucas polynomial $\{L_{n}(x)\}_{n\in\mathbb{N}}$  is defined as, $L_{n+1}(x)=k(x)L_{n}(x)+L_{n-1}(x)$ with $L_{0}(x)=2 ,L_{1}(x)=k(x)$ for $n\geq{1}$.
In this chapter, we establish two new generalizations of Fibonacci polynomial. First generalization  $\{M_{n}(x)\}$ of Fibonacci polynomial  generated by the recurrence relation, $M_{n+1}(x)=k(x)M_{n}(x)+M_{n-1}(x)for, n\geq{2}$, with initial conditions $M_{0}(x)=2$ and $M_{1}(x)=m(x)+k(x)$, where $k(x)$, $m(x)$ are polynomials with real coefficients. The second generalized Fibonacci polynomial $\widehat{F}_{ n}(x)$ and generalized Lucas polynomial $\widehat{L}_{ n}(x)$ are defined by the recurrence relation 
\begin{align*}
&\widehat{F}_{ n+1}(x)=x\widehat{F}_{n}(x)+\widehat{F}_{ n-1}(x) \quad\text{with} \quad \widehat{F}_{ 0}(x)=0\text{,}\quad \widehat{F}_{1}(x)=x^2+4\quad \text{and}\\
&\widehat{L}_{ n+1}(x)=x \widehat{L}_{ n}(x)+\widehat{L}_{ n-1}(x) \quad\text{with} \quad \widehat{L}_{ 0}(x)=2x^2+8\text{,}\quad \widehat{L}_{ 1}(x)=x^3+4x,\\&\text{for } n\geq{1},\quad \text{respectively}.
\end{align*}
\begin{table}[H]
\begin{center}
%\begin{footnotesize}
\begin{tabular} {|c|c|c|c|}\hline \label{d}
Polynomial &Initial value &Initial value &Recursive Formula \\
&$G_0(x)=a(x)$&$G_1(x)=b(x)$&$G_{n+1}(x)=a(x)G_{n}(x)+b(x)G_{n-1}(x)$\\\hline\hline
Fibonacci&0&1&$F_{n+1}(x) = F_{n}(x)+F_{n-1}(x)$\\
Lucas&2&x&$L_{n+1}(x) = L_{n}(x)+L_{n-1}(x)$\\
Pell&0&1&$P_{n+1}(x) = 2xP_{n}(x)+P_{n-1}(x)$\\
Pell-Lucas&2&2x&$Q_{n+1}(x) = 2xQ_{n}(x)+Q_{n-1}(x)$\\
Jacobsthal&0&1&$J_{n+1}(x) = J_{n}(x)+2xJ_{n-1}(x)$\\
Jacobsthal-Lucas&2&1&$j_{n+1}(x) = j_{n}(x)+2xj_{n-1}(x)$\\
Generalized Fibonacci&0&$x^2+4$&$\widehat{F}_{n+1}(x)=x \widehat{F}_{ n}(x)+\widehat{F}_{ n-1}(x)$\\
Generalized Lucas&$2x^2+8$&$x^3+4x$&$\widehat{L}_{n+1}(x)=x \widehat{L}_{n}(x)+\widehat{L}_{ n-1}(x)$\\
\hline
\end{tabular} 
\caption{{Recurrence relation of some GFP.}}
\label{Tab1} 
%\end{footnotesize}
\end{center}
\end{table}
\vspace{3mm}
\section{The Generalized Fibonacci Polynomial $M_{n}(x)$}
In this section, we introduce the new generalization of the  Fibonacci polynomial.
\begin{definition}
For any polynomial with real coefficients $k(x)\geq{1}$ and $m(x)\geq{0}$ the generalized Fibonacci polynomial $M_{n}(x),n\geq{1}$ is defined by:
$M_{n+1}(x)=k(x)M_{n}(x)+M_{n-1}(x),$  for , $n\geq{1}$,with $M_{0}(x)=2,M_{1}(x)=m(x)+k(x)$
\end{definition}
The Characteristic equation of the initial recurrence relation is  $r^{2}-k(x)r-1=0$, characteristic roots are $r_{1}=\dfrac{k(x)+\sqrt{k^{2}(x)+4}}{2}$ and $r_{2}=\dfrac{k(x)-\sqrt{k^{2}(x)+4}}{2}$. Characteristic roots verify the properties
\begin{align*}
r_{1}-r_{2}=\sqrt{k^{2}(x)+4},r_{1}+r_{2}=k(x),r_{1}.r_{2}=-1.
\end{align*}
It is clear from the definition of the Generalized Fibonacci polynomial that it satisfy
\begin{align*}
&M_{n}(x)= m(x) F_{n}(x)+L_{n}(x),\quad\text{ for  $n\geq{0}$}.
\end{align*}
\subsection{First Few Terms of the Generalized Fibonacci Polynomial}
\begin{align*}
&M_{0}(x)=2,\\
&M_{1}(x)=m(x)+k(x),\\
&M_{2}(x)=k^2(x)+m(x) k(x)+2,\\
&M_{3}(x)=k^3(x)+m(x) k^2(x)+3k(x)+m(x),\\
&M_{4}(x)=k^4(x)+m(x) k^3(x)+4k^2(x)+2m(x) k(x)+2,\\
&M_{5}(x)=k^5(x)+m(x) k^4(x)+5k^3(x)+3m(x) k^2(x)+5k(x)+m(x),\\
&M_{6}(x)=k^6(x)+m(x)k^5(x)+6k^4(x)+4m(x) k^3(x)+9k^2(x)+3 m(x) k(x)+2,\\
&M_{7}(x)=k^7(x)+m(x) k^6(x)+7k^5(x)+5m(x) k^4(x)+14k^3(x)+6m(x) k^2(x)\\&+7k(x)+m(x),\\
&M_{8}(x)=k^8(x)+m(x) k^7(x)+8k^6(x)+6m(x) k^5(x)+20k^4(x)+10m(x)k^3(x)\\&+16k^2(x)+4m(x)k(x)+2,\\
&M_{9}(x)=k^9(x)+m (x)k^8(x)+9k^7(x)+7m (x)k^6(x)+27k^5(x)+15m (x)k^4(x)\\&+30k^3(x)+10m (x)k^2(x)+9k(x)+m(x),\\
&M_{10}(x)=k^{10}(x)+m(x)k^9(x)+10k^8(x)+8m (x)k^7(x)+35k^6(x)+21m (x)k^5(x)\\&+50k^4(x)+20m (x)k^3(x)+25k^2(x)+5m(x)k(x)+2,\\
&M_{11}(x)=k^{11}(x)+m (x)k^{10}(x)+11k^9(x)+9m (x)k^8(x)+44 k^7(x)+28m (x)k^6(x)\\&+77k^5(x)+35m (x)k^4(x)+45k^3(x)+15m (x)k^2(x)+11k(x)+m(x).		
\end{align*}
\subsection{Fundamental Properties of the Generalized Fibonacci Polynomial}
In this section, we establish some Fundamental properties of  the generalized Fibonacci polynomial.
\begin{theorem} (Binet form)
The $M_{n}(x),n\in{N}$  is given by
\begin{align*}
M_{n}(x)=\dfrac{\left[ m(x)+r_{1}-r_{2}]r_{1}^{n}-[m(x)+r_{2}-r_{1}\right] r_{2}^{n}}{r_{1}-r_{2}}.
\end{align*}
\end{theorem}
\begin{proof}(1)
Using Binet's formula for Fibonacci polynomials and Lucas polynomials, we have
\begin{align*}
M_{n}(x) &= m(x)(x)F_{n}(x)+L_{n}(x),\\
M_{n}(x) &= m(x)(\frac{{r}_{1}^n-{r}_{2}^n}{r_{1}-r_{2}})+({r}_{1}^n+{r}_{2}^n),\\
M_{n}(x)&=\frac{[m(x)+r_{1}-r_{2}]r_{1}^{n}-[m(x)+r_{2}-r_{1}]r_{2}^{n}}{r_{1}-r_{2}}.
\end{align*}
hence proof.
\end{proof}
\begin{proof}(2)
The general form of the generalized Fibonacci polynomial may be expressed in the form
$$M_{n}(x)=A(x) r_{1}^n+B(x) r{2}^n$$ for some polynomials $A(x)$ and $B(x)$.
The polynomials $A(x)$ and $B(x)$ can be determined by the initial conditions $M_{0}=2=A(x)+B(x),M_{1}=m(x)+k(x)=A(x)r_{1}+B(x)r_{2}$.
Solving above equation system for $A(x)$ and $B(x)$, we get
\begin{align*}
A(x) & = \frac{{m(x)+k(x)}-2r_{2}}{r_{1}-r_{2}},\\
B(x) & = \frac{2r_{1}-[m(x)+k(x)]}{r_{1}-r_{2}},\\
\because\quad
2r_{1}& = k(x)+\sqrt{k^{2}(x)+4},2r_{2}=k(x)-\sqrt{k^{2}(x)+4},\\
\therefore \quad
M_{n}(x)& = \frac{[m(x)+\sqrt{k^{2}(x)+4}] {r}_{1}^n}{r_{1}-r_{2}} - \frac{[m(x)-\sqrt{k^{2}(x)+4}] {r}_{2}^n}{r_{1}-r_{2}},\\
\because \quad r_{1}-r_{2}& = \sqrt{k^{2}(x)+4},\\
\therefore\quad M_{n}(x)& = \frac{[m(x)+r_{1}-r_{2}]{r}_{1}^n-[m(x)+r_{2}-r_{1}]r_{2}^n}{r_{1}-r_{2}},\\
M_{n}(x)& = \frac{X({r}_{1}^n-Y{r}_{2}^n)}{r_{1}-r_{2}},
\end{align*}
where, $X={m(x)+r_{1}-r_{2}}$  and  $Y={m(x)+r_{2}-r_{1}}$.
\end{proof}
\begin{theorem} For $n, r\geq{1}$, we have
\begin{align*}
M_{n-r}(x).M_{n+r}(x)-{{M}_{n}(x)}^2={(-1)}^{n-r}{[{k^{2}(x)-m^{2}(x)+4}]F_{r}^2(x)}. 
\end{align*}
\end{theorem}
\begin{proof}
Using Binet's formula, we have
\begin{align*}
&M_{n-r}(x).M_{n+r}(x)-{{M}_{n}}^2(x) = \dfrac{{X}{r}_{1}^{n-r}-{Y}{r_{2}^{n-r}}}{r_{1}-r_{2}}.\dfrac{{X}{r}_{1}^{n+r}-{Y}{r_{2}^{n+r}}}{r_{1}-r_{2}}-[\dfrac{{X}{r}_{1}^{n}-{Y}{r_{2}^{n}}}{r_{1}-r_{2}})]^2\\
& = \dfrac{XY}{(r_{1}-r_{2})^2}.[2({r_{1}r_{2}})^n-r_{1}^{n-r}r_{2}^{n+r}-r_{2}^{n-r}r_{1}^{n+r}],\\
& = \dfrac{-[m^{2}(x)-k^{2}(x)-4]}{(r_{1}-r_{2})^2}[(r_{1}r_{2})^n(\dfrac{r_{2}}{r_{1}})^r+(r_{1}r_{2})^n(\dfrac{r_{1}}{r_{2}})^r-2((r_{1}r_{2})^n)],\\
& = \dfrac{-[m^{2}(x)-k^{2}(x)-4]}{(r_{1}-r_{2})^2}(r_{1}r_{2})^{n-r}[r_{1}^{2r}-2{(r_{1}r_{2})}^r+r_{2}^{2r}],\\
&= {[k^2(x)-m^2(x)+4]}{(-1)}^{n-r}F_{r}^2(x).
\end{align*}
Hence proof.
\end{proof}
\begin{theorem} For $n\geq{1}$, we have
\begin{align*}
M_{n-1}(x).M_{n+1}(x)-{{M}_{n}}^2(x)={(-1)}^{n+1}{[{k^{2}(x)-m^{2}(x)+4}]}.
\end{align*}
\end{theorem}
\begin{proof}
Consider the ${2}\times{2}$ linear system
$$M_{n}(x)u(x)+M_{n-1}(x)v(x)=M_{n+1}(x),$$
$$M_{n+1}(x)u(x)+M_{n}(x)v(x)=M_{n+2}(x).$$
$\because \quad {M}_{n}^2(x)-M_{n-1}(x)M_{n+1}(x)\neq{0}.$
This system has unique solution such that
 $$
\begin{bmatrix}{c}
    u(x) \\
    v(x) \\
\end{bmatrix}=
          \begin{bmatrix}{c}
            k(x) \\
           1\\
          \end{bmatrix}
        $$
Using Cramer's rule, we have
\begin{align*}
&v(x)=\dfrac{ \begin{vmatrix}
    M_{n}(x) & M_{n+1}(x) \\
    M_{n+1}(x) & M_{n+2}(x)
     \end{vmatrix}}
{  \begin{vmatrix}
 M_{n}(x) & M_{n-1}(x) \\
 M_{n+1}(x) & M_{n}(x) 
             \end{vmatrix}  }=1.\\
        &\therefore\quad M_{n+2}(x).M_{n}(x)-{M}_{n+1}^2(x)={M}_{n}^2(x)-M_{n-1}(x).M_{n+1}(x).\\
&\text{Now let,}\\
&P_{n}(x) =  M_{n-1}(x).M_{n+1}(x)-{M}_{n}^2(x),\\
&P_{n+1}(x) =  M_{n}(x).M_{n+2}(x)-{M}_{n+1}^2(x).\\
&\text{It gives that}\quad P_{n+1}(x)=-P_{n}(x),n\geq{1},\\
&\text{and}\quad P_{1}(x)=k^{2}(x)-m^{2}(x)+4.
\end{align*}
Solving the recurrence relation, $P_{n+1}(x)+P_{n}(x)=0$, with $P_{1}(x)=k^{2}(x)-m^{2}(x)+4$, we get
\begin{align*}
&P_{n}(x)={(-1)}^{n+1}{[k^{2}(x)-m^{2}(x)+4]}.
\end{align*}
Thus,\\ $M_{n-1}(x).M_{n+1}(x)-{{M}_{n}}^2(x)={(-1)}^{n+1}{[{k^{2}(x)-m^{2}(x)+4}]}$, for $n\geq{1}$.
\end{proof}
\begin{theorem}
For any integer $n$
$\left[ k^{2}(x)+4\right] M_{n}^2(x)+4{(-1)}^n\left[ m^{2}(x)-k^{2}(x)-4\right] $ is always a perfect square.
\end{theorem}
\begin{proof}
Case:(1) If $n$ is even
\begin{align*}
&{[k^{2}(x)+4]}M_{n}^2(x)+4{[m^{2}(x)-k^{2}(x)-4]}={[(m(x)+r_{1}-r_{2})r_{1}^n+(m(x)+r_{2}-r_{1})r_{2}^n]}.\\
&\text{Case:(2) If $n$ is odd}\\
&{[k^{2}(x)+4]}M_{n}^2(x)-4{[m^{2}(x)-k^{2}(x)-4]}={[(m(x)+r_{1}-r_{2})r_{1}^n-(m(x)+r_{2}-r_{1})r_{2}^n]}.
\end{align*}
Therefore the number $\left[ k^{2}(x)+4\right] M_{n}^2(x)+4{(-1)}^n\left[ m^{2}(x)-k^{2}(x)-4\right]$ is always a perfect square.
\end{proof}
\begin{theorem}
For $n,r\geq{1}$
\begin{align}
M_{r}(x)M_{n+1}(x)-M_{r+1}(x)M_{n}(x)=[k^{2}(x)-m^{2}(x)+4](-1)^nF_{r-n}(x).
\end{align}
\end{theorem}
\begin{proof}
Using Binet's formula,  we have
\begin{align*}
&M_{r}(x)\cdot M_{n+1}(x)-M_{r+1}(x)M_{n}(x)\\&=\frac{Xr_{1}^r-Yr_{2}^r}{r_{1}-r_{2}}\cdot\frac{Xr_{1}^{n+1}-Yr_{2}^{n+1}}
{r_{1}-r_{2}}-\frac{Xr_{1}^{r+1}-Yr_{2}^{r+1}}{r_{1}-r_{2}}\cdot\frac{Xr_{1}^{n}-Yr_{2}^{n}}{r_{1}-r_{2}},\\
&=\frac{XY(r_{1}^rr_{2}^n)(r_{1}-r_{2})-XY(r_{1}^nr_{2}^r)(r_{1}-r_{2})}{(r_{1}-r_{2})^2},\\
&={XY(-1)^nF_{r-n}(x)},\\
&=[k^{2}(x)-m^{2}(x)+4](-1)^nF_{r-n}(x).
\end{align*}
Hence proof.
\end{proof}
\begin{theorem}
For $r\geq{1},$
\begin{align*}
&M_{r}(x)\cdot M_{n+1}(x)+M_{r-1}(x)M_{n}(x)=M_{n+r}(x)+[k^2(x)\\&-\sqrt{k^{2}(x)+4}+m^{2}(x)+4]F_{n+r}+m(x)[2+\frac{1}{\sqrt{k^{2}(x)+4}}]L_{n+r}(x).
\end{align*}
\end{theorem}
\begin{proof}
Using Binet's formula,
\begin{align*}
&M_{r}(x).M_{n+1}(x)+M_{r-1}(x)M_{n}(x)=\\
&\frac{X^2r_{1}^{n+r}(r_{1}+\frac{1}{r_{1}})-XY[r_{1}^rr_{2}^n(r_{2}+\frac{1}{r_{1}})+r_{1}^nr_{2}^r(r_{1}+\frac{1}{r_{2}})]+Y^2r_{2}^{n+r}[r_{2}+\frac{1}{r_{2}}]}{(r_{1}-r_{2})^2},\\
&\because r_{1}.r_{2}=-1
\end{align*}
\begin{align*}
 &\therefore (r_{1}+\frac{1}{r_{2}})=0,(r_{2}+\frac{1}{r_{1}})=0,\\
 &\text{and}\quad (r_{1}+\frac{1}{r_{1}})=r_{1}-r_{2},(r_{2}+\frac{1}{r_{2}})=-(r_{1}-r_{2}).\\
&\therefore =\frac{X^{2}r_{1}^{n+r}-Y^{2}r_{2}^{n+r}}{(r_{1}-r_{2})}.\\
&\text{It gives that} \\
&M_{r}(x).M_{n+1}(x)+M_{r-1}(x)M_{n}(x)=M_{n+r}(x)+[k^2(x)\\
&-\sqrt{k^{2}(x)+4}+m^{2}(x)+4]F_{n+r}+m(x)[2+\frac{1}{\sqrt{k^{2}(x)+4}}]L_{n+r}(x).
\end{align*}
\end{proof}
\begin{theorem}
\begin{align*}
\lim_{n \to \infty }\frac{M_{n}(x)}{M_{n-r}(x)}=r_{1}^r.
\end{align*}
\end{theorem}
\begin{proof}(1)
Consider
\begin{align*}
&\lim_{n \to \infty }\dfrac{M_{n}(x)}{M_{n-r}(x)} =\lim_{n \to \infty }\dfrac{Xr_{1}^n-Yr_{2}^n}{Xr_{2}^{n-r}-Yr_{2}^{n-r}}=\lim_{n \to \infty }\dfrac{r_{1}^n[X-Y(\dfrac{r_{2}}{r_{1}})^n]}{r_{1}^{n-r}[X-Y(\dfrac{r_{2}}{r_{1}})^{n-r}]}.\\
&\text{Taking into account that} \lim_{n \to \infty }(\frac{r_{2}}{r_{1}})^n=0, \quad\text{and} \lim_{n \to \infty }(\frac{r_{2}}{r_{1}})^{n-r}=0.\\& \text{ since}
{|r_{2}|}<|r_{1}|. \quad\text{It gives that,}\quad
\lim_{n \to \infty }\frac{M_{n}(x)}{M_{n-r}(x)}= r_{1}^r.
\end{align*}
\end{proof}
\begin{proof}(2)
\begin{align*}
\text{It can be observed that sequence} \quad \{t_{n}(x)\}_{n=1}^{n=\infty}=\{\frac{M_{n}}{M_{n-1}}\}_{n=1}^{n=\infty} \quad \text{is convergent.}
\end{align*}
Let, $\lim_{n \to \infty }t_{n}(x)=t(x)$, a polynomial with real coefficients.
\begin{align*}
&\because \dfrac{M_{n+1}(x)}{M_{n}(x)}=k(x)+\dfrac{M_{n-1}(x)}{M_{n}(x)}\\
&\text{and} \quad k(x)>0, \text{for} \quad x>0, \text {we have,}\\
&\lim_{n \to \infty }\frac{M_{n+1}(x)}{M_{n}(x)}= k(x)+\lim_{n \to \infty }\dfrac{M_{n-1}(x)}{M_{n}(x)}= k(x)+\dfrac{1}{\lim_{n \to \infty }\dfrac{M_{n}(x)}{M_{n-1}(x)}}\\
&t(x)= k(x)+\frac{1}{t(x)}.
\end{align*}
It gives that, $t^{2}(x)-k(x)t(x)-1=0$, for $x>0$. This equation has single positive root $r_{1}$.
\begin{align*}
&\because \lim_{n \to \infty }\frac{M_{n}(x)}{M_{n-1}(x)}= r_{1},\\
&\frac{M_{n}(x)}{M_{n-r}(x)}= \frac{M_{n}(x)}{M_{n-1}(x)}\frac{M_{n-1}(x)}{M_{n-2}(x)}\cdots\cdots\frac{M_{n-r+1}(x)}{M_{n-r}(x)},\\
&\therefore\lim_{n \to \infty }\frac{M_{n}(x)}{M_{n-r}(x)}= \lim_{n \to \infty }\frac{M_{n}(x)}{M_{n-1}(x)}\lim_{n \to \infty}\frac{M_{n-1}(x)}{M_{n-2}(x)}\cdots\cdots\lim_{n \to \infty }\frac{M_{n-r(x)+1}}{M_{n-r}(x)},\\
&\therefore\lim_{n \to \infty }\frac{M_{n}(x)}{M_{n-r}(x)}= r_{1}^r.
\end{align*}
\end{proof}
\begin{theorem}
The following equalities are valid for $a,b,c\in{\mathbb{N}}$
\begin{align*}
&1.\quad M_{a+b-1}(x)= \frac{1}{X}{[M_{a}(x)M_{b}(x)+M_{a-1}(x)M_{b-1}(x)-2Yr_{2}^{a+b-1}]},\\
&2.\quad M_{a+b-2}(x)= \frac{1}{k(x)X}{[M_{a}(x)M_{b}(x)-M_{a-2}(x)M_{b-2}(x)-2k(x)Yr_{2}^{a+b-2}]},\\
&3.\quad M_{a+b+c-3}(x)= \frac{1}{k(x)X^2}[M_{a}(x)M_{b}(x)M_{c}(x)+k(x)M_{a-1}(x)M_{b-1}(x)M_{c-1}(x)\\&-M_{a-2}(x)M_{b-2}(x)M_{c-2}(x)-2k(x)Yr_{2}^{a+b-3}{(M_{a}(x)r_{2}+M_{a-1}(x)+Xr_{2}^a)}].
\end{align*}
\end{theorem}
\begin{proof} (1)
Using Binet's formula we have
\begin{align*} 
&M_{a}(x)M_{b}(x)+M_{a-1}(x)M_{b-1}(x)=\frac{Xr_{1}^a-Yr_{2}^a}{r_{1}-r_{2}}.\frac{Xr_{1}^b-Yr_{2}^b}{r_{1}-r_{2}}\\&+\frac{Xr_{1}^{a-1}-Yr_{2}^{a-1}}{r_{1}-r_{2}}.\frac{Xr_{1}^{b-1}-Yr_{2}^{b-1}}{r_{1}-r_{2}},\\
&\because r_{1}r_{2}=-1,\\
&=\frac{X^2[r_{1}^{a+b}+r_{2}^{a+b-2}]+Y^2[r_{2}^{a+b}+r_{2}^{a+b-2}]}{(r_{1}-r_{2})^2},\\
&=\frac{X^2r_{1}^{a+b-1}-Y^2r_{2}^{a+b-1}}{r_{1}-r_{2}}.\\
&\because X-Y=2(r_{1}-r_{2},\\
&M_{a+b-1}(x)= \frac{1}{X}{[M_{a}(x)M_{b}(x)+M_{a-1}(x)M_{b-1}(x)-2Yr_{2}^{a+b-1}]}.
\end{align*}
\end{proof}
\begin{theorem}
The generating function for the generalized Fibonacci  polynomial $\left\{M_{n}(x)\right\}_{n=0}^{\infty}$ is given by $G(t)=\dfrac{2+\left[m(x)-k(x)\right]t}{1-k(x)t-t^2}$.
\end{theorem}
\begin{proof}
We begin with formal power series representation of the generating function for $\left\{M_{n}(x)\right\}$
\begin{align*}
F(t)&=M_{0}(x)+M_{1}(x)t+M_{2}(x)t^2+\cdots\cdots\cdots+M_{n}(x)t^n+\cdots\cdots\\
&=\sum_{n=0}^{n=\infty}t^n M_{n}(x),\\
&[1-k(x)t-t^2]F(t)=M_{0}(x)+t[M_{1}(x)-k(x)M_{0}(x)]\\&+t^2[M_{2}(x)-k(x)M_{1}(x)-M_{0}(x)]+\cdots\cdots\cdots,\\
&=2+[m(x)-k(x)]t,\\
F(t)&=\frac{2+\left[m(x)-k(x)\right]t}{1-k(x)t-t^2}.
\end{align*}
\end{proof}
\begin{theorem}
If $F(t)=\sum_{n=0}^{n=\infty}t^n M_{n}(x)$, for $t\in(-\dfrac{1}{r_{1}},\dfrac{1}{r_{2}}),t>0$ then $M_{n}(x)=\dfrac{F^{(n)}(0)}{n!}$, where $F^{(n)}(t)$ denote the $n^{th}$ order derivative of the polynomial $F(t)$.
\end{theorem}
\begin{proof}
We have
\begin{align*}
M_{0}(x)t^0&=2,\\
F^{(1)}(t)&=\sum_{n=1}^{n=\infty}nM_{n}(x)t^{n-1}=M_{1}(x)+\sum_{n=2}^{n=\infty}nM_{n}(x)t^{n-1},\\
F^{(2)}(t)&=\sum_{n=2}^{n=\infty}n(n-1) M_{n}(x)t^{n-1}=2M_{2}(x)+\sum_{n=3}^{n=\infty}n(n-1)M_{n}(x)t^{n-2}\\\vdots\\
F^{(r)}(t)&=\sum_{n=r}^{n=\infty}n(n-1)(n-2)\cdots\cdots[n-(r-1)] M_{n}(x)t^{n-r}
\end{align*}
\begin{align*}
&=r(r-1)(r-2)\cdots\cdots2\times1M_{r}(x)\\&+\sum_{n=r+1}^{n=\infty}n(n-1)(n-2)\cdots\cdots[n-(r-1)] M_{n}(x)t^{n-r}\\
&=r!M_{r}(x)+\sum_{n=r+1}^{n=\infty}n(n-1)(n-2)\cdots\cdots[n-(r-1)] M_{n}(x)t^{n-r}.
\end{align*}
Put $t=0$ for $n\geq r$, it gives that $M_{r}(x)=\dfrac{F^{(r)}(0)}{r!}$. Hence for $n\geq{1}$, we have $M_{n}(x)=\dfrac{F^{(n)}(0)}{n!}$
\end{proof}
\begin{theorem}For $n\geq{1}$, we have
\begin{align*}
(-1)^{n}M_{-n}(x)=M_{n}(x)-2m(x)F_{n}(x).
\end{align*}
\end{theorem}
\begin{proof}
Using Binet form of $M_{n}(x)$, we have
\begin{align*}
(-1)^{n}M_{-n}(x)&=(r_{1}r_{2})^{n}\frac{Xr_{1}^{-n}-Yr_{2}^{-n}}{r_{1}-r_{2}}\\
&=\frac{Xr_{2}^{n}-Yr_{2}^{n}}{r_{1}-r_{2}}\\
&=\frac{Xr_{1}^{n}-Yr_{2}^{n}}{r_{1}-r_{2}}-(X+Y)\frac{r_{1}^{n}-r_{2}^{n}}{r_{1}-r_{2}}\\
(-1)^{n}M_{-n}(x)&=M_{n}(x)-2m(x)F_{n}(x).
\end{align*}
Hence proof.
\end{proof}
\begin{theorem}For $n\geq3, \quad r_{1}^{n-2} < M_{n}(x)$.
\end{theorem}
\begin{proof}
We use induction method to prove this theorem.
\begin{align*}
\text{Let}\quad P(n):M_{n}(x)>r_{1}^{n-2}, \quad\text{for $n\geq3$}.
\end{align*}
We must prove that $P(n)$ is true when $n\geq3$. First verify that $P(3)$ and $P(4)$ are true.
\begin{align*}
&\text{Now,}\quad r_{1}=\dfrac{k(x)+\sqrt{k^2(x)+4}}{2},\quad \frac{k(x)}{2}<k^{2}(x)+1< m(x)(k^{2}(x)+1),\\
&\dfrac{\sqrt{k^{2}(x)+4}}{2}<\dfrac{k^{2}(x)+4}{2}<k^{2}(x)+3<k(x)(k^{2}(x)+3).\\
&\therefore\quad \dfrac{k(x)+\sqrt{k^{2}(x)+4}}{2}<k(x)[k^{2}(x)+3]+m(x)[k^2(x)+1].\\
&\text{i.e.}\quad M_{3}(x)>r_{1}^{3-2}.\\
&\text{Now,}\quad r_{1}^2=[\dfrac{k(x)\sqrt{k^2(x)+4}+k^2(x)+2}{2}].\\
&\because\quad\dfrac{\sqrt{k^2(x)+4}}{2}<\dfrac{k^2(x)+4}{2}<\frac{2k^2(x)+4}{2}<k^2(x)+2,
\end{align*}
\begin{align*}
&\dfrac{k(x)\sqrt{k^2(x)+4}}{2}<k(x)[k^2(x)+2]<m(x)k(x)[k^2(x)+2],\\
&\dfrac{k^2(x)+2}{2}<k^2(x)+1<2k^2(x)+1<2[2k^2(x)+2]<k^4(x)+2[2k^2(x)+2].\\
&\text{It gives that,}\\&\quad \dfrac{k(x)\sqrt{k^2(x)+4}+k^2(x)+2}{2}<k^4(x)+2[2k^2(x)+2]+m(x)k(x)[k^2(x)+2].\\
&\text{i.e.}\quad r_{1}^{4-2}<M_{4}(x).
\end{align*}
\begin{align*}
&\text{Now assume that $P(t)$ is true i.e. $r_{1}^{t-2}<M_{t}(x)$, for all $3\leq{t}\leq n,$ where $n\geq4$}.\\
&\text{Furthermore we must prove that $r_{1}^{n-1}<M_{n+1}(x)$.}\\
&\text{Since $r_{1}$ is a solution of $t^2-k(x)t-1=0$.}\\
&\text{It gives that $r_{1}^2=k(x)r_{1}+1h$}\\
&\text{and $r_{1}^{n-1}=r_{1}^2\cdot r_{1}^{n-3}=[k(x)r_{1}+1]r_{1}^{n-3}=k(x)r_{1}^{n-2}+r_{1}^{n-3}$.}\\
&\text{But, $r_{1}^{n-2}<M_{n}(x)$,} \text{$r_{1}^{n-3}<M_{n-1}(x)$,}\\
&\text{and, \quad $M_{n+1}(x)=k(x)M_{n}(x)+M_{n-1}(x)>k(x)r_{1}^{n-2}+r_{1}^{n-3}$.}\\
&\text{i.e. $r_{1}^{n-1}<M_{n+1}(x)$,}
\text{thus, $r_{1}^{n-2}< M_{n}(x)$, for $n\geq3$.}
\end{align*}
Hence proof.
\end{proof}
\subsection{Summation Properties of the Generalized Fibonacci Polynomial}
In this section, we establish some summation properties of  the generalized Fibonacci polynomial.
\begin{theorem}(Sum of the first n-terms)
\begin{align*}
\sum_{i=0}^{i=n}M_{i}(x)=\frac{1}{k(x)}[M_{n+1}(x)+M_{n}(x)-m(x)+k(x)+2].
\end{align*}
\end{theorem}
\begin{proof}
Let,
\begin{align*}
&T_{n}(x)=\sum_{i=0}^{i=n}M_{i}(x)=M_{0}(x)+M_{1}(x)+M_{2}(x)+\cdots\cdots+M_{n}(x)\\
&=[m(x)F_{0}(x)+L_{0}(x)]+[m(x)F_{1}(x)+L_{1}(x)]+\cdots\cdots+[m(x)F_{n}(x)+L_{n}(x)]\\
&=m(x)[F_{0}(x)+F_{1}(x)+\cdots\cdots+F_{n}(x)]+[L_{0}(x)+L_{1}(x)+\cdots\cdots+L_{n}(x)]\\
&=\frac{m(x)}{k(x)}[F_{n+1}(x)+F_{n}(x)-1]+\frac{1}{k(x)}[L_{n+1}(x)+L_{n}(x)-2]+1.\\
&\therefore \quad \sum_{i=0}^{i=n}M_{i}(x)=\frac{1}{k(x)}[M_{n+1}(x)+M_{n}(x)-m(x)+k(x)+2].
\end{align*}
\end{proof}
\begin{theorem}(Sum of the first n-terms with odd indices)
\begin{align*}
\sum_{i=0}^{i=(n-1)}M_{2i+1}(x)=\frac{1}{k(x)}[M_{2n}(x)+2].
\end{align*}
\end{theorem}
\begin{proof}
Using Binet form of $M_{n}(x)$, we have
\begin{align*}
&\sum_{i=0}^{i=n-1}M_{2i+1}(x)=\sum_{i=0}^{n-1}[\frac{Xr_{1}^{2i+1}-Yr_{2}^{2i+1}}{r_{1}-r_{2}}]\\
&=\frac{X}{r_{1}-r_{2}}\sum_{i=0}^{i=n-1}r_{1}^{2i+1}-\frac{Y}{r_{1}-r_{2}}\sum_{i=0}^{i=n-1}r_{2}^{2i+1}\\
&=\frac{X}{r_{1}-r_{2}}\frac{r_{1}[1-r_{1}^{2n}]}{1-r_{1}^2}-\frac{Y}{r_{1}-r_{2}}\frac{r_{2}[1-r_{2}^{2n}]}{1-r_{2}^2}\\
&\because\quad
1-r_{1}^2=-k(x)r_{1},
1-r_{2}^2=-k(x)r_{2}.\\
&\text{We get,}\quad
\sum_{i=0}^{i=n-1}M_{2i+1}(x)=\frac{1}{k(x)}[\frac{Xr_{1}^{2n}-Yr_{2}^{2n}-X+Y}{r_{1}-r_{2}}]=\frac{1}{k(x)}[M_{2n}(x)+2].
\end{align*}
\end{proof}
\begin{theorem}
\begin{equation}
\sum_{i=0}^{i=n-1}M_{2i}(x)=\frac{1}{k(x)}[M_{2n-1}(x)+k(x)-m(x)].
\end{equation}
\end{theorem}
\begin{proof}
Using Binet form of $M_{n}(x)$, we have
\begin{align*}
&\sum_{i=0}^{i=n-1}M_{2i}(x)=\sum_{i=0}^{n-1}\frac{Xr_{1}^{2i}-Yr_{2}^{2i}}{r_{1}-r_{2}}=\frac{X}{r_{1}-r_{2}}\sum_{i=0}^{i=n-1}r_{1}^{2i}-\frac{Y}{r_{1}-r_{2}}\sum_{i=0}^{i=n-1}r_{2}^{2i},\\
&=\frac{X}{r_{1}-r_{2}}\frac{[1-r_{1}^{2n}]}{1-r_{1}^2}-\frac{Y}{r_{1}-r_{2}}\frac{[1-r_{2}^{2n}]}{1-r_{2}^2}.\\
&\because\quad
1-r_{1}^2=-k(x)r_{1},
1-r_{2}^2=-k(x)r_{2}.\\
&\text{We get,}
\sum_{i=0}^{i=n-1}M_{2i}(x)=\frac{1}{k(x)}[\frac{Xr_{1}^{2n-1}-Yr_{2}^{2n-1}-Xr_{1}^{-1}+Yr_{2}^{-1}}{r_{1}-r_{2}}],\\
&=\frac{1}{k(x)}[M_{2n-1}(x)+\frac{k(x)-m(x)}{k(x)}],\\
&\sum_{i=0}^{i=n-1}M_{2i}(x)=\frac{1}{k(x)}[M_{2n-1}(x)+k(x)-m(x)].
\end{align*}
\end{proof}
\begin{theorem}
For an integer $n\geq0$, we have
$\displaystyle\sum_{i=0}^{i=n}\binom {n}{i}k^i(x)M_{2i}(x)=M_{2n}(x)$.
\end{theorem}
\begin{proof}
Using Binet form of $M_{n}(x)$, we have
\begin{align*}
\displaystyle\sum_{i=0}^{i=n}\left(^{n} _{i} \right)k^i(x)M_{2i}(x)
&=\displaystyle\sum_{i=0}^{i=n}\left(^{n} _{i} \right)k^i(x)\frac{Xr_{1}^i-Yr_{2}^i}{r_{1}-r{2}},\\
&=\frac{X}{r_{1}-r_{2}}\displaystyle\sum_{i=0}^{i=n}\left(^{n} _{i} \right)k^i(x)r_{1}^i-\frac{Y}{r_{1}-r_{2}}\sum_{i=0}^{i=n}\left(^{n} _{i} \right)k^i(x)r_{2}^i,\\
&=\frac{X}{r_{1}-r_{2}}(k(x)r_{1}+1)^n-\frac{Y}{r_{1}-r_{2}}(k(x)r_{2}+1)^n,\\
&=\frac{X}{r_{1}-r_{2}}r_{1}^{2n}-\frac{Y}{r_{1}-r_{2}}r_{2}^{2n},\\
&=M_{2n}(x).\\
&\therefore\quad\displaystyle\sum_{i=0}^{i=n}\left(^{n} _{i} \right)k^i(x)M_{2i}(x)=M_{2n}(x).
\end{align*}
\end{proof}
\begin{theorem}
For arbitrary integers $p,q\geq1$, we have
\begin{align*}
\displaystyle\sum_{i=1}^{p}M_{qi}(x)=\dfrac{M_{pq+q}(x)-(-1)^qM_{pq}(x)-M_{q}(x)+2(-1)^q}{r_{1}^q+r_{2}^q-(-1)^q-1}.
\end{align*}
\end{theorem}
\begin{proof}
Using Binet form of $M_{n}(x)$, we have
\begin{align*}
&\displaystyle\sum_{i=1}^{p}M_{qi}(x)=\sum_{i=1}^{p}\dfrac{{Xr_{1}^{qi}}-Yr_{2}^{qi}}{r_{1}-r_{2}}=\dfrac{X}{r_{1}-r_{2}}\sum_{i=1}^{p}r_{1}^{qi}-\dfrac{Y}{r_{1}-r_{2}}\sum_{i=1}^{p}r_{2}^{qi},\\
&=\dfrac{X}{r_{1}-r_{2}}[\dfrac{r_{1}^{pq+q}-r_{1}^q}{r_{1}^q-1}]-\dfrac{Y}{r_{1}-r_{2}}[\dfrac{r_{2}^{pq+q}-r_{2}^q}{r_{2}^q-1}],\\
&=\dfrac{X[r_{1}^{pq+q}-r_{1}^q][r_{2}^q-1]-Y[r_{2}^{pq+q}-r_{2}^q][r_{1}^q-1]}{[r_{1}-r_{2}][r_{1}^q-1][r_{2}^q-1]},\\
&=\dfrac{X}{r_{1}-r_{2}}[\dfrac{r_{1}^{pq+q}r_{2}^q-(r_{1}r_{2})^q-r_{2}^{pq+q}r_{1}^q}{[r_{1}r_{2}]^q-r_{1}^q-r_{2}^q+1}],\\
&=\dfrac{[\dfrac{Xr_{1}^{pq+q}-Yr_{2}^{pq+q}}{r_{1}-r_{2}}]-[\dfrac{Xr_{1}^{q}-Yr_{2}^{q}}{r_{1}-r_{2}}]-(-1)^q[\dfrac{Xr_{1}^{pq}-Yr_{2}^{pq}}{r_{1}-r_{2}}]+2(-1)^q}{r_{1}^q+r_{2}^q-(-1)^q-1},\\
&\sum_{i=1}^{p}M_{qi}(x)=\dfrac{M_{pq+q}(x)-(-1)^qM_{pq}(x)-M_{q}(x)+2(-1)^q}{r_{1}^q+r_{2}^q-(-1)^q-1}.
\end{align*}
\end{proof}
\begin{theorem}
For arbitrary integers $p,q,j\geq1$ with $j\geq{q}$
\begin{align*}
\displaystyle\sum_{i=1}^{p}M_{qi+j}(x)=\dfrac{M_{pq+q+j}(x)-(-1)^qM_{pq+j}(x)-M_{q+j}(x)+(-1)^qM_{j}(x)}{r_{1}^q+r_{2}^q-(-1)^q-1}.
\end{align*}
\end{theorem}
\begin{proof}
Using Binet form of $M_{n}(x)$, we have
\begin{align*}
&\displaystyle\sum_{i=1}^{p}M_{qi+j}(x)=\sum_{i=1}^{p}\dfrac{{Xr_{1}^{qi+j}}-Yr_{2}^{qi+j}}{r_{1}-r_{2}},\\
&=\dfrac{X}{r_{1}-r_{2}}\sum_{i=1}^{p}r_{1}^{qi+j}-\dfrac{Y}{r_{1}-r_{2}}\sum_{i=1}^{p}r_{2}^{qi+j},\\
&=\dfrac{X}{r_{1}-r_{2}}[\dfrac{{r_{1}^{pq+q+j}r_{2}^q}-(r_{1}r_{2})^qr_{1}^j-r_{1}^{pq+q+j}+r_{1}^{q+j}}{(r_{1}r_{2})^q-r_{1}^q-r_{2}^q+1)}]\\&-\dfrac{Y}{r_{1}-r_{2}}[\dfrac{{r_{2}^{pq+q+j}r_{1}^q}-(r_{1}r_{2})^qr_{2}^j-r_{2}^{pq+q+j}+r_{2}^{q+j}}{(r_{1}r_{2})^q-r_{1}^q-r_{2}^q+1)}],\\
&\displaystyle\sum_{i=1}^{p}M_{qi+j}(x)= \dfrac{M_{pq+q+j}(x)-(-1)^qM_{pq+j}(x)-M_{q+j}(x)+(-1)^qM_{j}(x)}{r_{1}^q+r_{2}^q-(-1)^q-1}.
\end{align*}
\end{proof}
\begin{theorem}
For arbitrary integers $n,j\geq1$, we have
\begin{align*}
\displaystyle\sum_{i=1}^{n}M_{i+j}(x)=\frac{1}{k(x)}[{M_{n+j+1}(x)+M_{n+j}(x)-M_{j}(x)-M_{j-1}(x)}].
\end{align*}
\end{theorem}
\begin{proof}
Using Binet form of $M_{n}(x)$, we have
\begin{align*}
&\displaystyle\sum_{i=1}^{n}M_{i+j}(x)= \sum_{i=1}^{n}[\dfrac{{Xr_{1}^{i+j}}-Yr_{2}^{i+j}}{r_{1}-r_{2}}]= \dfrac{X}{r_{1}-r_{2}}\sum_{i=1}^{n}r_{1}^{i+j}-\dfrac{Y}{r_{1}-r_{2}}\sum_{i=1}^{n}r_{2}^{i+j},\\
&= \dfrac{X}{r_{1}-r_{2}}r_{1}^j[\dfrac{r_{1}^{n+1}-1}{r_{1}-1}]-\dfrac{Y}{r_{1}-r_{2}}r_{2}^j[\dfrac{r_{2}^{n+1}-1}{r_{2}-1}],\\
&=\dfrac{1}{k}[\dfrac{Xr_{1}^{n+j+1}-Yr_{2}^{n+j+1}}{r_{1}-r_{2}}+\dfrac{Xr_{1}^{n+j}-Yr_{2}^{n+j}}{r_{1}-r_{2}}-\dfrac{Xr_{1}^{j}-Yr_{2}^{j}}{r_{1}-r_{2}}-\dfrac{Xr_{1}^{j-1}-Yr_{2}^{j-1}}{r_{1}-r_{2}}],\\
&\sum_{i=1}^{n}M_{i+j}(x)=\dfrac{1}{k(x)}[{M_{n+j+1}(x)+M_{n+j}(x)-M_{j}(x)-M_{j-1}(x)}].
\end{align*}
\end{proof}
\begin{theorem}(Sum of square)
\begin{align*}
\displaystyle\sum_{i=1}^{i=n}M_{i}^2(x)=\dfrac{M_{n}(x)M_{n+1}(x)-2M_{1}(x)}{k(x)}.
\end{align*}
\end{theorem}
\begin{proof}
\begin{align*}
&\text{Let,} \quad T(x)=\sum_{i=1}^{i=n}M_{i}^2(x).\\
&\because\quad M_{i}(x)=\frac{M_{i+1}(x)-M_{i-1}(x)}{k(x)}.\\
&\therefore\quad T(x)= \sum_{i=1}^{i=n}[\frac{M_{i+1}(x)-M_{i-1}(x)}{k(x)}]^2\\
&=\frac{1}{k^2(x)}[\sum_{i=1}^{i=n}M_{i+1}^2(x)-2\sum_{i=1}^{i=n}M_{i+1}(x).M_{i-1}(x)+\sum_{i=1}^{i=n}M_{i-1}^2(x)],\\
&= \frac{1}{k^2(x)}[\sum_{i=1}^{i=n}M_{i+1}^2(x)-2\sum_{i=1}^{i=n}M_{i}^2(x)+\sum_{i=1}^{i=n}M_{i-1}^2(x)-2(m^2(x)-k^2(x)-4)\sum_{i=1}^{i=n}(-1)^i],\\
&=\frac{1}{k^2(x)}[\sum_{i=2}^{i=n+1}M_{i}^2(x)-2\sum_{i=1}^{i=n}M_{i}^2(x)+\sum_{i=0}^{i=n-1}M_{i}^2(x)-2(m^2(x)-k^2(x)-4)\sum_{i=1}^{i=n}(-1)^i],\\
&= \frac{1}{k^2(x)}[(T(x)+M_{n+1}^2(x)-M_{1}^2(x))+(T(x)-M_{n}^2(x)+M_{0}^2(x))-2T(x)\\
&-2(m^2(x)-k^2(x)-4)\sum_{i=1}^{i=n}(-1)^i],\\
&\because M_{0}(x)=2,M_{1}(x)=m(x)+k(x),\\
&=\frac{1}{k^2(x)}[M_{n+1}^2(x)-M_{n}^2(x)-(m(x)+k(x))^2+4-2(m^2(x)-k^2(x)-4)\sum_{i=1}^{i=n}(-1)^i],\\
&=\frac{1}{k^2(x)}[M_{n+1}(x)(M_{n+1}(x)M_{n-1}(x))-(m(x)+k(x))^2+4\\&-(-1)^n(m^2(x)-k^2(x)-4)- 2(m^2(x)-k^2(x)-4)\sum_{i=1}^{i=n}(-1)^i],\\
&=\frac{1}{k^2(x)}[M_{n+1}(x)(k(x)M_{n}(x))-2k(x)(m(x)+k(x))],\\
            &\sum_{i=1}^{i=n}M_{i}^2(x)=\frac{M_{n}(x)M_{n+1}(x)-2M_{1}(x)}{k(x)}.
            \end{align*}
\end{proof}

\begin{theorem}
\begin{footnotesize}
\begin{align*}
&\displaystyle\sum_{j=1}^{i=n}jM_{j}(x)\\&=\dfrac{(nk(x)-1)M_{n+1}(x)+(nk(x)-2)M_{n}(x)-M_{n-1}(x)-(k^2(x)+k(x)(m(x)-2)+2(m(x)+2))}{k^2(x)}
\end{align*}
\end{footnotesize}
\end{theorem}
\begin{proof}
Consider
\begin{align*}
&M_{1}(x)+M_{2}(x)+\cdots\cdots+M_{n}=\frac{1}{k(x)}[M_{n+1}(x)+M_{n}(x)-(m(x)+k(x)+2)],\\
&M_{2}(x)+M_{3}(x)+\cdots\cdots+M_{n}(x)=\frac{1}{k(x)}[M_{n+1}(x)+M_{n}(x)-(m(x)+k(x)+2)]\\&-M_{1}(x),\\
&M_{3}(x)+M_{4}(x)+\cdots\cdots+M_{n}(x)=\frac{1}{k(x)}[M_{n+1}(x)+M_{n}(x)-(m(x)+k(x)+2)]\\&-M_{1}(x)-M_{2}(x)\\
&\vdots&\\
& M_{n}(x)=\frac{1}{k(x)}[M_{n+1}(x)+M_{n}(x)-(m(x)+k(x)+2)]-(M_{1}(x)-M_{2}(x)-\cdots\cdots\\&-M_{n-1}(x).\\
&\text{Sum of the left hand side is }
\sum_{j=1}^{i=n}jM_{j}(x).\\
&\text{We have for $n=1,2,3,\cdots\cdots, n-1$}\\
&\frac{1}{k(x)}[M_{n+1}(x)+M_{n}(x)-(m(x)+k(x)+2)]-(M_{1}(x)-M_{2}(x)-\cdots\cdots-M_{n-1}(x))\\
&=\frac{1}{k(x)}[M_{n+1}(x)+M_{n}(x)-(m(x)+k(x)+2)]\\&-\frac{1}{k(x)}[M_{t+1}(x)+M_{t}(x)-(m(x)+k(x)+2)].\\
&\text{ Summing the right hand side, we have}\\
&\sum_{j=1}^{i=n}jM_{j}(x)=\frac{n}{k(x)}[M_{n+1}(x)+M_{n}(x)-(m(x)+k(x)+2)]\\&-\frac{\displaystyle\sum_{i=1}^{i=n-1}M_{i}(x)+\displaystyle\sum_{i=2}^{i=n-1}M_{i}(x)-(n-1)(m(x)+k(x)+2)}{k(x)},\\
&=\frac{n}{k(x)}[M_{n+1}(x)+M_{n}(x)-(m(x)+k(x)+2)]-
\frac{1}{k(x)}[\frac{1}{k(x)}[M_{n-1}(x)\\&+M_{n}(x)-(m(x)+k(x)+2)]+\frac{1}{k(x)}[M_{n}(x)+M_{n+1}(x)-(m(x)+k(x)+2)-2]\\&-(n-1)(m(x)+k(x)+2)].
\end{align*}
\begin{footnotesize}
\begin{align*}
&\therefore\quad\sum_{j=1}^{i=n}jM_{j}(x)\\&=\frac{(nk(x)-1)M_{n+1}(x)+(nk(x)-2)M_{n}(x)-M_{n-1}(x)-(k^2(x)+k(x)(m(x)-2)+2(m(x)+2))}{k^2(x)}
\end{align*}
\end{footnotesize}
\end{proof}
\begin{theorem}
For each real number $p>r_{1}$, we have
\begin{align*}
\displaystyle\sum_{i=1}^{i=\infty}\frac{M_{i}(x)}{p^i}=\dfrac{p(m(x)+k(x))+2}{p^2-k(x)p-1}.	
\end{align*}
\end{theorem}
\begin{proof}We have
\begin{align*}
& \sum_{i=1}^{i=\infty}\frac{M_{i}(x)}{p^i}=\lim_{n \to \infty }\sum_{i=1}^{i=n}\frac{M_{i}(x)}{p^i},\\
&=\lim_{n \to \infty }\sum_{i=1}^{i=n}\frac{\frac{Xr_{1}^i-Yr_{2}^i}{r_{1}-r_{2}}}{p^i},\\	
&=\lim_{n \to \infty }\sum_{i=1}^{i=n}\frac{X(\frac{r_{1}}{p})^i-Y(\frac{r_{2}}{p})^i}{r_{1}-r_{2}},\\
&=\frac{1}{r_{1}-r_{2}}[\frac{Xr_{1}}{p-r_{1}}-\frac{Yr_{2}}{p-r_{2}}],\\
&=\frac{1}{r_{1}-r_{2}}[\frac{p(Xr_{1}-Yr_{2})+(X-Y)}{(p-r_{1})(p-r_{2})}].\\
&\because\quad Xr_{1}-Yr_{2}=(r_{1}-r_{2})(m(x)+r_{1}+r_{2}),X-Y=2(r_{1}-r_{2}).\\
&\therefore\quad\sum_{i=1}^{i=\infty}\frac{M_{i}(x)}{p^i}=\frac{p(m(x)+k(x))+2}{p^2-k(x)p-1}.
\end{align*}
\end{proof}
\subsection{Properties of Generalized Fibonacci Polynomial by Matrix Methods}
In this section, we obtain some  properties of  the generalized Fibonacci polynomials using matrix methods.
\begin{theorem}For $P={
          \begin{bmatrix}
            k(x) & 1 \\
            1& 0 \\
          \end{bmatrix}
        }$, we have
\begin{align*}
&{
 \begin{bmatrix}
    M_{n+1}(x) & M_{n}(x) \\
    M_{n}(x) & M_{n-1}(x) \\
  \end{bmatrix}
}=P^n{
          \begin{bmatrix}
            m(x)+k(x) & 2 \\
            2 & m(x)-k(x) \\
          \end{bmatrix}
        }.
\end{align*}
\end{theorem}
\begin{proof}
We use induction to prove this theorem. For $n=1$
\begin{align*}
&{ \begin{bmatrix}
    M_{2}(x) & M_{1}(x) \\
    M_{1}(x) & M_{0}(x) \\
  \end{bmatrix}
}={
          \begin{bmatrix}
            k(x)(m(x)+k(x))+2 & m(x)+k(x) \\
            m(x)+k(x) & 2 \\
          \end{bmatrix}
      },\\
&P{\begin{bmatrix}
            (m(x)+k(x)) & 2 \\
            2	& (m(x)-k(x)) \\
          \end{bmatrix}
       }={
          \begin{bmatrix}
            k(x) & 1\\
            1 & 0 \\
          \end{bmatrix}
      }{
          \begin{bmatrix}
            (m(x)+k(x)) & 2 \\
            2 & m(x)-k(x) \\
          \end{bmatrix}
        },\\	
        &={
          \begin{bmatrix}
            k(x)(m(x)+k(x))+2 & m(x)+k(x) \\
            m(x)+k(x) & 2 \\
          \end{bmatrix}
       },\\
		&\therefore 			
	{
 \begin{bmatrix}
    M_{2}(x) & M_{1}(x) \\
    M_{1}(x) & M_{0}(x) \\
  \end{bmatrix}
}=P^1{
          \begin{bmatrix}
            m(x)+k(x) & 2 \\
            2 & m(x)-k(x) \\
          \end{bmatrix}}.\\
&\text{Now, assume that result is true for $n-1$. }\\
&{
 \begin{bmatrix}
    M_{n}(x) & M_{n-1}(x) \\
    M_{n-1}(x) & M_{n-2}(x)\\
  \end{bmatrix}
}=P^{n-1}{
          \begin{bmatrix}
            m(x)+k(x) & 2 \\
            2 & m(x)-k(x) \\
          \end{bmatrix}
       }.\\				
&\text{Next, we prove that result is true for $n$.}\\
&P^{n}{
          \begin{bmatrix}
            m(x)+k(x) & 2 \\
            2 & m(x)-k(x) \\
          \end{bmatrix}
      }=P\left[ P^{n-1}{
          \begin{bmatrix}
            m(x)+k(x) & 2 \\
            2 & m(x)-k(x) \\
          \end{bmatrix}
      }\right], \\										
	&=P{
 \begin{bmatrix}
    M_{n}(x) & M_{n-1}(x) \\
    M_{n-1}(x) & M_{n-2}(x) \\
  \end{bmatrix}
},\\
&={
 \begin{bmatrix}
    k(x) & 1 \\
    1 & 0 \\
  \end{bmatrix}
)}{
 \begin{bmatrix}
    M_{n}(x) & M_{n-1}(x) \\
    M_{n-1}(x) & M_{n-2}(x) \\
  \end{bmatrix}
},\\					
&={
 \begin{bmatrix}
    M_{n+1}(x) & M_{n}(x)\\
    M_{n}(x) & M_{n-1}(x) \\
  \end{bmatrix}
}.\\
&\therefore \quad{
 \begin{bmatrix}
    M_{n+1}(x) & M_{n}(x) \\
    M_{n}(x) & M_{n-1}(x) \\
  \end{bmatrix}
}=P^n{\begin{bmatrix}
            m(x)+k(x) & 2 \\
            2 & m(x)-k(x)\\
          \end{bmatrix}
        }.
 \end{align*}	
 \end{proof}
\begin{theorem}(Simpson's identity for negative n)
\begin{align}
M_{-n+1}(x).M_{-n-1}(x)-M_{-n}^2(x)=(m^2(x)-k^2(x)-4)(-1)^n.
\end{align}
\end{theorem}
\begin{proof}
\begin{align*}
&{
 \begin{bmatrix}
    M_{n+1}(x) & M_{n}(x) \\
    M_{n}(x) & M_{n-1}(x) \\
  \end{bmatrix}
}=P^n{
          \begin{bmatrix}
            m(x)+k(x) & 2 \\
            2 & m(x)-k(x) \\
          \end{bmatrix}
        },\\
		&\text{Where,} \quad P={
          \begin{bmatrix}
            k(x) & 1 \\
            1& 0 \\
          \end{bmatrix}
      }.\\
&\text{And,}\quad P^n={
 \begin{bmatrix}
    F_{n+1}(x) & F_{n}(x) \\
    F_{n}(x) & F_{n-1}(x) \\
  \end{bmatrix}
}.\\
&\text{Now,}\quad P^{-n}=(P^n)^{-1}=\frac{1}{F_{n+1}(x)F_{n-1}(x)-F_{n}^2(x)}
{ \begin{bmatrix}
    F_{n-1}(x) & -F_{n}(x) \\
    -F_{n}(x) & F_{n+1}(x) \\
  \end{bmatrix}
}.\\
\end{align*}	
 \begin{align*}	
&\because\quad
F_{n+1}(x)F_{n-1}(x)-F_{n}^2(x)=(-1)^n,\\
&\therefore\quad P^{-n}=\frac{1}{(-1)^n}
{
 \begin{bmatrix}
    F_{n-1}(x) & -F_{n}(x) \\
    -F_{n}(x)& F_{n+1}(x) \\
  \end{bmatrix}
},\\
&\therefore\quad P^{-n}
{
 \begin{bmatrix}
   m(x)+k(x) & 2 \\
    2& m(x)-k(x) \\
  \end{bmatrix}
}=(P^{n})^{-1}
{
 \begin{bmatrix}
   m(x)+k(x) & 2 \\
    2& m(x)-k(x) \\
  \end{bmatrix}
},\\
&\quad
{
 \begin{bmatrix}
   M_{-n+1}(x) & M_{-n}(x) \\
    M_{-n}(x)& M_{-n-1}(x) \\
  \end{bmatrix}
}\\&=\frac{1}{(-1)^n}{
 \begin{bmatrix}
    F_{n-1}(x) & -F_{n}(x) \\
    -F_{n}(x) & F_{n+1}(x) \\
  \end{bmatrix}
}{
 \begin{bmatrix}
   m(x)+k(x) & 2 \\
    2& m(x)-k(x) \\
  \end{bmatrix}
},\\
&{ \begin{bmatrix}
   M_{-n+1}(x) & M_{-n}(x) \\
    M_{-n}(x)& M_{-n-1}(x) \\
  \end{bmatrix}
}\\&=\frac{1}{(-1)^n}{
 \begin{bmatrix}
    (m(x)+k(x))(F_{n-1}(x)-2F_{n}(x)) & 2F_{n-1}(x)-(m(x)-k(x))F_{n}(x) \\
    2F_{n+1}(x)-(m(x)+k(x))F_{n}(x) & (m(x)-k(x))(F_{n+1}(x)-2F_{n}(x)) \\
  \end{bmatrix}
},\\
&\text{Compairing both sides, we get}\\
&M_{-n+1}(x).M_{-n-1}(x)-M_{-n}^2(x)= F_{n-1}(x)F_{n+1}(x)[(m(x)+k(x))(m(x)-k(x))-4]\\
&-F_{n}^2(x)[(m(x)+k(x))(m(x)-k(x))-4],\\
&M_{-n+1}(x).M_{-n-1}(x)-M_{-n}^2(x)= [F_{n-1}(x)F_{n+1}(x)\\&-F_{n}^2(x)][(m(x)+k(x))(m(x)-k(x))-4],\\
&M_{-n+1}(x).M_{-n-1}(x)-M_{-n}^2(x)= (m^2(x)-k^2(x)-4)(-1)^n.\\
\end{align*}
\end{proof}
\begin{theorem}For arbitrary integer $n,r\geq0$, we have
\begin{align*}
	M_{n-r+1}(x)&=(-1)^r\left[ F_{r-1}(x)M_{k,n+1}-F_{k,r}M_{k,n}\right] ,\\
	M_{k,n-r}(x)&=(-1)^r\left[ F_{r-1}(x)M_{n}(x)-F_{r}(x)M_{n-1}(x)\right] ,\\
	M_{n-r-1}(x)&=(-1)^r\left[ F_{r+1}(x)M_{n-1}(x)-F_{r}(x)M_{n}(x)\right] .
\end{align*}
\end{theorem}
\begin{proof}
\begin{align*}
&\text{Since,}
P^{n-r}=P^n.P^{-r},\\
&P^{n-r}=(-1)^r{
 \begin{bmatrix}
    F_{r-1}(x) & -F_{r}(x) \\
    -F_{r}(x) & F_{r+1}(x) \\
  \end{bmatrix}
}P^n.\\
&\text{Now,}\quad P^{n-r}{
 \begin{bmatrix}
    m(x)+k(x) & 2 \\
    2 & m(x)-k(x) \\
  \end{bmatrix}
}={
 \begin{bmatrix}
 		M_{n-r+1}(x) & M_{n-r}(x) \\
    M_{n-r}(x) & M_{n-r-1}(x)\\
  \end{bmatrix}
},\\
&P^{n-r}{
 \begin{bmatrix}
    m(x)+k(x) & 2 \\
    2 & m(x)-k(x) \\
  \end{bmatrix}
}\\&=(-1)^r{
 \begin{bmatrix}
    F_{r-1}(x) & -F_{r}(x) \\
    -F_{r}(x) & F_{r+1}(x) \\
  \end{bmatrix}
}P^n{
 \begin{bmatrix}
    m(x)+k(x) & 2 \\
    2 & m(x)-k(x) \\
  \end{bmatrix}
}\\
\end{align*}	
 \begin{align*}	
&=(-1)^r{
 \begin{bmatrix}
    F_{r-1}(x) & -F_{r}(x)\\
    -F_{r}(x) & F_{r+1}(x) \\
  \end{bmatrix}
}{
 \begin{bmatrix}
 		M_{n+1}(x) & M_{n}(x) \\
    M_{n}(x) & M_{n-1}(x) \\
  \end{bmatrix}
},\\
&=(-1)^r{
 \begin{bmatrix}
 		F_{r-1}(x)M_{n+1}(x)-F_{r}(x)M_{n}(x) & F_{r-1}(x)M_{n}(x)-F_{r}(x)M_{n-1}(x) \\
    F_{r+1}(x)M_{n}(x)-F_{r}(x)M_{n+1}(x) & F_{r+1}(x)M_{n-1}(x)-F_{r}(x)M_{n}(x) \\
  \end{bmatrix}
},\\
&\text{It gives that}\\
&M_{n-r+1}(x)=(-1)^r[F_{r-1}(x)M_{n+1}(x)-F_{r}(x)M_{n}(x)],\\
&M_{n-r}(x)=(-1)^r[F_{r-1}(x)M_{n}(x)-F_{r}(x)M_{n-1}(x)],\\
&M_{n-r-1}(x)=(-1)^r[F_{r+1}(x)M_{n-1}(x)-F_{r}(x)
M_{n}(x)].
\end{align*}
\end{proof}
\begin{theorem} For $n,r\geq0$, we have
\begin{align*}
	(1-F_{r+1}(x))\sum_{j=0}^{j=n}M_{rj+1}(x)-F_{r}(x)\sum_{j=0}^{j=n}M_{rj}(x)&=(m(x)+k(x))-M_{rn+r+1}(x),\\
	(1-F_{r+1}(x))\sum_{j=0}^{j=n}M_{rj}(x)-F_{r}(x)\sum_{j=0}^{j=n}M_{rj+1}(x)&=2-M_{rn+r}(x),\\
	(1-F_{r-1}(x))\sum_{j=0}^{j=n}M_{rj-1}(x)-F_{r}(x)\sum_{j=0}^{j=n}M_{rj}(x)&=(m(x)-k(x))-M_{rn+r-1}(x).
\end{align*}
\end{theorem}
\begin{proof}
\begin{align*}
&\text{Since,}
(I-P^{r})\sum_{j=0}^{j=n}(P^r)^j=I-(P^r)^{n+1}.\\
&\text{Now,}\\
&[I-(P^r)^{n+1}]{
 \begin{bmatrix}
    m(x)+k(x) & 2 \\
    2 & m(x)-k(x) \\
  \end{bmatrix}
}= {
 \begin{bmatrix}
    m(x)+k(x) & 2 \\
    2 & m(x)-k(x)\\
  \end{bmatrix}
}\\&-P^{rn+r}{
 \begin{bmatrix}
    m(x)+k(x) & 2 \\
    2 & m(x)-k(x)\\
  \end{bmatrix}
}\\
&={
 \begin{bmatrix}
    m(x)+k(x) & 2 \\
    2 & m(x)-k(x) \\
  \end{bmatrix}
}-{
 \begin{bmatrix}
    M_{rn+r+1}(x) & M_{rn+r}(x) \\
    M_{rn+r}(x) & M_{rn+r-1}(x)\\
  \end{bmatrix}
}\\
&={
 \begin{bmatrix}
   (m(x)+k(x))M_{rn+r+1}(x) & 2-M_{rn+r}(x) \\
    2-M_{rn+r}(x) & (m(x)-k(x))M_{rn+r-1}(x)\\
  \end{bmatrix}
},\\
&\sum_{j=0}^{j=n}(P^r)^j{
 \begin{bmatrix}
    m(x)+k(x) & 2 \\
    2 & m(x)-k(x) \\
  \end{bmatrix}
}={
 \begin{bmatrix}
    \sum_{j=0}^{j=n}M_{rj+1}(x) & \sum_{j=0}^{j=n}M_{rj}(x) \\
    \sum_{j=0}^{j=n}M_{rj}(x) & \sum_{j=0}^{j=n}M_{rj-1}(x) \\
  \end{bmatrix}
},\\
&\therefore\quad
(I-P^r)\sum_{j=0}^{j=n}(P^r)^j{
 \begin{bmatrix}
    m(x)+k(x) & 2 \\
    2 & m(x)-k(x) \\
  \end{bmatrix}
}\\
\end{align*}	
 \begin{align*}	&= (I-P^r){
 \begin{bmatrix}
    \sum_{j=0}^{j=n}M_{rj+1}(x) & \sum_{j=0}^{j=n}M_{rj}(x) \\
    \sum_{j=0}^{j=n}M_{rj}(x)& \sum_{j=0}^{j=n}M_{rj-1}(x) \\
  \end{bmatrix}
}\\
&={
 \begin{bmatrix}
    (1-F_{r+1}(x)) & -F_{r}(x) \\
    -F_{r}(x) & (1-F_{r-1})(x) \\
  \end{bmatrix}
}{
 \begin{bmatrix}
    \sum_{j=0}^{j=n}M_{rj+1}(x) & \sum_{j=0}^{j=n}M_{rj}(x) \\
    \sum_{j=0}^{j=n}M_{rj}(x) & \sum_{j=0}^{j=n}M_{rj-1}(x) \\
  \end{bmatrix}
}.\\
&\text{Compairing both sides, we get}\\
&(m(x)+k(x))-M_{rn+r+1}(x)=(1-F_{r+1}(x))\sum_{j=0}^{j=n}M_{rj+1}(x)-F_{r}(x)\sum_{j=0}^{j=n}M_{rj}(x),\\
&2-M_{rn+r}(x)=(1-F_{r+1}(x))\sum_{j=0}^{j=n}M_{rj}(x)-F_{r}(x)\sum_{j=0}^{j=n}M_{rj+1}(x),\\
&(m(x)-k(x))-M_{rn+r-1}(x)=(1-F_{r-1}(x))\sum_{j=0}^{j=n}M_{rj-1}(x)-F_{r}(x)\sum_{j=0}^{j=n}M_{rj}(x).
\end{align*}
\end{proof}

\section{Generalized Fibonacci Polynomials $\widehat{F}_{ n}(x)$ and $\widehat{L}_{n}(x)$}
In this section, we define generalized Fibonacci polynomials  $\widehat{F}_{ n}(x)$ and $\widehat{L}_{n}(x)$.
\begin{definition}
The generalized Fibonacci polynomial $\widehat{F}_{ n}(x)$ is defined by the recurrence relation 
\begin{equation*}\label{b1}
\widehat{F}_{ n+1}(x)=x\widehat{F}_{n}(x)+\widehat{F}_{ n-1}(x) \quad\text{with} \quad \widehat{F}_{ 0}(x)=0\text{,}\quad \widehat{F}_{1}(x)=x^2+4\text{, for } n\geq{1}.
\end{equation*}
\end{definition}
\begin{definition}
The generalized Lucas polynomial $\widehat{L}_{ n}(x)$ is defined by the recurrence relation 
\begin{equation*}\label{b2}
\widehat{L}_{ n+1}(x)=x \widehat{L}_{ n}(x)+\widehat{L}_{ n-1}(x) \quad\text{with} \quad \widehat{L}_{ 0}(x)=2x^2+8\text{,}\quad \widehat{L}_{ 1}(x)=x^3+4x\text{, for } n\geq{1}.
\end{equation*}
\end{definition}
\noindent Characteristic equation of the initial recurrence relation (\ref{b1} and \ref{b2}) is
\begin{equation*}\label{b3}
r^{2}-xr-1=0. 
\end{equation*} 
Characteristic roots of (\ref{b3}) are
\begin{equation*}\label{b4}
 r_{1}(x)=\frac{x+\sqrt{x^2+4}}{2} \text{,}\quad r_{2}(x)=\frac{x-\sqrt{x^2+4}}{2}.
\end{equation*}
 Characteristic roots (\ref{b4}) satisfy the properties
  \begin{equation*}\label{b5} 
   r_{1}(x)-r_{2}(x) = \sqrt{x^{2}+4} = \sqrt{\Delta(x)}\text{,}
 \quad r_{1}(x)+r_{2}(x)=x\text{,}\quad r_{1}(x)r_{2}(x)=-1.
\end{equation*}   
Binet forms for both  $\widehat{F}_{n}(x)$ and $\widehat{L}_{n}(x)$ are \begin{equation}\label{b6} 
\widehat{F}_{n}(x)= {r_{1}(x)}^n - {r_{2}(x)}^n,
\end{equation}
\begin{equation}\label{b7} 
\widehat{L}_{k,n}= \left[ r_{1}(x)^2+r_{2}(x)^2+2\right]\left[ r_{1}(x)^n+r_{2}(x)^n)\right]. 
\end{equation}
The most commonly used matrix in relation to the recurrence relation(\ref{b3}) is
\begin{equation*}\label{b8} 
 M(x) = {\left[
          \begin{array}{cc}
            x& 1\\
            1 & 0 \\
          \end{array}
        \right]}.
\end{equation*}
Using principle of mathematical induction the matrix $M$ can be generalized to
\begin{align*}
 M(x)^n = {\left[
          \begin{array}{cc}
            \dfrac{\widehat{F}_{n+1}(x)}{\Delta(x)} &\dfrac{\widehat{F}_{n}(x)}{\Delta(x)} \\
            \dfrac{\widehat{F}_{n}(x)}{\Delta(x)} &\dfrac{ \widehat{F}_{n-1}(x)}{\Delta(x)} \\
          \end{array}
        \right]}.
\end{align*}
where, $n$ is an integer.

\noindent Several identities for $\widehat{F}_{n}(x) $ and $\widehat{L}_{n}(x) $ can be established using (\ref{b6}), (\ref{b7}). Some of these are listed below
\begin{align*}
\widehat{L}_{n}(x)&=\widehat{F}_{n+1}(x) +\widehat{F}_{n-1}(x),  \\\bigskip
\Delta(x) \widehat{F}_{n}(x)&=\widehat{F}_{n+1}(x) +\widehat{L}_{n-1}(x),  \\ \bigskip
\Delta(x) \widehat{F}^2_{n}(x)&=\widehat{F}_{2n}(x) -2\Delta(x)(-1)^n , \\\bigskip
\widehat{F}_{m}(x)\widehat{L}_{n}(x)&=\Delta\widehat {F}_{m+n}(x)-\Delta(x)(-1)^{m}\widehat{L}_{n-m}(x), \\\bigskip
\widehat{F}_{m}(x)\widehat{F}_{n}(x)&=\widehat{L}_{m+n}(x)-(-1)^{m}\widehat{L}_{n-m}(x), \\\bigskip
(-1)^{n-m+1}\widehat{F}_{m}(x)^2&=\widehat{F}_{m+n}(x)\widehat{F}_{n-m}(x)-\widehat{F}_{n}(x)^{2},   \\\bigskip
(-1)^{n-m}\widehat{F}_{m}(x)^2&=\widehat{L}_{m+n}(x)\widehat{L}_{n-m}(x)-\widehat{L}_{n}(x)^{2},  \\\bigskip
\widehat{F}_{m}(x)\widehat{F}_{n+r+m}(x)&=\widehat{F}_{m+n}(x)\widehat{F}_{r+m}(x)-(-1)^{m}\widehat{F}_{n}(x)\widehat{F}_{r}(x),  \\\bigskip
2\Delta(x) \widehat{L}_{(m+1)n}(x)&=\widehat{L}_{mn}(x)\widehat{L}_{n}(x)+\Delta(x) \widehat{F}_{mn}(x)\widehat{F}_{n}(x,)\\\bigskip
2\Delta(x) \widehat{F}_{(m+1)n}(x)&=\widehat{F}_{mn}(x)\widehat{L}_{n}(x)+\widehat{L}_{mn}(x)\widehat{F}_{n}(x),\\\bigskip
\Delta(x) \widehat{F}_{2n+m}(x)\widehat{F}_{m}(x)&=\widehat{L}_{m+n}(x)^2+(-1)^{m-1}\widehat{L}_{n}(x)^2, \\\bigskip
\Delta(x) \widehat{F}_{2n}(x)\widehat{F}_{m}(x)&=\widehat{L}_{m+n}(x)\widehat{L}_{n}(x)+(-1)^{m+1}\widehat{L}_{n-m}(x)\widehat{L}_{n}(x),  \\\bigskip
\widehat{F}_{2(r+1)n+m}(x)\widehat{F}_{m}(x)&=\widehat{F}_{m+2rn}(x)\widehat{F}_{2n+m}(x)+(-1)^{m+1}\widehat{F}_{2rn}(x)\widehat{F}_{2n}(x), \\\bigskip
\widehat{F}_{-n}(x)&=(-1)^{n+1}\widehat{F}_{n}(x),\\\bigskip
\widehat{F}_{-n-1}(x)&=(-1)^{n}\widehat{F}_{n+1}(x).
\end{align*}
\subsection{Properties of Generalized Fibonacci Polynomials $\widehat{F}_{ n}(x)$ and $\widehat{L}_{n}(x)$ using Matrix Methods}
In this section, we derive some properties of generalized Fibonacci polynomials  $\widehat{F}_{ n}(x)$ and $\widehat{L}_{n}(x)$ using matrix methods.
\begin{lemma} \label{lemma3}
If $X$ is a square matrix with $\Delta(x) X^2=xX+I$, then $ \Delta(x) X^{n}=\widehat{F}_{n}(x)X+\widehat{F}_{n-1}(x)I$, for all $n\in Z$.
\end{lemma}
\begin{proof}
For $n=0$ the result is true. \\
For $n=1$ 
\begin{align*}
\Delta(x)(X)^1&=\widehat{F}_{1}(x)X+\widehat{F}_{0}(x)I\\
		 &=\Delta(x) X+0I\\
		 &=\Delta(x) X.
\end{align*}		
Hence result is true for $n=1$. \\
Assume that, $ \Delta(x) X^{n}=\widehat{F}_{n}(x)X+\widehat{F}_{n-1}(x)I$, and prove that, $ \Delta(x) X^{n+1}=\widehat{F}_{n+1}(x)X+\widehat{F}_{n}(x)I$.\\
Consider
\begin{align*}
\widehat{F}_{n+1}(x)X+\widehat{F}_{n}(x)I&=(\widehat{F}_{n}(x)x+\widehat{F}_{n-1}(x)I)X+\widehat{F}_{n}(x)I\\
									 &=(xX+I)\widehat{F}_{n}(x)+X\widehat{F}_{n-1}(x)\\
									 &=X^2\widehat{F}_{n}(x)+X\widehat{F}_{n-1}(x)\\
									& =X(X\widehat{F}_{n}(x)+\widehat{F}_{n-1}(x))\\
									 &=X(\Delta(x) X^{n})\\
									& =\Delta(x) X^{n+1}\\\text{Hence,}\quad  \Delta(x) X^{n+1}&=\widehat{F}_{n+1}(x)X+\widehat{F}_{n}(x)I.
\end{align*}		
\noindent We now show that $ \Delta(x) X^{-n}=\widehat{F}_{-n}(x)X+\widehat{F}_{-n-1}(x)I$, for all $n\in Z^+$.\\
Let $Y=xI-X$, then
\begin{align*}
								Y^2&=(xI-X)^2\\
									 &=x^2I-2xX+X^2\\
									 &=x^2I-2xX+xX+I\\
									 &=x^2I-xX+I\\
									 &=x(xI-X)+X+I\\
									 &=xY+I.\\
\text{This shows that}\quad\Delta Y^n&=\widehat{F}_{n}(x)Y+\widehat{F}_{n-1}(x)I,\\
		\Delta(x)(-X^{-1})^n&=\widehat{F}_{n}(x)(xI-X)+\widehat{F}_{n-1}(x)I,\\
			 \Delta(x)(-1)^nX^{-n}&=-\widehat{F}_{n}(x)X+\widehat{F}_{n+1}(x)I,\\
						 \Delta(x) X^{-n}&=(-1)^{n+1}\widehat{F}_{n}(x)X+(-1)^n\widehat{F}_{n+1}(x)I.\\
\end{align*}	
It gives that $ \Delta(x) X^{-n}=\widehat{F}_{-n}(x)X+\widehat{F}_{-n-1}(x)I$, for all $n\in Z^+$.
\end{proof}
\begin{corollary}\label{b}
Let $S(x)={\left[
 \begin{array}{cc}
    \dfrac{x}{2} & \dfrac{\Delta(x)}{2} \\
    \dfrac{1}{2}& \dfrac{x}{2} \\
	\end{array}
	\right]}$, then  $\Delta(x) S(x)^n={\left[
 \begin{array}{cc}
    \dfrac{\widehat{L}_{n}(x)}{2} & \dfrac{\Delta\widehat{F}_{n}(x)}{2} \\
    \dfrac{\widehat{F}_{n}(x)}{2} & \dfrac{\widehat{L}_{n}(x)}{2} \\
	\end{array}
	\right]}$, for every $n\in Z$.	
\end{corollary}
\begin{proof}
\begin{align*}
&\text{Since}\quad S(x)^2={\left[
 \begin{array}{cc}
    \frac{x^2+2}{2} & \frac{x\Delta(x)}{2} \\
    \frac{x}{2}& \frac{x^2+2}{2} \\
	\end{array}
	\right]}\\&=x S(x)+I.
	\end{align*}
	Using Lemma (\ref{lemma3}), it is clear that
	\begin{align*}
	&\Delta(x) S(x)^n=\widehat{F}_{n}(x)S(x)+\widehat{F}_{n-1}(x)I\\
	&= {\left[
 \begin{array}{cc}
    \frac{x\widehat{F}_{n}(x)}{2} & \frac{\Delta(x)\widehat{F}_{n}(x)}{2} \\
    \frac{\widehat{F}_{n}(x)}{2}& \frac{x\widehat{F}_{n}(x)}{2} \\
	\end{array}
	\right]}+{\left[
 \begin{array}{cc}
    \widehat{F}_{n-1}(x) & 0 \\
    0& \widehat{F}_{n-1}(x) \\
	\end{array}
	\right]}\\
	&={\left[
 \begin{array}{cc}
    \frac{\widehat{L}_{n}(x)}{2} & \frac{\Delta(x)\widehat{F}_{n}(x)}{2} \\
    \frac{\widehat{F}_{n}(x)}{2} & \frac{\widehat{L}_{n}(x)}{2} \\
	\end{array}
	\right]}.
\end{align*}
\end{proof}
\begin{lemma}\label{25}
\begin{align*} \widehat{L}^2_{n}(x)-\Delta(x) \widehat{F}^2_{n}(x)=4\Delta(x)^2(-1)^n,\qquad \text{for all} \quad n\in Z.
\end{align*}
\end{lemma}
\begin{proof}
\begin{align*}
&\text{Since,} \qquad \det(S(x))= -1,\\
&\det(S(x)^n)=(-1)^n.\\
&\text{Moreover since,} \quad\Delta(x) S(x)^n={\left[
 \begin{array}{cc}
    \frac{\widehat{L}_{n}(x)}{2} & \frac{\Delta\widehat{F}_{n}(x)}{2} \\
    \frac{\widehat{F}_{n}(x)}{2} & \frac{\widehat{L}_{n}(x)}{2} \\
	\end{array}
	\right]}.\\
&\text{We get,}\quad	\det(\Delta(x) S(x)^n)=\frac{\widehat{L}^2_{n}(x)}{4}-\frac{\Delta(x) \widehat{F}^2_{n}(x)}{4}.\\
&\text{Thus it follows that,}\quad
 \widehat{L}^2_{n}(x)-\Delta(x) \widehat{F}^2_{n}(x)=4\Delta(x)^2(-1)^n,\qquad \text{for all,} \quad n\in Z.
\end{align*}
\end{proof}
\begin{lemma}
\begin{align*}
2\Delta(x) \widehat{L}_{n+m}(x)&= \widehat{L}_{n}(x) \widehat{L}_{m}(x)+\Delta(x)  \widehat{F}_{n}(x) \widehat{F}_{m}(x),\\
2\Delta(x) \widehat{F}_{n+m}(x)&= \widehat{F}_{n}(x) \widehat{L}_{m}(x)+ \widehat{L}_{n}(x) \widehat{F}_{m}(x),\quad \text{for all}\quad n, m\in Z.
\end{align*}
\end{lemma}
\begin{proof}
\begin{align*}
&\text{Since,} \quad \Delta(x)^2 S(x)^{n+m}=\Delta(x) S(x)^n.\Delta(x) S(x)^m\\
&={\left[
\begin{array}{cc}
    \frac{ \widehat{L}_{n}(x)}{2} & \frac{\Delta(x)  \widehat{F}_{n}(x)}{2} \\
    \frac{ \widehat{F}_{n}(x)}{2} & \frac{ \widehat{L}_{n}(x)}{2} \\
	\end{array}
	\right]}.{\left[
 \begin{array}{cc}
    \frac{ \widehat{L}_{m}(x)}{2} & \frac{\Delta(x)  \widehat{F}_{m}(x)}{2} \\
    \frac{ \widehat{F}_{m}(x)}{2} & \frac{ \widehat{L}_{m}(x)}{2} \\
	\end{array}
	\right]}\\
	&={\left[
 \begin{array}{cc}
    \frac{ \widehat{L}_{n}(x) \widehat{L}_{m}(x)+\Delta(x)  \widehat{F}_{n}(x) \widehat{F}_{m}(x)}{4} & \frac{\Delta(x)[ \widehat{L}_{n}(x) \widehat{F}_{m}(x)+ \widehat{F}_{n}(x) \widehat{L}_{m}(x)}{4} \\
    \frac{ \widehat{L}_{n}(x) \widehat{F}_{m}(x)+ \widehat{F}_{n}(x) \widehat{L}_{m}(x)}{4} & \frac{ \widehat{L}_{n}(x) \widehat{L}_{m}(x)+\Delta(x)  \widehat{F}_{n}(x) \widehat{F}_{m}(x)}{4} \\
	\end{array}
	\right]}.\\
&\text{But,} \quad 	\Delta(x) S(x)^{n+m}={\left[
 \begin{array}{cc}
    \frac{\widehat{L}_{n+m}(x)}{2} & \frac{\Delta(x) \widehat{F}_{n+m}(x)}{2} \\
    \frac{\widehat{F}_{n+m}(x)}{2} & \frac{\widehat{L}_{n+m}(x)}{2} \\
	\end{array}
	\right]}.\\
	&\text{It gives that} \quad 	2\Delta(x) \widehat{L}_{n+m}(x)= \widehat{L}_{n}(x) \widehat{L}_{m}(x)+\Delta(x)  \widehat{F}_{n}(x) \widehat{F}_{m}(x),\quad \text{for all}\quad n, m\in Z.\\
&2\Delta(x) \widehat{F}_{n+m}(x)= \widehat{F}_{n}(x) \widehat{L}_{m}(x)+ \widehat{L}_{n}(x) \widehat{F}_{m}(x),\quad \text{for all}\quad n, m\in Z.
\end{align*}		
\end{proof}
\begin{lemma}
\begin{align*}
&2(-1)^m\Delta(x)^2 \widehat{L}_{n-m}(x)=\widehat{L}_{n}(x)\widehat{L}_{m}(x)-\Delta(x) \widehat{F}_{n}(x)\widehat{F}_{m}(x),\quad \text{for all}\quad n, m\in Z,\\&
2(-1)^m\Delta(x)^2 \widehat{F}_{n-m}(x)=\widehat{F}_{n}(x)\widehat{L}_{m}(x)-\widehat{L}_{n}(x),\widehat{F}_{m}(x)\quad \text{for all}\quad n, m\in Z.
\end{align*}
\end{lemma}
\begin{proof}
\begin{align*}
&\text{Since}\qquad
\Delta(x)^2 S(x)^{n-m}=\Delta(x) S(x)^n.\Delta(x) S(x)^{-m}\\
			 &=\Delta(x) S(x)^n.\Delta(x)[S(x)^{m}]^{-1}\\
			 &=\Delta(x) S(x)^n.(-1)^m{\left[
 \begin{array}{cc}
    \frac{\widehat{L}_{m}(x)}{2} & \frac{-\Delta(x) \widehat{F}_{m}(x)}{2} \\
    \frac{-\widehat{F}_{m}(x)}{2} & \frac{\widehat{L}_{m}(x)}{2} \\
	\end{array}
	\right]}\\
	&=(-1)^m{\left[
 \begin{array}{cc}
    \frac{\widehat{L}_{n}(x)}{2} & \frac{\Delta(x) \widehat{F}_{n}(x)}{2} \\
    \frac{\widehat{F}_{n}(x)}{2} & \frac{\widehat{L}_{n}(x)}{2} \\
	\end{array}
	\right]}.{\left[
 \begin{array}{cc}
    \frac{\widehat{L}_{m}(x)}{2} & \frac{-\Delta(x) \widehat{F}_{m}(x)}{2} \\
    \frac{-\widehat{F}_{m}(x)}{2} & \frac{\widehat{L}_{m}(x)}{2} \\
	\end{array}
	\right]}\\								
	&={\left[
 \begin{array}{cc}
    \frac{\widehat{L}_{n}(x)\widehat{L}_{m}(x)-\Delta \widehat{F}_{n}(x)\widehat{F}_{m}(x)}{4} & \frac{\Delta(x)[\widehat{L}_{n}(x)\widehat{F}_{m}(x)-\widehat{F}_{n}(x)\widehat{L}_{m}(x)}{4} \\
    \frac{\widehat{L}_{n}(x)\widehat{F}_{m}(x)-\widehat{F}_{n}(x)\widehat{L}_{m}(x)}{4} & \frac{\widehat{L}_{n}(x)\widehat{L}_{m}(x)-\Delta(x) \widehat{F}_{n}(x)\widehat{F}_{m}(x)}{4} \\
	\end{array}
	\right]}.\\
	&\text{But}\qquad
\Delta(x)^2S(x)^{n-m}={\left[
 \begin{array}{cc}
    \frac{L_{n-m}(x)}{2} & \frac{\Delta(x) F_{n-m}(x)}{2} \\
    \frac{F_{n-m}(x)}{2} & \frac{L_{n-m}(x)}{2} \\
	\end{array}
	\right]}.\\
	&\text{It gives that}\qquad
	2(-1)^m\Delta(x)^2 \widehat{L}_{n-m}(x)=\widehat{L}_{n}(x)\widehat{L}_{m}(x)-\Delta(x) \widehat{F}_{n}(x)\widehat{F}_{m}(x),\\
&2(-1)^m\Delta(x)^2 \widehat{F}_{n-m}(x)=\widehat{F}_{n}(x)\widehat{L}_{m}(x)-\widehat{L}_{n}(x)\widehat{F}_{m}(x),\quad \text{for all}\quad n, m\in Z.
	\end{align*}		
\end{proof}
\begin{lemma}
\begin{align*}
(-1)^m\Delta(x) \widehat{L}_{n-m}(x)+\Delta (x)\widehat{L}_{n+m}(x)=\widehat{L}_{n}(x)\widehat{L}_{m}(x),\\
(-1)^m\Delta(x) \widehat{F}_{n-m}(x)+\Delta(x) \widehat{F}_{n+m}(x)=\widehat{F}_{n}(x)\widehat{L}_{m}(x).
\end{align*}
\end{lemma}
\begin{proof} Consider
\begin{align*}
&\Delta(x)^2S(x)^{n+m}+(-1)^m\Delta(x)^2S(x)^{n-m}\\&={\left[
 \begin{array}{cc}
    \dfrac{\Delta(x) \widehat{L}_{n+m}(x)+(-1)^m\Delta(x) \widehat{L}_{n-m}(x)}{2} & \dfrac{\Delta(x)[\widehat{F}_{n+m}(x)+(-1)^m\widehat{F}_{n-m}(x)}{2} \\
    \dfrac{\Delta(x) \widehat{F}_{n+m}(x)+(-1)^m\Delta(x) \widehat{F}_{n-m}(x)}{2} & \dfrac{\Delta(x) \widehat{L}_{n+m}(x)+(-1)^m\Delta(x) \widehat{L}_{n-m}(x)}{2} \\
	\end{array}
	\right]}.
\end{align*}
	On the other hand
	\begin{align*}
&\Delta(x)^2 S(x)^{n+m}+(-1)^m\Delta(x)^2 S(x)^{n-m}=\Delta(x) S(x)^n\Delta(x) S(x)^m+(-1)^m\Delta(x) S(x)^n\Delta(x) S(x)^{-m}\\
&=\Delta(x) S(x)^n[\Delta(x) S(x)^m+(-1)^m\Delta(x) S(x)^{-m}]
\end{align*}				
 \begin{align*}
&={\left[
 \begin{array}{cc}
    \frac{\widehat{L}_{n}}{2} & \frac{\Delta(x) \widehat{F}_{n}(x)}{2} \\
    \frac{\widehat{F}_{n}(x)}{2} & \frac{\widehat{L}_{n}(x)}{2} \\
	\end{array}
	\right]}\left\lbrace {\left[
 \begin{array}{cc}
    \frac{\widehat{L}_{m}(x)}{2} & \frac{\Delta(x) \widehat{F}_{m}(x)}{2} \\
    \frac{\widehat{F}_{m}(x)}{2} & \frac{\widehat{L}_{m}(x)}{2} \\
	\end{array}
	\right]}+{\left[
 \begin{array}{cc}
    \frac{\widehat{L}_{m}(x)}{2} & \frac{-\Delta(x) \widehat{F}_{m}(x)}{2} \\
    \frac{-\widehat{F}_{m}}{2} & \frac{\widehat{L}_{m}(x)}{2} \\
	\end{array}
	\right]}\right\rbrace \\			
					&={\left[
 \begin{array}{cc}
    \frac{\widehat{L}_{n}(x)}{2} & \frac{\Delta(x) \widehat{F}_{n}(x)}{2} \\
    \frac{\widehat{F}_{n}(x)}{2} & \frac{\widehat{L}_{n}(x)}{2} \\
	\end{array}
	\right]}.{\left[
 \begin{array}{cc}
    \widehat{L}_{m}(x) & 0 \\
    0 & \widehat{L}_{m}(x) \\
		\end{array}
	\right]}\\
				&={\left[
 \begin{array}{cc}
    \frac{\widehat{L}_{m}(x)\widehat{L}_{n}(x)}{2} & \frac{\Delta(x) \widehat{F}_{n}(x)\widehat{L}_{m}(x)}{2} \\
    \frac{\widehat{F}_{n}(x)\widehat{L}_{m}(x)}{2} & \frac{\widehat{L}_{m}(x)\widehat{L}_{n}(x)}{2}\\
	\end{array}
	\right]}.
	\end{align*}
		\begin{align*}
		\text{It gives that}\quad
(-1)^m\Delta(x) \widehat{L}_{n-m}(x)+\Delta(x) \widehat{L}_{n+m}(x)&=\widehat{L}_{n}(x)\widehat{L}_{m}(x),\\
(-1)^m\Delta(x) \widehat{F}_{n-m}(x)+\Delta(x) \widehat{F}_{n+m}(x)&=\widehat{F}_{n}(x)\widehat{L}_{m}(x).
\end{align*}
\end{proof}
\begin{lemma}
\begin{align*}
&8\Delta(x)^2 \widehat{F}_{t+y+z}(x)=\widehat{L}_{t}(x)\widehat{L}_{y}(x)\widehat{F}_{z}(x)+\widehat{F}_{t}(x)\widehat{L}_{y}(x)\widehat{L}_{z}(x)+\widehat{L}_{t}(x)\widehat{F}_{y}(x)\widehat{L}_{z}(x)+\Delta(x) \widehat{F}_{t}(x)\widehat{F}_{y}(x)\widehat{F}_{z}(x),\\
&8\Delta(x)^2 \widehat{L}_{t+y+z}(x)=\widehat{L}_{t}(x)\widehat{L}_{y}(x)\widehat{L}_{z}(x)+\Delta(x) \left[ \widehat{L}_{t}(x)\widehat{F}_{y}(x)\widehat{F}_{z}(x)+\widehat{F}_{t}(x)\widehat{L}_{y}(x)\widehat{F}_{z}(x)+\widehat{F}_{t}(x)\widehat{F}_{y}(x)\widehat{L}_{z}(x)\right]. 
\end{align*}
\end{lemma}
\begin{proof}
Since
	\begin{align*}
	\Delta(x)^2 S(x)^{t+y+z}&={\left[
 \begin{array}{cc}
    \frac{\Delta(x) L_{t+y+z}(x)}{2} & \frac{\Delta(x)^2 F_{t+y+z}(x)}{2} \\
    \frac{\Delta(x) F_{t+y+z}(x)}{2}& \frac{\Delta(x) L_{t+y+z}(x)}{2} \\
	\end{array}
	\right]}.
	\end{align*}
	On the other hand
	\begin{align*}
	\Delta(x)^2 S(x)^{t+y+z}&=\Delta(x) S(x)^{t+y}\Delta(x) S(x)^z\\
 &={\left[
 \begin{array}{cc}
    \frac{\widehat{L}_{t+y}(x)}{2} & \frac{\Delta(x) \widehat{F}_{t+y}(x)}{2} \\
    \frac{\widehat{F}_{t+y}(x)}{2} & \frac{\widehat{L}_{t+y}(x)}{2} \\
	\end{array}
	\right]}.{\left[
 \begin{array}{cc}
    \frac{\widehat{L}_{z}(x)}{2} & \frac{\Delta(x) \widehat{F}_{z}(x)}{2} \\
    \frac{\widehat{F}_{z}(x)}{2} & \frac{\widehat{L}_{z}(x)}{2} \\
	\end{array}
	\right]}\\
 &={\left[
 \begin{array}{cc}
    \frac{\widehat{L}_{t+y}(x)\widehat{L}_{z}(x)+\Delta(x) \widehat{F}_{t+y}(x)\widehat{F}_{z}(x)}{4} & \frac{\Delta(x)[\widehat{L}_{t+y}(x)\widehat{F}_{z}(x)+\widehat{F}_{t+y}(x)\widehat{L}_{z}(x)}{4} \\
    \frac{\widehat{L}_{z}(x)\widehat{F}_{t+y}(x)+\widehat{F}_{z}(x)\widehat{L}_{t+y}(x)}{4} & \frac{\widehat{L}_{t+y}(x)\widehat{L}_{z}(x)+\Delta(x) \widehat{F}_{t+y}(x)\widehat{F}_{z}(x)}{4} \\
	\end{array}
	\right]}.
	\end{align*}
	After simplifications, it gives that
	\begin{align*}
		&8\Delta(x)^2 \widehat{F}_{t+y+z}(x)=\widehat{L}_{t}(x)\widehat{L}_{y}(x)\widehat{F}_{z}(x)+\widehat{F}_{t}(x)\widehat{L}_{y}(x)\widehat{L}_{z}(x)+\widehat{L}_{t}(x)\widehat{F}_{y}(x)\widehat{L}_{z}(x)+\Delta(x) \widehat{F}_{t}(x)\widehat{F}_{y}(x)\widehat{F}_{z}(x),\\
&8\Delta(x)^2 \widehat{L}_{t+y+z}(x)=\widehat{L}_{t}(x)\widehat{L}_{y}(x)\widehat{L}_{z}(x)+\Delta(x) \left[ \widehat{L}_{t}(x)\widehat{F}_{y}(x)\widehat{F}_{z}(x)+\widehat{F}_{t}(x)\widehat{L}_{y}(x)\widehat{F}_{z}(x)+\widehat{F}_{t}(x)\widehat{F}_{y}(x)\widehat{L}_{z}(x)\right]. 
\end{align*}
\end{proof}
\begin{theorem}
\begin{small}
\begin{align*}
&\widehat{L}^2_{t+y}(x)-(-1)^{t+y+1}\widehat{F}_{z-t}(x)\widehat{L}_{t+y}(x)\widehat{F}_{y+z}(x)-\Delta(x)(-1)^{t+z}(x)\widehat{F}^2_{y+z}(x)=(-1)^{y+z}\widehat{L}^2_{z-t}(x),\\
&\Delta(x) \widehat{L}^2_{t+y}(x)-(-1)^{x+z}\widehat{L}_{z-t}(x)\widehat{L}_{t+y}(x)\widehat{L}_{y+z}(x)+(-1)^{x+z}\Delta (x)\widehat{L}^2_{y+z}(x)=(-1)^{y+z+1}\Delta(x)^2 \widehat{F}^2_{z-t}(x),\\
&\Delta (x) \widehat{F}^2_{t+y}(x)-\widehat{L}_{t-z}(x)\widehat{F}_{t+y}(x)\widehat{F}_{y+z}(x)+\Delta(x) (-1)^{x+z}\widehat{F}^2_{y+z}(x)=(-1)^{y+z}\Delta(x) \widehat{F}^2_{z-t},
\end{align*}
for all $ t$, $y$, $z\in Z$, $t\neq z$.
\end{small}
\end{theorem}
\begin{proof}
\begin{align*}
&\text{Consider the matrix multiplication }\\
&{\left[
 \begin{array}{cc}
    \frac{\widehat{L}_{t}(x)}{2} & \frac{\Delta(x) \widehat{F}_{t}(x)}{2} \\
    \frac{\widehat{F}_{z}(x)}{2} & \frac{\widehat{L}_{z}(x)}{2} \\
	\end{array}
	\right]}{\left[
 \begin{array}{c}
    \widehat{L}_{y}(x) \\
    \widehat{F}_{y}(x) \\
	\end{array}
	\right]}={\left[
 \begin{array}{c}
    \Delta(x) \widehat{L}_{t+y}(x) \\
    \Delta(x) \widehat{F}_{y+z}(x) \\
	\end{array}
	\right]}\\
		&\det{\left[
 \begin{array}{cc}
    \frac{\widehat{L}_{t}(x)}{2} & \frac{\Delta(x) \widehat{F}_{t}(x)}{2} \\
    \frac{\widehat{F}_{z}(x)}{2} & \frac{\widehat{L}_{z}(x)}{2} \\
	\end{array}
	\right]}=\frac{\widehat{L}_{t}(x)\widehat{L}_{z}(x)-\Delta(x) \widehat{F}_{t}\widehat{F}_{z}(x)}{4}\\
					&=\frac{(-1)^t\Delta(x) \widehat{L}_{z-t}(x)}{2}\\
					&=R\\
					&\neq  0.\\
		&\text{Hence}\quad
{\left[
 \begin{array}{c}
    \widehat{L}_{y}(x) \\
    \widehat{F}_{y}(x) \\
	\end{array}
	\right]}={\left[
 \begin{array}{cc}
    \frac{\widehat{L}_{t}(x)}{2} & \frac{\Delta(x) \widehat{F}_{t}(x)}{2} \\
    \frac{\widehat{F}_{z}(x)}{2} & \frac{\widehat{L}_{z}(x)}{2} \\
	\end{array}
	\right]^{-1}}.{\left[
 \begin{array}{c}
    \Delta(x) \widehat{L}_{t+y}(x) \\
    \Delta(x) \widehat{F}_{y+z}(x) \\
	\end{array}
	\right]}\\
	&=\frac{1}{R}{\left[
 \begin{array}{cc}
    \frac{\widehat{L}_{z}(x)}{2} & \frac{-\Delta(x) \widehat{F}_{t}(x)}{2} \\
    \frac{-\widehat{F}_{z}(x)}{2} & \frac{\widehat{L}_{t}(x)}{2} \\
	\end{array}
	\right]}{\left[
 \begin{array}{c}
    \Delta(x) \widehat{L}_{t+y}(x) \\
    \Delta(x) \widehat{F}_{y+z}(x) \\
	\end{array}
	\right]},\\	
			&\widehat{L}_{y}(x)=\frac{(-1)^t[\widehat{L}_{z}(x)\widehat{L}_{t+y}(x)-\Delta(x) \widehat{F}_{t}\widehat{F}_{y+z}(x)]}{\widehat{L}_{z-t}(x)},\\
	&\widehat{F}_{y}(x)=\frac{(-1)^t[\widehat{L}_{t}(x)\widehat{F}_{z+y}(x)-\widehat{F}_{z}(x)\widehat{L}_{y+t}(x)]}{\widehat{L}_{z-t}(x)}\\
	&\text{Since}
\widehat{L}^2_{y}(x)-\Delta(x) \widehat{F}^2_{y}(x)=4\Delta(x)^2(-1)^y.
\end{align*}
We get
\begin{align*}
&[\widehat{L}_{z}(x)\widehat{L}_{t+y}(x)-\Delta(x) \widehat{F}_{t}(x)\widehat{F}_{y+z}(x)]^2-\Delta(x)[\widehat{L}_{t}(x)\widehat{F}_{z+y}(x)-\widehat{F}_{z}(x)\widehat{L}_{y+t}(x)]^2=4(-1)^y\Delta(x)^2 \widehat{L}^2_{z-t}.
\end{align*}
\noindent
After simplifications, we get
\begin{align*}
&(\widehat{L}^2_{z}(x)\widehat{L}^2_{t+y}(x)-2\Delta(x) \widehat{L}_{z}(x)\widehat{F}_{t+y}(x)\widehat{F}_{y+z}(x)+\Delta(x)^2\widehat{F}^2_{t}(x)\widehat{F}^2_{y+z}(x))-\\&\Delta(x)(\widehat{L}^2_{t}(x)\widehat{F}^2_{y+z}(x)-2\widehat{L}_{t}(x)\widehat{F}_{z}(x)\widehat{F}_{y+z}(x)\widehat{L}_{t+y}(x)+\widehat{F}^2_{z}(x)\widehat{L}^2_{t+y}(x))\\&=4(-1)^y\Delta(x)^2 L^2_{z-t}(x).
\end{align*}
It gives that
\begin{align*}
&\widehat{L}^2_{t+y}(x)-(-1)^{t+y+1}\widehat{F}_{z-t}(x)\widehat{L}_{t+y}(x)\widehat{F}_{y+z}(x)-\Delta(x)(-1)^{t+z}\widehat{F}^2_{y+z}(x)=(-1)^{y+z}\widehat{L}^2_{z-t}(x),\\&\text{for all}\quad t, y, z\in Z.
\end{align*}
\end{proof}
\begin{theorem}
For $n\in N$ and $m, t\in Z$ with $m\neq 0$, we have
\begin{align*}
&\sum_{j=0}^{j=n}\widehat{L}_{mj+t}(x)=\dfrac{\Delta (x)\widehat{L}_{t}(x)-\Delta(x)\widehat{L}_{mn+m+t}(x)+(-1)^m\Delta(x)(\widehat{L}_{mn+t}(x)-\widehat{L}_{t-m}(x))}{\Delta(x)+(-1)^m\Delta(x)-\widehat{L}_{m}(x)},\\
&\sum_{j=0}^{j=n}\widehat{F}_{mj+t}(x)=\dfrac{\Delta(x) \widehat{F}_{t}(x)-\Delta(x) \widehat{F}_{mn+m+t}(x)+(-1)^m\Delta(x) (\widehat{F}_{mn+t}(x)-\widehat{F}_{t-m}(x))}{\Delta(x)+(-1)^m\Delta(x)-\widehat{L}_{m}(x)}.
\end{align*}
\end{theorem}
\begin{proof}
\begin{align*}
&\text{Since}\quad
\Delta(x)^2 I-\Delta(x)^2(S(x)^m)^{n+1}=(\Delta(x) I-\Delta(x) S(x)^m)\sum_{j=0}^{j=n}\Delta(x) (S(x)^m)^j,\\
&\det(\Delta(x) I-\Delta(x) S(x)^m)\neq 0. 
\end{align*}
Therefore, we get
\begin{align*}
&\left( \Delta(x) I-\Delta(x)(S(x)^m)\right) ^{-1}(\Delta(x) I-\Delta(x)(S(x)^m)^{n+1})\Delta(x) S(x)^t\\&=\sum_{j=0}^{j=n}(\Delta(x) S(x)^{mj+t})={\left[
 \begin{array}{cc}
    \dfrac{1}{2}{\displaystyle\sum_{j=0}^{j=n}(\widehat{L}_{mj+t}(x))} & \dfrac{\Delta(x)}{2}{\displaystyle\sum_{j=0}^{j=n}(\widehat{F}_{mj+t}(x))}\\
    \dfrac{1}{2}{\displaystyle\sum_{j=0}^{j=n}(\widehat{F}_{mj+t}(x))} & \dfrac{1}{2}{\displaystyle\sum_{j=0}^{j=n}(\widehat{L}_{mj+t}(x))}
	\end{array}
	\right]}.
\end{align*}
For $m\neq 0$, take $ D=\Delta(x)+(-1)^m \Delta(x)-\widehat{L}_{m}(x)$, then we can write
\begin{align*}
&(\Delta(x) I-\Delta(x) S(x)^m)^{-1}=\dfrac{1}{D}{\left[
 \begin{array}{cc}
   \Delta(x) -\dfrac{(\widehat{L}_{m}(x))}{2} & \Delta(x)\dfrac{(\widehat{F}_{m}(x))}{2}\\
    \dfrac{(\widehat{F}_{m}(x))}{2} & \Delta(x) -\dfrac{(\widehat{L}_{m}(x))}{2}
	\end{array}
	\right]}\\
	&=\dfrac{1}{D}\left[(\Delta(x) -\dfrac{(\widehat{L}_{m}(x))}{2})I+\dfrac{(\widehat{F}_{m}(x))}{2}R\right].
\end{align*}
It gives that
\begin{align*}
&(\Delta(x) I-\Delta(x) S(x)^m)^{-1}(\Delta(x)^2 S(x)^t-\Delta(x)^2 S(x)^{mn+m+t})\\&=\frac{1}{D}\left[(\Delta(x) -\frac{(\widehat{L}_{m}(x))}{2})I+\frac{(\widehat{F}_{m}(x))}{2}R\right](\Delta(x)^2 S(x)^t-\Delta(x)^2 S(x)^{mn+m+t}),
\end{align*}
\begin{align*}
&(\Delta(x) I-\Delta(x) S(x)^m)^{-1}(\Delta(x)^2 S(x)^t-\Delta(x)^2 S(x)^{mn+m+t})\\&=\frac{1}{D}\left[(\Delta(x) -\frac{(\widehat{L}_{m}(x))}{2})(\Delta(x)^2 S(x)^t-\Delta(x)^2 S(x)^{mn+m+t})+\frac{(\widehat{F}_{m}(x))}{2}R(\Delta(x)^2 S(x)^t-\Delta(x)^2 S(x)^{mn+m+t})\right].
\end{align*}
After simplifications, we get
	\begin{align*}
&R(\Delta(x)^2 S(x)^t-\Delta(x)^2 S(x)^{mn+m+t})\\&=\Delta(x) {\left[
 \begin{array}{cc}
  \Delta(x)\frac{(\widehat{F}_{t}(x)-\widehat{F}_{mn+m+t}(x))}{2} &  \Delta(x)\frac{(\widehat{L}_{t}(x)-\widehat{L}_{mn+m+t}(x))}{2}\\
    \frac{(\widehat{L}_{t}(x)-\widehat{L}_{mn+m+t}(x))}{2} & \Delta(x)\frac{(\widehat{F}_{t}(x)-\widehat{F}_{mn+m+t}(x))}{2}
	\end{array}
	\right]},
\end{align*}
\begin{align*}
&(\Delta(x) I-\Delta(x) S(x)^m)^{-1}(\Delta(x)^2 S(x)^t-\Delta(x)^2 S(x)^{mn+m+t})\\&=\dfrac{\Delta(x)}{D}(\Delta(x) -\dfrac{\widehat{L}_{m}(x)}{2})
\left[\begin{array}{cc}
  \Delta(x)\frac{(\widehat{L}_{t}(x)-\widehat{L}_{mn+m+t}(x))}{2} &  \Delta(x)\frac{(\widehat{F}_{t}(x)-\widehat{F}_{mn+m+t}(x))}{2}\\
    \dfrac{(\widehat{F}_{t}(x)-\widehat{F}_{mn+m+t}(x))}{2} & \dfrac{(\widehat{L}_{t}(x)-\widehat{L}_{mn+m+t}(x))}{2}
	\end{array}
	\right]	\\&+\dfrac{\Delta(x) \widehat{F}_{m}(x)}{2D}\left[
 \begin{array}{cc}
  \dfrac{\Delta(x)(\widehat{F}_{t}(x)-\widehat{F}_{mn+m+t}(x))}{2} &  \Delta(x)\dfrac{(\widehat{L}_{t}(x)-\widehat{L}_{mn+m+t}(x))}{2}\\
    \dfrac{(\widehat{L}_{t}(x)-\widehat{L}_{mn+m+t}(x))}{2} & \Delta(x)\dfrac{(\widehat{F}_{t}(x)-\widehat{F}_{mn+m+t}(x))}{2}\end{array}
	\right].
\end{align*}
	Hence, it gives that
	\begin{align*}
	&\displaystyle\sum_{j=0}^{j=n}\widehat{L}_{mj+t}(x)=\dfrac{ \Delta(x)^2}{D}\left[(\Delta (x)-\frac{\widehat{L}_{m}(x)}{2})(\widehat{L}_{t}(x)-\widehat{L}_{mn+m+t}(x))+\frac{  \widehat{F}_{m}(x)}{2}
(\widehat{F}_{t}(x)-\widehat{F}_{mn+m+t}(x)\right],\\
&\displaystyle\sum_{j=0}^{j=n}\widehat{F}_{mj+t}(x)=\dfrac{\Delta(x)^2}{D}\left[(\Delta (x)-\frac{\widehat{L}_{m}(x)}{2})(\widehat{F}_{t}(x)-\widehat{F}_{mn+m+t}(x))+\frac{\widehat{F}_{m}(x)}{2}
(\widehat{L}_{t}(x)-\widehat{L}_{mn+m+t}(x)\right].
\end{align*}
Using previous results, we get
\begin{align*}
&\displaystyle\sum_{j=0}^{j=n}\widehat{L}_{mj+t}(x)=\dfrac{\Delta (x)\widehat{L}_{t}(x)-\Delta(x)\widehat{L}_{mn+m+t}(x)+(-1)^m\Delta(x)(\widehat{L}_{mn+t}(x)-\widehat{L}_{t-m}(x))}{\Delta(x)+(-1)^m\Delta(x)-\widehat{L}_{m}(x)},\\
&\displaystyle\sum_{j=0}^{j=n}\widehat{F}_{mj+t}(x)=\dfrac{\Delta (x)\widehat{F}_{t}(x)-\Delta (x)\widehat{F}_{mn+m+t}(x)+(-1)^m\Delta(x) (\widehat{F}_{mn+t}(x)-\widehat{F}_{t-m}(x))}{\Delta(x)+(-1)^m\Delta(x)-\widehat{L}_{m}(x)}
\end{align*}
\end{proof}

\begin{theorem}
For $n\in N$ and $m$, $t\in Z$, we have
\begin{align*}
&\displaystyle\sum_{j=0}^{j=n}(-1)^j\widehat{L}_{mj+t}(x)=\dfrac{\Delta (x)\widehat{L}_{t}(x)-\Delta (x)\widehat{L}_{mn+m+t}(x)+(-1)^m\Delta (x)(\widehat{L}_{mn+t}(x)-\widehat{L}_{t-m}(x))}{\Delta(x) +(-1)^m \Delta (x)-\widehat{L}_{m}(x)},\\
&\displaystyle\sum_{j=0}^{j=n}(-1)^j\widehat{F}_{mj+t}(x)=\dfrac{\Delta (x)\widehat{F}_{t}(x)-\Delta (x)\widehat{F}_{mn+m+t}(x)+(-1)^m\Delta (x)(\widehat{F}_{mn+t}(x)-\widehat{F}_{t-m}(x))}{\Delta(x)+(-1)^m\Delta (x)-\widehat{L}_{m}(x)}.
\end{align*}
\end{theorem}
\begin{proof}
\textbf{Case:$1$} If $n$ is an even natural number then we have
\begin{align*}
\Delta(x)^2 I+\Delta(x)^2 (S(x)^m)^{n+1}=(\Delta (x)I+\Delta (x)S(x)^m)\displaystyle\sum_{j=0}^{j=n}(\Delta (x)S(x)^m)^j(-1)^j.
\end{align*}
It is clear that $\det(\Delta (x)I+\Delta(x) S(x)^m)\neq 0$, hence we get
\begin{align*}
&\Delta(x) I+\Delta(x) (S(x)^m)^{-1}(\Delta(x) I+\Delta(x)(S(x)^m)^{n+1})\Delta (x)S(x)^t\\&=\displaystyle\sum_{j=0}^{j=n}(-1)^j\Delta(x)(S(x)^{mj+t})={\left[
 \begin{array}{cc}
    \dfrac{1}{2}{\displaystyle\sum_{j=0}^{j=n}(-1)^j(\widehat{L}_{mj+t}(x))} & \dfrac{\Delta(x)}{2}{\displaystyle\sum_{j=0}^{j=n}(-1)^j(\widehat{F}_{mj+t}(x))}\\
    \dfrac{1}{2}{\sum_{j=0}^{j=n}(-1)^j(\widehat{F}_{mj+t}(x))} & \dfrac{1}{2}{\displaystyle\sum_{j=0}^{j=n}(-1)^j(\widehat{L}_{mj+t}(x))}
	\end{array}
	\right]}.
\end{align*}
For $m\neq 0$ take $ d=\Delta(x)  +\Delta(x)(-1)^m+\widehat{L}_{m}(x)$, we get
\begin{align*}
&(\Delta(x) I+\Delta (x)S(x)^m)^{-1}=\frac{1}{d}{\left[
 \begin{array}{cc}
   \Delta (x)+\frac{(\widehat{L}_{m}(x))}{2} & -\Delta(x)\frac{(\widehat{F}_{m}(x))}{2}\\
    \frac{(-\widehat{F}_{m}(x))}{2} & \Delta(x) +\frac{(\widehat{L}_{m}(x))}{2}
	\end{array}
	\right]}\\&=\frac{1}{d}\left[(\Delta (x)+\frac{(\widehat{L}_{m}(x))}{2})I-\frac{(\widehat{F}_{m}(x))}{2}R\right],\\&
(\Delta(x) I+\Delta (x)S(x)^m)^{-1}(\Delta(x)^2 S(x)^t+\Delta(x)^2 S(x)^{mn+m+t})\\&=\frac{1}{d}\left[(\Delta (x)+\frac{(\widehat{L}_{m}(x))}{2})I-\frac{(\widehat{F}_{m})}{2}R\right](\Delta(x)^2 S(x)^t+\Delta(x)^2 S(x)^{mn+m+t}).
	\end{align*}
Using previous corollary, we get
	\begin{align*}
&R(\Delta(x)^2 S(x)^t+\Delta(x)^2 S(x)^{mn+m+t})=\Delta (x){\left[
 \begin{array}{cc}
  \Delta(x)\dfrac{(\widehat{F}_{t}(x)+\widehat{F}_{mn+m+t}(x))}{2} &  \Delta(x)\dfrac{(\widehat{L}_{t}(x)+\widehat{L}_{mn+m+t}(x))}{2}\\
    \dfrac{(\widehat{L}_{t}(x)+\widehat{L}_{mn+m+t}(x))}{2} & \Delta(x)\dfrac{(\widehat{F}_{t}(x)+\widehat{F}_{mn+m+t}(x))}{2}
	\end{array}
	\right]}
\end{align*}
It follows that
\begin{align*}
&(\Delta (x)I+\Delta (x)S(x)^m)^{-1}(\Delta(x)^2 S(x)^t+\Delta(x)^2 S(x)^{mn+m+t})\\&=\dfrac{\Delta(x)}{d}(\Delta (x)+\dfrac{\widehat{L}_{m}(x)}{2})\left[\begin{array}{cc}
  \Delta(x)\dfrac{(\widehat{L}_{t}(x)+\widehat{L}_{mn+m+t}(x))}{2} &  \Delta(x)\dfrac{(\widehat{F}_{t}(x)+\widehat{F}_{mn+m+t}(x))}{2}\\
    \dfrac{(\widehat{F}_{t}(x)+\widehat{F}_{mn+m+t}(x))}{2} & \dfrac{(\widehat{L}_{t}(x)+\widehat{L}_{mn+m+t}(x))}{2}
	\end{array}
	\right]\\&
	-\dfrac{\Delta (x)\widehat{F}_{m}(x)}{2d}\left[
 \begin{array}{cc}
  \Delta(x)\dfrac{\Delta(x)(\widehat{F}_{t}(x)+\widehat{F}_{mn+m+t}(x))}{2} &  \Delta(x)\dfrac{(\widehat{L}_{t}(x)+\widehat{L}_{mn+m+t}(x))}{2}\\
    \dfrac{(\widehat{L}_{t}(x)+\widehat{L}_{mn+m+t}(x))}{2} & \Delta(x)\dfrac{(\widehat{F}_{t}(x)+\widehat{F}_{mn+m+t}(x))}{2}\end{array}
	\right].
\end{align*}
Thus, it gives that
\begin{align*}
\sum_{j=0}^{j=n}(-1)^j\widehat{L}_{mj+t}(x)=\dfrac{\Delta(x)^2}{d}\left[(\Delta (x)+\frac{\widehat{L}_{m}(x)}{2})(\widehat{L}_{t}(x)+\widehat{L}_{mn+m+t}(x))-\frac{\widehat{F}_{m}(x)\Delta(x)}{2}(\widehat{F}_{t}(x)+\widehat{F}_{mn+m+t}(x)\right],\\
\sum_{j=0}^{j=n}(-1)^j\widehat{F}_{mj+t}(x)=\dfrac{\Delta(x)^2}{d}\left[(\Delta (x)+\frac{\widehat{L}_{m}(x)}{2})(\widehat{F}_{t}(x)+\widehat{F}_{mn+m+t}(x))-\frac{\widehat{F}_{m}(x)\Delta(x)}{2}(\widehat{L}_{t}(x)+\widehat{L}_{mn+m+t}(x)\right].
\end{align*}
After simplifications, we get
\begin{align*}
&\sum_{j=0}^{j=n}(-1)^j\widehat{L}_{mj+t}(x)=\dfrac{\Delta (x)\widehat{L}_{t}(x)-\Delta (x)\widehat{L}_{mn+m+t}(x)+(-1)^m\Delta (x)(\widehat{L}_{mn+t}(x)-\widehat{L}_{t-m}(x))}{\Delta (x)+(-1)^m \Delta (x)-\widehat{L}_{m}(x)},\\&
\sum_{j=0}^{j=n}(-1)^j\widehat{F}_{mj+t}(x)=\dfrac{\Delta (x)\widehat{F}_{t}(x)-\Delta (x)\widehat{F}_{mn+m+t}(x)+(-1)^m\Delta (x)(\widehat{F}_{mn+t}(x)-\widehat{F}_{t-m}(x))}{\Delta(x)+(-1)^m\Delta (x)-\widehat{L}_{m}(x)}.
\end{align*}
\textbf{Case:$2$ }If $n$ is an odd natural number, then we have
\begin{align*}
\sum_{j=0}^{j=n}(-1)^j\widehat{L}_{mj+t}(x)=\sum_{j=0}^{j=n-1}(-1)^j\widehat{L}_{mj+t}(x)-\widehat{L}_{mn+t}(x).
\end{align*}
Since $n$ is an odd natural number then $(n-1)$ is an even number, hence using case(1), it follows that
\begin{align*}
&\sum_{j=0}^{j=n}(-1)^j\widehat{L}_{mj+t}(x)=\frac{\Delta (x)\widehat{L}_{t}(x)+\Delta (x)\widehat{L}_{mn+t}(x)+(-1)^m\Delta (x)(\widehat{L}_{mn-m+t}(x)+\widehat{L}_{t-m}(x))}{\Delta (x)+(-1)^m\Delta (x)+\widehat{L}_{m}(x)}-\widehat{L}_{mn+t}(x)\\
&=\frac{\Delta (x)\widehat{L}_{t}(x)+(-1)^m\Delta(x)(\widehat{L}_{mn-m+t}(x)+\widehat{L}_{t-m}(x))-(-1)^m\Delta (x)\widehat{L}_{mn+t}(x)-\Delta (x)\widehat{L}_{m}(x)\widehat{L}_{mn+t}(x)}{\Delta (x)+(-1)^m\Delta (x)+\widehat{L}_{m}(x)}.
\end{align*}
After simplifications, we get
\begin{align*}
&\sum_{j=0}^{j=n}(-1)^j\widehat{L}_{mj+t}(x)=\frac{\Delta (x)\widehat{L}_{t}(x)+\Delta (x)\widehat{L}_{mn-m+t}(x)+(-1)^m\Delta (x)(\widehat{L}_{t-m}(x)-\widehat{L}_{mn+t}(x))}{\Delta (x)+(-1)^m\Delta (x)+\widehat{L}_{m}(x)},\\&
\sum_{j=0}^{j=n}(-1)^j\widehat{F}_{mj+t}(x)=\frac{\Delta (x)\widehat{F}_{t}(x)-\Delta (x)\widehat{L}_{mn+m+t}(x)+(-1)^m\Delta (x)(\widehat{F}_{t-m}(x)-\widehat{F}_{mn+t}(x))}{\Delta (x)+(-1)^m\Delta (x)+\widehat{L}_{m}(x)}.
\end{align*}
\end{proof}
\subsection{Properties of Generalized Fibonacci Polynomials $\widehat{F}_{ n}(x)$ and $\widehat{L}_{n}(x)$ using Vector Methods}
In this section, we derive some properties of generalized Fibonacci polynomials  $\widehat{F}_{ n}(x)$ and $\widehat{L}_{n}(x)$ using vector methods.
 \begin{definition}{\textbf{(Generalized Fibonacci Polynomial Vector of Length $d$ )}}\\
 For all integers $n$, generalized Fibonacci polynomial vector $\stackrel{\rightarrow}{F_{n}^d}(x)$ and generalized Lucas polynomial  vector $\stackrel{\rightarrow}{L_{n}^d}(x)$ of length $d$ are defined as follows
 
 \begin{center}
   $ \stackrel{\rightarrow}{F_{n}^d}(x) = 
 \left[\begin{array}{c}
           \bigskip 
           \widehat{F}_{n}(x)\\
            \bigskip 
	 	   \widehat{F}_{n+1}(x)\\
	 	    \bigskip 
	       \widehat{F}_{n+2}(x)\\
	        \bigskip 
	       \vdots\\
	        \bigskip 
	   \widehat{F}_{n+d-1}(x)
\end{array}
\right]$ and $ \stackrel{\rightarrow}{L_{n}^d}(x) = 
 \left[\begin{array}{c}
           \bigskip 
           \widehat{L}_{n}(x)\\
            \bigskip 
	 	  \widehat{L}_{n+1}(x)\\
	 	    \bigskip 
	    \widehat{L}_{n+2}(x)\\
	        \bigskip 
	       \vdots\\
	        \bigskip 
	       \widehat{L}_{n+d-1}(x)
\end{array}
\right]$.
 \end{center}
 \end{definition}
  \begin{definition}{\textbf{(Reverse Generalized Fibonacci Polynomial Vector of Length $d$)}}\\
 For all integers $n$, Reverse generalized Fibonacci polynomial vector $\stackrel{\rightarrow}{f_{n}^d}(x)$ and Reverse generalized Lucas polynomial  vector $\stackrel{\rightarrow}{l_{n}^d}(x)$ of length $d$ are defined as follows
  \begin{center}
   $\stackrel{\rightarrow}{f_{n}^d}(x) = 
 \left[\begin{array}{c}
           \bigskip 
           \widehat{L}_{n+d-1}(x)\\
            \bigskip 
	 	   \widehat{L}_{n+d-2}(x)\\
	 	    \bigskip 
	       \widehat{L}_{n+d-3}(x)\\
	        \bigskip 
	       \vdots\\
	        \bigskip 
	 \widehat{L}_{n}(x)
\end{array}
\right]$ and $\stackrel{\rightarrow}{l_{n}^d}(x)= 
 \left[\begin{array}{c}
           \bigskip 
         \widehat{L}_{n+d-1}(x)\\
            \bigskip 
	 	   \widehat{L}_{n+d-2}(x)\\
	 	    \bigskip 
	     \widehat{L}_{n+d-3}(x)\\
	        \bigskip 
	       \vdots\\
	        \bigskip 
	    \widehat{L}_{n}(x)
\end{array}
\right]$.
 \end{center}
 \end{definition} 
   \begin{definition}{\textbf{(Vectors $\stackrel{\rightarrow}{a}$, $\stackrel{\rightarrow}{b} $, $\stackrel{\rightarrow}{c}$ and $\stackrel{\rightarrow}{d}$)}}
   
\noindent For all integers $n$, the vectors $\stackrel{\rightarrow}{a}$, $\stackrel{\rightarrow}{b} $, $\stackrel{\rightarrow}{c}$ and $\stackrel{\rightarrow}{d}$ of length $d$ are defined as follows
  \begin{center}
  $ \stackrel{\rightarrow}{a} = 
 \left[\begin{array}{c}
           \bigskip 
           1\\
            \bigskip 
	 	   r_1(x)\\
	 	    \bigskip 
	       r_1^2(x)\\
	        \bigskip 
	       \vdots\\
	        \bigskip 
	       r_1^{d-2}(x)\\
	       \bigskip 
	       r_1^{d-1}(x)
\end{array}
\right]$, $ \stackrel{\rightarrow}{b} = 
 \left[\begin{array}{c}
           \bigskip 
           1\\
            \bigskip 
	 	   r_2(x)\\
	 	    \bigskip 
	       r_2^2(x)\\
	        \bigskip 
	       \vdots\\
	        \bigskip 
	       r_2^{d-2}(x)\\
	       \bigskip 
	       r_2^{d-1}(x)
\end{array}
\right]$, $ \stackrel{\rightarrow}{c} = 
 \left[\begin{array}{c}
           \bigskip 
           r_1^{d-1}(x)\\
            \bigskip 
	 	   r_1^{d-2}(x)\\
	 	    \bigskip 
	       r_1^{d-3}(x)\\
	        \bigskip 
	       \vdots\\
	        \bigskip 
	       r_1(x)\\
	       \bigskip 
	       1
\end{array}
\right]$ and $ \stackrel{\rightarrow}{d} = 
 \left[\begin{array}{c}
           \bigskip 
           r_2^{d-1}(x)\\
            \bigskip 
	 	   r_2^{d-2}(x)\\
	 	    \bigskip 
	       r_2^{d-3}(x)\\
	        \bigskip 
	       \vdots\\
	        \bigskip 
	       r_2(x)\\
	       \bigskip 
	       1
\end{array}
\right]$.
 \end{center}
 \end{definition}
   \begin{definition} Define $d \times d$ matrix $T$ by
 \begin{center}
 $ \begin{bmatrix}
 0 & 1 & 0 & 0 & 0 & \cdots & 0 \\ 
  0 & 0 & 1 & 0 & 0 & \cdots & 0 \\ 
  0 & 0 & 0 & 1 & 0 & \cdots & 0 \\ 
 \cdots & \cdots & \cdots & \cdots & \cdots & \cdots & \cdots \\ 
 0 & \cdots & 0 & 0 & 0 & 0 & 1 \\ 
  0 & \cdots & 0 & 1 & 0 & 1 & x\\ 
 \end{bmatrix}_{d \times d}$.
 \end{center} 
  \end{definition}
  \begin{definition} Define $d \times d$ matrix $S$ by
 \begin{center}
 $ \begin{bmatrix}
 x & 1 & 0 & 0 & 0 & \cdots & 0 \\ 
  1 & 0 & 0 & 0 & 0 & \cdots & 0 \\ 
  0 & 1 & 0 & 0 & 0 & \cdots & 0 \\ 
 \cdots & \cdots & \cdots & \cdots & \cdots & \cdots & \cdots \\ 
 0 & \cdots & 0 & 0 & 1 & 0 & 0 \\ 
  0 & \cdots & 0 & 0 & 0 & 1 & 0\\ 
 \end{bmatrix}_{d \times d}$.
 \end{center} 
  \end{definition}
  \noindent The matrices $T$ and $S$ have characteristic polynomial $r^{(m-2)}(r^2 - xr - 1)$. The non
zero eigenvalues of both are $r_1(x)$ and $r_2(x)$ and eigenspace associated with $r_1(x)$ and $r_2(x)$ for $T$ are spanned
respectively by $ \stackrel{\rightarrow}{a}$, $ \stackrel{\rightarrow}{b}$ and for $S$ are spanned respectively by $ \stackrel{\rightarrow}{c}$ and $ \stackrel{\rightarrow}{d}$.
\begin{theorem}({ Binet form for $\stackrel{\rightarrow}{F_{n}^d}(x)$, $\stackrel{\rightarrow}{L_{n}^d}(x)$, $\stackrel{\rightarrow}{f_{n}^d}(x)$ and $\stackrel{\rightarrow}{l_{n}^d}(x)$ })
For all integers $n$
  \begin{align*}
  &\stackrel{\rightarrow}{F_{n}^d}(x) = [r_1(x) - r_2(x)][r_1^n(x) \stackrel{\rightarrow}{a}  - r_2^n(x) \stackrel{\rightarrow}{b}], \\
  &\stackrel{\rightarrow}{L_{n}^d}(x) = [r_1(x) - r_2(x)]^2[r_1^n(x) \stackrel{\rightarrow}{a}  + r_2^n(x) \stackrel{\rightarrow}{b}], \\
  &\stackrel{\rightarrow}{f_{n}^d}(x) = [r_1(x) - r_2(x)][r_1^n(x) \stackrel{\rightarrow}{c}  - r_2^n(x) \stackrel{\rightarrow}{d}],\\ 
 & \stackrel{\rightarrow}{l_{n}^d}(x) = [r_1(x) - r_2(x)]^2[r_1^n(x) \stackrel{\rightarrow}{c}  + r_2^n(x) \stackrel{\rightarrow}{d}].
 \end{align*}
 \end{theorem} 
  \begin{theorem}
 For all integers $n$
  \begin{align*}
  &\stackrel{\rightarrow}{F_{n+1}^d}(x) =T \stackrel{\rightarrow}{F_{n}^d}(x),\\ 
  &\stackrel{\rightarrow}{L_{n+1}^d}(x)=T \stackrel{\rightarrow}{L_{n}^d}(x), \\
 & \stackrel{\rightarrow}{f_{n+1}^d}(x) =S \stackrel{\rightarrow}{F_{n}^d}(x),\\ 
  &\stackrel{\rightarrow}{l_{n+1}^d}(x) =S \stackrel{\rightarrow}{F_{n}^d}(x). 
 \end{align*}
 \end{theorem} 
 \begin{theorem}
For all integers $t\geq n + 1$
  \begin{align*}
   &\stackrel{\rightarrow}{F_{n+1}^d}(x) =T^{(n-t+1)} \stackrel{\rightarrow}{F_{n}^d}(x),\\ 
  & \stackrel{\rightarrow}{L_{n+1}^d}(x) =T^{(n-t+1)} \stackrel{\rightarrow}{L_{n}^d}(x),\\
 & \stackrel{\rightarrow}{f_{n+1}^d}(x) =S^{(n-t+1)} \stackrel{\rightarrow}{F_{n}^d}(x),\\
  &\stackrel{\rightarrow}{l_{n+1}^d}(x) =S^{(n-t+1)} \stackrel{\rightarrow}{F_{n}^d}(x). 
\end{align*}
\end{theorem}
\begin{theorem} For $k > 0$
\begin{align*}
&\stackrel{\rightarrow}{a}\cdot  \stackrel{\rightarrow}{a} = \stackrel{\rightarrow}{c}\cdot  \stackrel{\rightarrow}{c} = \Vert \stackrel{\rightarrow}{a} \Vert^2 = \Vert \stackrel{\rightarrow}{c} \Vert^2  =
 \begin{cases}
 \dfrac{\widehat{F}_{n}(x)r_1^{d-1}(x)}{(r_1(x) - r_2(x))}, & \text {if $d$ is even};\\
\dfrac{\widehat{L}_{n}(x)r_1^{d-1}(x)}{(r_1(x) - r_2(x))^2}, & \text{if $d$ is odd}
 \end{cases},\\
 &\stackrel{\rightarrow}{b}\cdot  \stackrel{\rightarrow}{b}=\stackrel{\rightarrow}{d}\cdot  \stackrel{\rightarrow}{d}  =  \Vert \stackrel{\rightarrow}{b} \Vert^2 = \Vert \stackrel{\rightarrow}{d} \Vert^2 =
 \begin{cases}
 \dfrac{- \widehat{F}_{n}(x)r_2^{d-1}(x)}{(r_1(x) - r_2(x))}, & \text {if $d$ is even};\\
\dfrac{\widehat{L}_{n}(x)r_2^{d-1}(x)}{(r_1(x) - r_2(x))^2}, & \text{if $d$ is odd}
 \end{cases},\\
&\stackrel{\rightarrow}{a}\cdot  \stackrel{\rightarrow}{b} = \stackrel{\rightarrow}{c}\cdot  \stackrel{\rightarrow}{d}  =
 \begin{cases}
 0, & \text {If $d$ is even};\\
 1, & \text{If $d$ is odd}
 \end{cases}.
\end{align*}
\end{theorem}
\begin{theorem}
For all integer $d > 0$ and for integers $n_1, n_2$, we have
\begin{align*}
&\stackrel{\rightarrow}{F_{n_1}^d}(x)\cdot  \stackrel{\rightarrow}{F_{n_2}^d}(x) = \stackrel{\rightarrow}{f_{n_1}^d}(x)\cdot  \stackrel{\rightarrow}{f_{n_2}^d}(x) =
 \begin{cases}
 \dfrac{\widehat{F}_{d}(x)\widehat{F}_{d+n_1+n_2-1}(x)}{(r_1(x) - r_2(x))^2}, & \text {if $d$ is even};\\
 \widehat{L}_{d}(x)\widehat{L}_{d+n_1+n_2-1}(x)-\dfrac{(-1)^{n_1}\widehat{L}_{n_2-n_1}(x)}{(r_1(x) - r_2(x))}, & \text{if $d$ is odd}
 \end{cases},\\
&\stackrel{\rightarrow}{L_{n_1}^d}(x)\cdot  \stackrel{\rightarrow}{L_{n_2}^d}(x) = \stackrel{\rightarrow}{l_{n_1}^d}(x)\cdot  \stackrel{\rightarrow}{l_{n_2}^d}(x) =
 \begin{cases}
 \widehat{F}_{d}(x)\widehat{F}_{d+n_1+n_2-1}(x), & \text {if $d$ is even};\\
 \dfrac{\widehat{L}_{d}(x)\widehat{L}_{d+n_1+n_2-1}(x)}{(r_1(x) - r_2(x))^2}+\dfrac{(-1)^{n_1}\widehat{L}_{n_2-n_1}(x)}{(r_1(x) - r_2(x))}, & \text{if $d$ is odd}
 \end{cases},\\
&\stackrel{\rightarrow}{F_{n_1}^d}(x)\cdot  \stackrel{\rightarrow}{L_{n_2}^d}(x) = \stackrel{\rightarrow}{f_{n_1}^d}(x)\cdot  \stackrel{\rightarrow}{l_{n_2}^d}(x) =
 \begin{cases}
  \widehat{F}_{d}(x)\widehat{L}_{d+n_1+n_2-1}(x), & \text {if $d$ is even};\\
  \dfrac{\widehat{L}_{d}(x)\widehat{F}_{d+n_1+n_2-1}(x)}{(r_1(x) - r_2(x))^2}+\dfrac{(-1)^{n_1}\widehat{F}_{n_2-n_1}(x)}{(r_1(x) - r_2(x))}, & \text{if $d$ is odd}
 \end{cases}.
 \end{align*}
\end{theorem}
\begin{theorem} For all integer $d > 0$ and for integer $n$, we have
\begin{align*}
&\Vert \stackrel{\rightarrow}{F_{n_1}^d}(x)\Vert^2 = \Vert \stackrel{\rightarrow}{f_{n_1}^d}(x)\Vert^2 =
 \begin{cases}
  \widehat{F}_{d}(x)\widehat{F}_{d+2n-1}(x), & \text {if $d$ is even};\\
\widehat{L}_{d}(x)\widehat{L}_{d+2n-1}(x)-\dfrac{2(-1)^{n}}{(r_1(x) - r_2(x))^2}, & \text{if $d$ is odd}
 \end{cases},\\ 
&\Vert\stackrel{\rightarrow}{L_{n_1}^d}(x)\Vert^2 = \Vert\stackrel{\rightarrow}{l_{n_1}^d}(x)\Vert^2=
 \begin{cases}
 \widehat{F}_{d}(x)\widehat{F}_{d+2n-1}(x), & \text {if $d$ is even};\\
\widehat{L}_{d}(x)\widehat{L}_{d+2n-1}(x)+\dfrac{2(-1)^{n}}{(r_1(x) - r_2(x))^2}, & \text{if $d$ is odd}
 \end{cases}.
 \end{align*}
\end{theorem}
\begin{theorem}
For all integer $d > 0$ and for integers $n_1, n_2$, we have
\begin{align*}
&\sum_{i=0}^{i=d-1}\widehat{F}_{n_1+i}(x) \widehat{F}_{n_2+i}(x)      =
 \begin{cases}
  \widehat{F}_{d}(x) \widehat{F}_{d+n_1+n_2-1}(x), & \text {if $d$ is even};\\
 \widehat{L}_{d}(x)\widehat{L}_{d+n_1+n_2-1}(x)-\dfrac{(-1)^{n_1}\widehat{L}_{n_2-n_1}(x)}{(r_1(x) - r_2(x))}, & \text{if $d$ is odd}
 \end{cases}, \\
&\sum_{i=0}^{i=d-1}\widehat{L}_{n_1+i}(x)\widehat{L}_{n_2+i}(x) =
 \begin{cases}
(r_1(x) - r_2(x))^2  \widehat{F}_{d}(x) \widehat{F}_{d+n_1+n_2-1}(x), & \text {if $d$ is even};\\
 \widehat{L}_{d}(x)\widehat{L}_{d+n_1+n_2-1}(x)+\dfrac{(-1)^{n_1}\widehat{L}_{n_2-n_1}(x)}{(r_1(x) - r_2(x))}, & \text{if $d$ is odd}
 \end{cases},\\
 &\sum_{i=0}^{i=d-1} \widehat{F}_{n_1+i}(x) \widehat{L}_{n_2+i}(x) =
 \begin{cases}
 (r_1(x) - r_2(x))^2  \widehat{F}_{d}(x) \widehat{L}_{d+n_1+n_2-1}(x), & \text {if $d$ is even};\\
  \widehat{L}_{d}(x)\widehat{F}_{d+n_1+n_2-1}(x)+\dfrac{(-1)^{n_1}\widehat{F}_{n_2-n_1}(x)}{(r_1(x) - r_2(x))}, & \text{if $d$ is odd}
 \end{cases}.
 \end{align*}
\end{theorem}
\begin{corollary} For all integer $d > 0$ and for integer $n$, we have
\begin{align*}
&\sum_{i=0}^{i=d-1}{\widehat{F}_{n}(x)}^2 = 
 \begin{cases}
 \widehat{F}_{d}(x)\widehat{F}_{d+2n-1}(x), & \text {if $d$ is even};\\
 \dfrac{\widehat{L}_{d}(x)\widehat{L}_{d+2n-1}(x)}{(r_1(x) - r_2(x))^2}-2(-1)^{n}, & \text{if $d$ is odd}
 \end{cases},\\
&\sum_{i=0}^{i=d-1}{\widehat{L}_{n}(x)}^2 = 
 \begin{cases}
 \widehat{F}_{d}(x)\widehat{F}_{d+2n-1}(x), & \text {if $d$ is even};\\
 \widehat{L}_{d}\widehat{L}_{d+2n-1}(x)+2(-1)^{n_1})(r_1(x) - r_2(x))^2, & \text{if $d$ is odd}
 \end{cases}.
 \end{align*}
\end{corollary}
  \begin{corollary} For all integer $d > 0$ and for all integer $t$ and $n$, we have
\begin{align*}
&\widehat{F}_{n+d-2}(x)\widehat{F}_{n+d-t-1}(x)+\widehat{F}_{n-1}(x)\widehat{F}_{n-t}(x) =\\ &
 \begin{cases}
 (r_1(x) + r_2(x)) \widehat{F}_{d-1}(x)\widehat{F}_{d+2n-t-2}(x), & \text {if $d$ is even};\\
\dfrac{(r_1(x) + r_2(x))^2}{({r_1(x)}^2 - {r_2(x)}^2)} \widehat{L}_{d-1}(x)\widehat{L}_{d+2n-t}(x)-\dfrac{2(-1)^{n-t}\widehat{L}_{t-1}(x)}{(r_1(x) - r_2(x))}, & \text{if $d$ is odd}
 \end{cases},\\&
\widehat{L}_{n+d-2}(x)\widehat{L}_{n+d-t-1}(x)+\widehat{L}_{n-1}(x)\widehat{L}_{n-t}(x) =\\ &
 \begin{cases}
 (r_1(x) + r_2(x))(r_1(x) - r_2(x))^2\widehat{L}_{d-1}(x)\widehat{L}_{d+2n-t-2}(x), & \text {if $d$ is even};\\
(r_1(x) + r_2(x))\widehat{L}_{d-1}(x)\widehat{L}_{d+2n-t}(x)+2(-1)^{n-1}\widehat{L}_{t+1}(x)(r_1(x) - r_2(x)), & \text{if $d$ is odd}
 \end{cases},\\&
 \widehat{F}_{n+d-2}(x)\widehat{L}_{n+d-t-1}(x)+\widehat{F}_{n-1}(x)\widehat{L}_{n-t}(x) = \\&
 \begin{cases}
 (r_1(x) + r_2(x))(r_1(x) - r_2(x))^2 \widehat{F}_{d-1}(x)\widehat{L}_{d+2n-t-2}(x), & \text {if $d$ is even};\\
(r_1(x) + r_2(x)) \widehat{L}_{d-1}(x)\widehat{F}_{d+2n-t}(x)+2(-1)^{n-1}\widehat{F}_{t+1}(x)(r_1(x) - r_2(x)), & \text{if $d$ is odd}
 \end{cases}.
 \end{align*}
\end{corollary}
\section{Concluding Remarks} In this chapter, we have generalized and derived some identities of $M_{n}(x)$, $\widehat{F}_{n}(x)$ and $\widehat{L}_{n}(x)$ using matrix and vector methods.